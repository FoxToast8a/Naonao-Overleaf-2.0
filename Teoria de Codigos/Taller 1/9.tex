%!TEX root = main.tex

Resuelva el ejercicio 1.5.1 de las notas de clase.
Construya un código que no sea instantáneo cuyas longitudes  de palabra cumplen la desigualdad de Kraft.
\begin{sols}
Sea $C=\{1,10,100,1000\}$. Este código claramente no es instantáneo, ya que no es prefijo, por que la palabra 1 es segmento inicial de todas la demas palabras del código. Para este código tenemos que $D=2$ ya que es un código binario, y tenemos longitudes de palabra $\ell_1=1,\ell_2=2,\ell_3=3$ y $\ell_4=4.$ Así
\begin{align*}
       \sum_{i=1}^4 D^{-\ell_i}&=2^{-1}+2^{-2}+2^{-3}+2^{-4}\\
       &=2^{-4}(2^3+2^2+2+1)\\
       &=\frac{8+4+2+1}{16}\\
       &=\frac{15}{16}\leq 1.
   \end{align*} 

   De esta forma $C$ es un código no instantáneo que cumple la desigualdad de Kraft como queriamos.  
\end{sols}