%!TEX root = main.tex
Considere que una fuente sin memoria emite los símbolos del alfabeto $S = \{a, b, c\}$. Se sabe que la distribución de probabilidad de los símbolos en la salida del canal es 
\[
P_Y = \{0\text{.}6, 0\text{.}3, 0\text{.}1\}.
\]
Los símbolos entran a un canal ruidoso que puede cambiar unos símbolos en otros de acuerdo a la distribución de probabilidad que muestra la siguiente tabla de probabilidades condicionales $p(x|y)$:

\[
\begin{array}{c|ccc}
\text{entra} \backslash \text{sale} & a & b & c \\
\hline
a & 0\text{.}95 & 0\text{.}03 & 0\text{.}02 \\
b & 0\text{.}03 & 0\text{.}95 & 0\text{.}02 \\
c & 0\text{.}02 & 0\text{.}02 & 0\text{.}96 \\
\end{array}
\]

Denote por $X$ y $Y$ las variables aleatorias que modelan los símbolos que salen de la fuente y del canal respectivamente. Resuelva las siguientes preguntas:

\begin{enumerate}
    \item[A]. Determine las entropías de la fuente y el canal.
    \begin{sol}
    Para determinar las entropías que salen del canal y de la fuente necesitamos conocer los valores de $P_X$ y $P_Y$ respectivamente, como el ejercicio nos da explicitamente $P_Y$ vamos a comenzar calculando la entropía de los símbolos que salen del canal, entonces
    \begin{align*}
        H(Y)=-0.6 log_2 0.6-0.3 log_2 0.3-0.1 log_2 0.1\approx 1.29546\text{bits} 
    .\end{align*}
    La entropía de los símbolos que salen la fuente no la obtenemos directamente puesto que no contamos con la distribución de manera explícita , por lo cual debemos calcular $P_X$, de la siguiente manera,
    \begin{align*}
        \begin{array}{ccc}
p(a, a) = p(a|a) P_Y(a)  & p(a, b) = p(a|b) P_Y(b)  & p(a, c) = p(a|c) P_Y(c)\\
\hspace{0.8cm}= 0.95 \cdot 0.6 & \hspace{0.8cm}= 0.03 \cdot 0.3 &\hspace{0.8cm}= 0.02 \cdot 0.1\\
 = 0.57   &\hspace{0.2cm}= 0.009   &\hspace{0.2cm}= 0.002 \\
 & & &
p(b, a) = p(b|a) P_Y(a)   & p(b, b) = p(b|b) P_Y(b) & p(b, c) = p(b|c) P_Y(c)
& \hspace{0.8cm}= 0.03 \cdot 0.6 &\hspace{0.8cm}= 0.95 \cdot 0.3  &\hspace{0.8cm}= 0.02 \cdot 0.1
&\hspace{0.2cm}= 0.018   &\hspace{0.2cm}= 0.285  &\hspace{0.2cm}= 0.002 \\
& & &
p(c, a) = p(c|a) P_Y(a) & p(c, b) = p(c|b) P_Y(b)  & p(c, c) = p(c|c) P_Y(c)\\
 \hspace{0.8cm}= 0.02 \cdot 0.6 &\hspace{0.8cm} = 0.02 \cdot 0.3 &\hspace{0.8cm}= 0.96 \cdot 0.1 \\
\hspace{0.2cm} = 0.012  &\hspace{0.2cm}= 0.006 &\hspace{0.2cm}= 0.096
\end{array}
    .\end{align*}

Por lo cual tenemos que 
\begin{align*}
    P_X=\{0\text{.}581, 0\text{.}305, 0\text{.}0114\},
\end{align*}
para a,b y c respectivamente.
Por lo cual la entropía de la fuente es 
\begin{align*}
    H(X) =-(0.581)log(0.581) - (0.305)log(0.305) - (0.114)log(0.114) = 1.3348\text{ bits}
.\end{align*}
\textbf{Nota: }Como $H(Y|X)$ se puede tomar como entropía del canal, específicamente del ruido, también se podría calcular en este numeral sin embargo la calcularemos en el item $C$ puesto que es ahí donde se pide de manera explícita.
    \end{sol}
    \item[B]. Determine la entropía del sistema.\\
\begin{sol}
Utilizando las probabilidades que calculamos en el numeral anterior, podemos calcular la entropía del sistema por definición, de la siguiente manera,
    \begin{align*}
        H(X, Y) &= - \sum_{x,y} p(x, y) \log_2 p(x, y)\\
        &=-0.570 \log_2 (0.570)- 0.009 \log_2 (0.009)-2(0.002 \log_2 (0.002))-0.018 \log_2 (0.018) \\
        &\hspace{0.5cm}- 0.285 \log_2 (0.285)- 0.012 \log_2 (0.012)- 0.006 \log_2 (0.006) + 0.096 \log_2 (0.096) \\
        &=1.62517 \text{ bits}
 .\end{align*}
\end{sol}
\item[C]. Determine e interprete las entropías condicionales $H(X|Y)$ y $H(Y|X)$.\\
\begin{sol}
Para resolver este numeral vamos a utilizar dos propiedades de las entropías que se tienen para cualesquiera dos variables aleatorias $X,Y$, estas dicen que 
\begin{align*}
    H(X,Y)&=H(X)+H(Y|X)\\
     H(X,Y)&=H(Y)+H(X|Y)
.\end{align*}
Por lo cual $H(Y|X)$ es igual a 
\begin{align*}
    H(Y|X)&=1.62517-1.3348\\
    &=0.29037\text{ bits}
.\end{align*}
de manera similar, para $H(X|Y)$  se tiene que
\begin{align*}
    H(X|Y)&=1.62517- 1.29546\\
    &= 0.32971\text{ bits}
.\end{align*}
Podemos decir que la probabilidad condicional $H(Y|X)$ calcula el ruido que se puede encontrar en el canal de información puesto que calcula las probabilidades de entrada dado que ya se tiene una salida. Mientras que la probabilidad condicional $H(X|Y)$ se refiere a la información que se pierde en el canal, esta va orientada a dado que un símbolo entra al canal como podemos suponer que va a salir del mismo.

\end{sol}

    \item[D]. Calcule la información común entre la entrada y la salida del sistema.\\
    \begin{sol}

    Por definición tenemos que la información común entre la entrada y la salida del sistema, está dada por

    \begin{align*}
        I(X;Y) = \sum_{x,y} p(x, y) \log_2 \left( \frac{p(x, y)}{p(x) p(y)}\right)
    .\end{align*}
    Sin embargo, podemos utilizar el numeral $A$ y $C$ y la siguiente propiedad para dos variables aleatorias cualesquiera,
    \begin{align*}
        I(X;Y) =H(X)-H(X|Y)
    .\end{align*}
    Luego reemplazando valores tenemos que 
    \begin{align*}
        I(X;Y) &=1.3348-0.32971\\
        &=1.00509 \text{ bits}
    .\end{align*}
    \end{sol}


    \item[E]. Determine cuánta información se perdió dentro del canal.

\begin{sol}
La información que se perdió en el canal de información está dada por la probabilidad condicional $H(X|Y)$, la cual, calculamos en el numeral $C$ nos dió como resultado $0,32971 \text{ bits}$

\end{sol}

    \item[F]. Determine la pérdida de información por símbolo.
  \begin{sol}
Para este punto vamos a utilizar $H(X|Y)$, sin embargo, como nos piden que la calculemos por símbolo lo haremos para cuando $Y$ toma los valores $a,b$ y $c$.\\
 Para el caso en el que $Y=a$, calcularemos $H(X|a)$
    \begin{align*}
    H(X| a) &= - \sum_{x} p(x|a) \log_2 p(x|a) \\
    &= -(0.95) \log_2 (0.95) - (0.03) \log_2 (0.03) - (0.02) \log_2 (0.02) \\
    &= 0.33494 \, \text{bits}
    \end{align*}

En el caso donde $Y=b$ tenemos que calcular $H(X| b)$ la cual nos da
    \begin{align*}
    H(X| b) &= - \sum_{x} p(x|b) \log_2 p(x|b) \\
    &= -(0.03) \log_2 (0.03) - (0.95) \log_2 (0.95) - (0.02) \log_2 (0.02) \\
    &=0.33494 \, \text{bits}
    \end{align*}

Por último tenemos que cuando $Y=c$, se tiene que calcular $H(X| c)$, por lo tanto
    \begin{align*}
    H(X|c) &= - \sum_{x} p(x|c) \log_2 p(x|c) \\
    &= -(0.02) \log_2 (0.02) - (0.02) \log_2 (0.02) - (0.96) \log_2 (0.96) \\
    &= 0.28229\, \text{bits}
    \end{align*}


\end{sol}  
\end{enumerate}


