%!TEX root = ../main.tex

Una fuente $F$ genera símbolos de un alfabeto $\{a, b, c\}$ y se modela como un proceso markoviano $\{X_i\}_{i \in \mathbb{N}}$ con matriz de transición
\begin{equation*}
    P = \begin{bmatrix} 0\text{.}6 & 0\text{.}3 & 0\text{.}1 \\ 0\text{.}2 & 0\text{.}5 & 0\text{.}3 \\ 0\text{.}1 & 0\text{.}4 & 0\text{.}5 \end{bmatrix}
\end{equation*}

\begin{itemize}
    \item[A)] Determine la distribución de probabilidad para las variables aleatorias $X_i$.
    \begin{sol}
    Para encontar la distribución de probabilidad, como este es un proceso markoviano podemos plantear la siguiente igualdad.\\

$$
\begin{pmatrix}
p(a)
p(b) 
p(c)
\end{pmatrix}
=
\begin{pmatrix}
p(a) 
p(b) 
p(c)
\end{pmatrix}
\begin{pmatrix}
0.6 & 0.3 & 0.1 \\
0.2 & 0.5 & 0.3 \\
0.1 & 0.4 & 0.5
\end{pmatrix}
$$

Luego podemos plantear el siguiente sistema de ecuaciones de la siguiente forma, incluyendo que las probabilidades de los símbolos del alfabeto suman $1$.

$$
\begin{cases}
0.6p(a) + 0.2p(b) + 0.1p(c) = p(a) \\
0.3p(a) + 0.5p(b) + 0.4p(c) = p(b) \\
0.1p(a) + 0.3p(b) + 0.5p(c) = p(c) \\
p(a) + p(b) + p(c) = 1
\end{cases}
$$

Con lo que obtenemos las siguientes ecuaciones

\begin{align*}
   -0.4p(a) + 0.2p(b) + 0.1p(c) &= 0 \\
0.3p(a) - 0.5p(b) + 0.4p(c) &= 0 \\
0.1p(a) + 0.3p(b) - 0.5p(c) &= 0 \\
p(a) + p(b) + p(c) &= 1 
.\end{align*}
Resolvemos el siguiente sistema de ecuaciones, para eso multiplicamos la tercera ecuación por $4$ y la restamos por la primera 
\begin{align*}
    -0.4p(a) + 0.2p(b) + 0.1p(c)+0.4p(a)+1.2(b)-2c&=0\\
    1.4p(b)-1.9p(c)&=0
.\end{align*}
Por lo cual concluimos que $c=\dfrac{1.4b}{1.9}$.\\
Ahora tomamos la segunda ecuacion y la multiplicamos por $-4$ y la restamos con la primera ecuación
\begin{align*}
    0.3p(a)-0.5p(b)+0.4p(c)+1.6p(a)-0.8(b)-0.4p(c)&=0\\
    1.9p(a)-1.3p(b)&=0
.\end{align*}
Así, $a=\dfrac{1.3b}{1.9}$. Luego, con la restricción de que la suma de la probabilidad de los símbolos es $1$, entonces, realizando truncamiento y aproximación de la última cifra tenemos que 
\begin{align*}
    a&=0.283\\
    b&=0.412\\
    c&=0.304
.\end{align*}
    \end{sol}
    \item[B)] ¿Cuál es la probabilidad de que la fuente emita la palabra \texttt{aabc} (considerando que las letras salen de derecha a izquierda, es decir, primero \texttt{c})?
\begin{sol}
Sabemos que la palabras que se emiten en un proceso markoviano tiene probabilidad $ P(X_{n+1} = x_{n+1} \mid X_n = x_n)$. Por lo tanto, para calcular la probabilidad de aabc donde c se emite primero luego b y las 2 as, utilizamos las probabilidades de la matriz de transición y por lo tanto

\[
P(b \mid c) \cdot P(a \mid b) \cdot P(a \mid a) = (0.4) \cdot (0.2) \cdot (0.6) = 0.048
\]
\end{sol}

    \item[C)]  Dibuje el grafo de estados que modela el proceso markoviano.
    \begin{center}
         \begin{tikzpicture}[node distance = 6cm, on grid, auto]
            \node (a) [state] {$a$};
            \node (c) [state, below right of=a] {$c$};
            \node (b) [state, above right of=c] {$b$};
            
            \path[thick]
            (a) edge [loop above] node [above] {$0.6$} ()
            (a) edge [bend  left=15] node [above] {$0.3$} (b)
            (a) edge [bend right=15] node [left] {$0.1$} (c)
            (b) edge [loop above] node [above] {$0.5$} ()
            (b) edge [bend  left=15] node [below] {$0.2$} (a)
            (b) edge [bend left=15] node [right] {$0.3$} (c)
            (c) edge [loop below] node [below] {$0.5$} ()
            (c) edge [bend  right=15] node [right] {$0.1$} (a)
            (c) edge [bend left=15] node [left] {$0.9$} (b);
            \end{tikzpicture}
            \end{center} 
\end{itemize}