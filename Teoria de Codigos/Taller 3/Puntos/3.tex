%!TEX root = ../main.tex

Sea $C$ un código lineal $[n,m,d]$ sobre un cuerpo finito $GF(D).$ El código dual de $C$ es el conjunto
$$C^{\perp}=\{x\in GF(D)^n:\langle x,y\rangle=0,\forall y\in C\}.$$
Demuestre que $C^{\perp}$ es un código lineal de dimensión $n-m$ y que su matriz generadora corresponde a la matriz de verificación de $C.$
\begin{sproof}
    Primero tenemos que probar que el código dual es lineal, para esto basta probar que es un subespacio de $GF(D)^n.$\\
    Primero probemos que es cerrado bajo la suma. Sean $x_1,x_2\in C^\perp$, por definición $\langle x_1,y\rangle=0$ y $\langle x_2,y\rangle=0,$ para cualquier $y\in C$, luego como el producto interno es lineal en cada componente, tenemos que $\langle x_1+x_2,y\rangle=\langle x_1,y\rangle+\langle x_2,y\rangle=0,$ así $x_1+x_2\in C^\perp.$ Ahora falta probar que la multiplicación por escalares también es cerrada. Tomemos $\lambda\in GF(D),$ luego si tomamos $x_1$ igual que en la anterior parte, nuevamente como el producto interno es lineal tenemos que $\langle \lambda x_1,y\rangle=\lambda\langle x_1,y\rangle=0.$ Así como es cerrado para ambas propiedades hemos probado que $C^\perp$ es un subespacio y por tanto un código lineal.\\
    Ahora para encontrar la dimensión del código notemos que como $C$ es de dimensión $m$, existen vectores $g_1,\ldots,g_m\in C$, que forman una base para $C$, tales que
    $$G=\begin{bmatrix}
        g_1\\
    \vdots\\
    g_m
    \end{bmatrix}.$$
    Es la matriz generadora de $C$, recordemos que 
    $$\ker(G)=\{x\in GF(D)^n:Gx^T=0\}.$$
    La idea sera probar que $C^\perp=\ker(G)$. Sea $x\in C^\perp$,
    note que 
    $$Gx^T=\begin{bmatrix}
        g_1\\
    \vdots\\
    g_m
    \end{bmatrix}x^T=\begin{bmatrix}
        \langle x,g_1\rangle\\
        \vdots\\
        \langle x,g_m\rangle
    \end{bmatrix},$$
    Note que esto se tiene por la definición del producto de matrices, Pero como cada $g_i\in C$ y $x\in C^\perp$ tenemos que $\langle x,g_i\rangle=0$ para cada $i=1,\ldots, m.$ Así $Gx^T=0,$ luego $x\in \ker(G).$ Esto prueba $C^\perp \subseteq \ker(G).$ Para ver la otra contenecia considere $x\in ker (G),$ por definición
    $$0=Gx^T=\begin{bmatrix}
        g_1\\
    \vdots\\
    g_m
    \end{bmatrix}x^T=\begin{bmatrix}
        \langle x,g_1\rangle\\
        \vdots\\
        \langle x,g_m\rangle
    \end{bmatrix}.$$
    De esta manera $\langle x,g_i\rangle=0$ para cada $i,$ pero recordemos que los $g_i$ forman una base para $C$, luego dado $y\in C$, existen $\alpha_i\in GF(D)$ tales que $y=\sum_{i=1}^m\alpha_ig_i$, y por la linealidad del producto interno
    \begin{align*}
         \langle x,y\rangle&=\left\langle x,\sum_{i=1}^m\alpha_ig_i\right\rangle\\
         &\sum_{i=1}^m\alpha_i\langle x,g_i\rangle\\
         &=0,
     \end{align*}
     así como $y$ era arbitrario, tenemos que $\langle x,y\rangle=0$ para todo $y\in C$, luego $x\in C^\perp,$ probando así por la doble contenencia la igualdad de los conjuntos. \\
     Con estos hechos por el teorema de rango-nulidad tenemos que
     $$\dim(\ker(G))+\dim(\text{Im}(G))=n,$$
     Pero recordemos que la dimensión de la imagen de una matriz, es la dimensión del subespacio generado por sus filas, como sus filas generan $C$ y este código tiene dimensión $m$, tenemos que $\dim(\text{Im}(G))=m,$ luego como $\ker(G)=C^\perp$, si reemplazamos en la anterior expresión obtenemos
     $$\dim(C^\perp)+m=n.$$
     Así la dimensión de el código dual es $n-m.$
     Por ultimo nos falta probar que la matriz de verificacion $H$ para $C$ es la matriz generadora de $C^\perp.$ Sea $G=[\begin{array}{c|c}
         I_m& P
     \end{array}]$, donde $P$ es de tamaño $m\times (n-m)$, la matriz sistemática del código $C$, recordemos que la matriz de paridad esta dada por $H=[\begin{array}{c|c}
         -P^T& I_{n-m}
     \end{array}]$, por un hecho visto en clase sabemos que $\dim(\text{Im}(H))=n-m$, es decir que la dimensión del espacio generado por sus filas es $n-m$, si podemos probar que las filas de $H$ pertenecen a $C^\perp$, por la parte anterior, como la dimensión del espacio fila coincide con la de $C^\perp$, habremos terminado. Primero notemos que la matriz $G$ es de tamaño $m\times n$, mientras que $H$ es de tamaño $(n-m)\times n$, así que consideremos el siguiente producto matricial
     $$GH^T=[\begin{array}{c|c}
         I_m& P
     \end{array}]\left[\begin{array}{c}
         -P\\
         \hline 
         I_{n-m}
     \end{array}\right]$$
     Note que hacemos esto para que las dimensiones coincidan, el resultado serán matrices de tamaño $m\times(n-m)$, pero ademas podemos notar que la primeras $m$ columnas de $G$ son las entradas de la identidad, mientras que las primeras $m$ filas de $H^T$ son las entradas de $-P$, luego pro el producto de matrices en bloque tenemos que
     $$[\begin{array}{c|c}
         I_m& P
     \end{array}]\left[\begin{array}{c}
         -P\\
         \hline 
         I_{n-m}
     \end{array}\right]=-I_mP+PI_{n-m}=-P+P=0.$$
     Esto quiere decir que las columnas de $H^T$ son ortogonales a las filas de $G$, o de manera equivalente, las filas de $H$ son ortogonales a las de $G$. esto debido a que las entradas del producto matricial son el producto interno usual. Luego como las filas de $G$ son una base para $C$, por un argumento análogo al hecho en la prueba de $\ker(G)=C^\perp$, tenemos que las filas de $H$pertenecen a $C^\perp$, de esta manera por lo dicho al inicio de la prueba, hemos concluido que la matriz $H$ es la matriz generadora del código dual de $C.$

\end{sproof}