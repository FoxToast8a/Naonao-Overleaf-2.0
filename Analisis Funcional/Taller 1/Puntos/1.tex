%!TEX root = ../main.tex
Sea $(E, \| \cdot \|)$ un espacio vectorial normado. Defina
\[
K = \{x \in E : \|x\| = 1\}.
\]
Demuestre que $E$ es de Banach si y solamente si $K$ es completo.
\begin{sproof}
  $(\Rightarrow)$ Supongamos que $E$ es un espacio de Banach y considere $(x_n)_{n\in\mathbb{N}}$ una sucesión de Cauchy en $K\subset E$, como $E$ es completo, existe $x\in E$ tal que $x_n\to x$. Por lo que faltaría ver que $x\in K$, es decir, que $\|x\|=1.$ Por la convergencia de la sucesión, tenemos que dado $\varepsilon>0$ existe $N\in\mathbb{Z}^+$ tal que si $n\geq N$, entonces
  $$\|x_n-x\|<\varepsilon.$$
  Ahora, tenemos que cada $x_n$ es de norma $1$ ya que $x_n$ es una sucesión de Cauchy en $K,$ luego por la desigualdad triangular tenemos que
  $$\|x\|\leq \|x-x_n\|+\|x_n\|<\varepsilon+1,$$
  y además
  $$1=\|x_n\|\leq\|x_n-x\|+\|x\|<\varepsilon+\|x\|.$$
  Si juntamos las dos desigualdades, obtenemos que
  $$1-\varepsilon<\|x\|<1+\varepsilon.$$
  Así tomando $\varepsilon\to 0$ tenemos que $\|x\|=1$, mostrando así que $K$ es completo.\\

  $(\Leftarrow)$ Sea $(x_n)_{n\in\mathbb{N}}$ una sucesión de Cauchy en $E$. Por lo cual, la sucesión  $(\|x_n\|)_{n\in\mathbb{N}}$ es de Cauchy en $\mathbb{R}$ y como este espacio es completo, se sigue que $\|x_n\|\to a.$ Consideremos dos casos, si $a=0$, por la definición de convergencia, dado $\varepsilon>0$, existe $N\in\mathbb{Z^+}$ tal que si $n\geq N$ tenemos que $\|\|x_n\|-0\|<\varepsilon,$ esto es igual a $\|x_n\|<\varepsilon,$ así,  concluimos que $x_n\to 0$ por lo cual hemos acabado en este caso.\\

  Si $a\neq 0$, sin pérdida de generalidad podemos asumir que $x_n\neq 0$ para todo $n\in \mathbb{N}$, ya que en caso contrario la cantidad de ceros sería finita, lo cual no afectarían a la convergencia porque al suponer que la cantidad de ceros es infinita, al ser $x_n$ una sucesión de Cauchy eso implicaría que $x_n$ converge a $0$ ya que existiría una subsucesión convergente a $0$, y ese caso fue el anterior. Así, definimos $y_n=\dfrac{x_n}{\|x_n\|},$ luego como las sucesiones de Cauchy en $\mathbb{R}$ son acotadas, existen constantes tales que $0<M_1\leq\|x_n\|\leq M_2$ a partir de un $n\geq N$ tenemos que   
  \begin{align*}
      \|y_n-y_m\|&=\left\|\frac{x_n}{\|x_n\|}-\frac{x_m}{\|x_m\|}\right\|\\
      &=\left\|\frac{x_n\|x_m\|-x_m\|x_n\|}{\|x_n\|\|x_m\|}\right\|\\
      &\leq\frac{1}{M_1^2}\|x_n\|x_m\|-x_m\|x_n\|\|\\
      &=\frac{1}{M_1^2}\|x_n\|x_m\|-x_m\|x_m\|+x_m\|x_m\|-x_m\|x_n\|\|\\
      &=\frac{1}{M_1^2}\|(x_n-x_m)\|x_m\|+x_m(\|x_m\|-\|x_n\|)\|\\
      &\leq\frac{1}{M_1^2}(\|(x_n-x_m)\|x_m\|\|+\|x_m(\|x_m\|-\|x_n\|)\|)\\
      &\leq\frac{M_2}{M_1^2}(\|x_n-x_m\|+\|\|x_m\|-\|x_n\|\|)\\
  .\end{align*}
  Luego, como $(x_n)$ y $(\|x_n\|)$ son de Cauchy para $n$ y $m$ suficientemente grandes $\|x_n-x_m\|<\varepsilon$ y $\|\|x_m\|-\|x_n\|\|<\varepsilon.$ Así hemos concluido que $(y_n)$ es de Cauchy, pero claramente $y_n\in K$, como este es completo por hipótesis tenemos que $y_n\to y$. Podemos notar que $x_n=\|x_n\|y_n$, así tenemos que
  \begin{align*}
      \|x_n-ay\|&=\|\|x_n\|y_n-ay_n+ay_n-ay\|\\
      &=\|y_n(\|x_n\|-a)+a(y_n-y)\|\\
      &\leq\|\|x_n\|-a\|+\|a\|\|(y_n-y)\|
  .\end{align*}
  Como $\|x_n\|\to a$ y $y_n\to y$, por la desigualdad concluimos que $x_n\to ay,$ mostrando así que $(x_n)$ converge en $E$ y por tanto es Banach.










\end{sproof}