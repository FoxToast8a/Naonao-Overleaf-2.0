%!TEX root = ../main.tex

Demuestre que si $T \in \mathcal{L}(E, F)$, entonces:

\begin{enumerate}
    \item[(i)] $\|T x\|_F \leq \|T\| \, \|x\|_E$, para todo $x \in E$.
    
    \item[(ii)] $
    \|T\| = \displaystyle\sup_{\substack{x \in E \\ x \neq 0}} \frac{\|T x\|_F}{\|x\|_E}.$
    
    
    \item[(iii)] 
    $
    \|T\| = \displaystyle\sup_{\|x\|_E = 1} \|T x\|_F.
    $
    
    \item[(iv)] 
    $
    \|T\| = \inf \left\{ M > 0 : \|T x\|_F \leq M \|x\|_E, \, \forall x \in E \right\}.
    $
\end{enumerate}
\begin{proof}
\hfill
\begin{enumerate}
    \item[(i)]Sea \(\mathcal{L}(E, F)\) un espacio vectorial con la norma
\[
\|T\| = \sup_{\substack{x \in E \\ \| x\| \leq 1}} \frac{\|Tx\|_F}{\|x\|_E}.
\]
Por definición de supremo, se tiene que \(\|Tx\| \leq \|T\|\) para todo \(x \in E\) con \(\|x\| \leq 1\).

Si \(x = 0\), la desigualdad se cumple trivialmente. Tomemos ahora \(x \in E\) con \(x \neq 0\), y definamos
\[
y = \frac{x}{\|x\|}.
\]
Entonces, usando la linealidad de \(T\), se tiene:
\[
\|Ty\| = \left\|T\left(\frac{x}{\|x\|}\right)\right\| = \frac{1}{\|x\|} \|Tx\|.
\]
Por la definición del supremo, como \(\|y\| = 1\), se cumple que \(\|Ty\| \leq \|T\|\), y por lo tanto:
\[
\frac{1}{\|x\|} \|Tx\| \leq \|T\|.
\]
Multiplicando ambos lados por \(\|x\|\), obtenemos:
\[
\|Tx\| \leq \|T\|\|x\|.
\]

Así, se concluye que para todo \(x \in E\), se cumple \(\|Tx\| \leq \|T\|\|x\|\), como queríamos.

\item[(ii-iv)] Definamos:
\begin{align*}
    \alpha &= \sup_{\|x\|_E = 1} \|T x\|_F,\\
    \beta &= \sup_{\substack{x \in E \\ x \neq 0}} \frac{\|T x\|_F}{\|x\|_E},\\
    \gamma &= \inf \left\{ M > 0 : \|T x\|_F \leq M \|x\|_E, \, \forall x \in E \right\}.
\end{align*}

Veamos que \(\|Tx\| \leq \alpha\) para todo \(x \in E\) con \(\|x\| = 1\).  
Tomemos \(y \in E\) con \(y \neq 0\) tal que \(x = \frac{y}{\|y\|}\), entonces:
\begin{align*}
    \|Tx\| = \left\|T\left(\frac{y}{\|y\|}\right)\right\| = \frac{\|Ty\|}{\|y\|} \leq \alpha.
\end{align*}
Como esto vale para todo \(y \in E\) con \(y \neq 0\), se concluye que \(\beta \leq \alpha\).

Por otro lado, para todo \(x \in E\) con \(x \neq 0\), se cumple:
\[
\frac{\|Tx\|}{\|x\|} \leq \beta.
\]
Entonces, usando la linealidad de \(T\),
\[
\left\|T\left(\frac{x}{\|x\|}\right)\right\| \leq \beta.
\]
Si definimos \(y = \frac{x}{\|x\|}\), entonces \(\|y\| = 1\), y se obtiene que \(\|Ty\| \leq \beta\) para todo \(y \in E\) con \(\|y\| = 1\). Por lo tanto, \(\alpha \leq \beta\).

En consecuencia, \(\alpha = \beta\).

Ahora, si \(M > 0\) es cualquier número en el conjunto que define a \(\gamma\), entonces se cumple que \(\|Tx\| \leq M \|x\|\) para todo \(x \in E\). Esto implica que
\[
\frac{\|Tx\|}{\|x\|} \leq M,
\]
y por lo tanto, \(\beta \leq M\) para todo \(M\) en dicho conjunto. En consecuencia, \(\beta \leq \gamma\).

Por otro lado, ya sabemos que \(\frac{\|Tx\|}{\|x\|} \leq \beta\) para todo \(x \in E\), \(x \neq 0\), lo cual equivale a \(\|Tx\| \leq \beta \|x\|\). Es decir, \(\beta\) también cumple la propiedad que del conjunto que define \(\gamma\), así que \(\gamma \leq \beta\). Concluimos entonces que:
\[
\alpha = \beta = \gamma.
\]

Finalmente, notemos que \(\|T\| \geq \alpha\), ya que:
\[
\{x \in E : \|x\| = 1\} \subseteq \{x \in E : \|x\| \leq 1\}.
\]
Por otro lado, si \(M\) pertenece al conjunto que define a \(\gamma\), entonces para todo \(x \in E\) con \(\|x\| \leq 1\), se tiene \(\|Tx\| \leq M\), y como \(M\) es una cota superior de \(\|Tx\|\) sobre la bola unitaria, se concluye que \(\|T\| \leq M\). Por ser esto válido para todo \(M\) del conjunto que define a \(\gamma\), se tiene que \(\|T\| \leq \gamma\).

Por lo tanto,
\[
\|T\| = \alpha = \beta = \gamma.
\]


\end{enumerate}
    
    
\end{proof}
