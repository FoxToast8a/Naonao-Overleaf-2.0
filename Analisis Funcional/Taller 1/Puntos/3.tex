%!TEX root = ../main.tex
Demuestre que si $T \in \mathcal{L}(E, F)$, entonces:

\begin{enumerate}
    \item[(i)] $\|T x\|_F \leq \|T\| \, \|x\|_E$, para todo $x \in E$.
    
    \item[(ii)] $
    \|T\| = \displaystyle\sup_{\substack{x \in E \\ x \neq 0}} \frac{\|T x\|_F}{\|x\|_E}.$
    
    
    \item[(iii)] 
    $
    \|T\| = \displaystyle\sup_{\substack{x\in\|x\|_E = 1} \|T x\|_F.
    $
    
    \item[(iv)] 
    $
    \|T\| = \inf \left\{ M > 0 : \|T x\|_F \leq M \|x\|_E, \, \forall x \in E \right\}.
    $
\end{enumerate}
\begin{proof}
\hfill
\begin{enumerate}
<<<<<< HEAD
    %\iem[(i)]
=======
    % \item[(i)]
>>>>>>> fc6fabea4da37f17c4f0bcc8f9225a0725c0cbb9
    %\textcolor{blue}{prueba más sencilla} Por hipótesis tenemos que $T\in \mathcal{L}(E,F)$, entonces como $T$ es una aplicación lineal continua en particular, $\|Tx\|_F\leq \|T\| \|x\|_E$ para todo $x\in E$.



 \item[(i)]Sea \(\mathcal{L}(E, F)\) un espacio vectorial con la norma
\[
\|T\| = \sup_{\substack{x \in E \\ \| x\| \leq 1}}\|Tx\|_F.
\]
Por definición de supremo, se tiene que \(\|Tx\| \leq \|T\|\) para todo \(x \in E\) con \(\|x\| \leq 1\), si \(x = 0\), la desigualdad se cumple trivialmente. 

Ahora, tomemos \(x \in E\) con \(x \neq 0\) y $\|x\|>1$, definamos
\[
y = \frac{x}{\|x\|},
\]
 usando la linealidad de \(T\), se tiene que
\[
\|Ty\| = \left\|T\left(\frac{x}{\|x\|}\right)\right\| = \frac{1}{\|x\|} \|Tx\|.
\]
Por la definición del supremo, como \(\|y\| = 1\), se cumple que \(\|Ty\| \leq \|T\|\), y por lo tanto,
\[
\frac{1}{\|x\|} \|Tx\| \leq \|T\|,
\]
multiplicando ambos lados por \(\|x\|\),
\[
\|Tx\| \leq \|T\|\|x\|.
\]

Así, se concluye que para todo \(x \in E\), se cumple \(\|Tx\| \leq \|T\|\|x\|\), como queríamos.

\item[(ii-iv)] Definamos:
\begin{align*}
    \alpha &= \sup_{\substack{x\in E\\ \|x\|_E = 1}} \|T x\|_F,\\
    \beta &= \sup_{\substack{x \in E \\ x \neq 0}} \frac{\|T x\|_F}{\|x\|_E},\\
    \gamma &= \inf \left\{ M > 0 : \|T x\|_F \leq M \|x\|_E, \, \forall x \in E \right\}.
\end{align*}

Veamos que \(\|Tx\| \leq \alpha\) para todo \(x \in E\) con \(\|x\| = 1\).  
Tomemos \(y \in E\) con \(y \neq 0\) tal que \(x = \frac{y}{\|y\|}\), por lo que
\begin{align*}
    \|Tx\| = \left\|T\left(\frac{y}{\|y\|}\right)\right\| = \frac{\|Ty\|}{\|y\|} \leq \alpha,
\end{align*}
como esto es válido para todo \(y \in E\) con \(y \neq 0\), se concluye que \(\beta \leq \alpha\).

Por otro lado, para todo \(x \in E\) con \(x \neq 0\), se cumple que,
\[
\frac{\|Tx\|}{\|x\|} \leq \beta,
\]
entonces, usando la linealidad de \(T\),
\[
\left\|T\left(\frac{x}{\|x\|}\right)\right\| \leq \beta,
\]
si definimos \(y = \frac{x}{\|x\|}\), entonces \(\|y\| = 1\), y se obtiene que \(\|Ty\| \leq \beta\) para todo \(y \in E\) con \(\|y\| = 1\). Por lo tanto, \(\alpha \leq \beta\).

En consecuencia, \(\alpha = \beta\).

Ahora, si \(M > 0\) es un número en el conjunto que define a \(\gamma\), entonces se cumple que \(\|Tx\| \leq M \|x\|\) para todo \(x \in E\). Esto implica que
\[
\frac{\|Tx\|}{\|x\|} \leq M, \text{ donde $\|x\| \neq 0$}
\]
y por lo tanto, \(\beta \leq M\) para todo \(M\) en dicho conjunto. En consecuencia, \(\beta \leq \gamma\).

Por otro lado, ya sabemos que \(\dfrac{\|Tx\|}{\|x\|} \leq \beta\) para todo \(x \in E\), \(x \neq 0\), lo cual equivale a \(\|Tx\| \leq \beta \|x\|\). Es decir, \(\beta\) también cumple la propiedad del conjunto que define \(\gamma\), así que \(\gamma \leq \beta\). Entonces, podemos concluir que,
\[
\alpha = \beta = \gamma.
\]

Finalmente, notemos que \(\|T\| \geq \alpha\), ya que,
\[
\{x \in E : \|x\| = 1\} \subseteq \{x \in E : \|x\| \leq 1\}.
\]
Si \(M\) pertenece al conjunto que define a \(\gamma\), entonces para todo \(x \in E\) con \(\|x\| \leq 1\), se tiene \(\|Tx\| \leq M\), y como \(M\) es una cota superior de \(\|Tx\|\) sobre la bola unitaria, se concluye que \(\|T\| \leq M\). Esto válido para todo \(M\) del conjunto que define a \(\gamma\), por lo que se tiene que \(\|T\| \leq \gamma\).

Por lo tanto,
\[
\|T\| = \alpha = \beta = \gamma.
\]
Con lo cual, concluimos la demostración.


\end{enumerate}
    
    
\end{proof}
