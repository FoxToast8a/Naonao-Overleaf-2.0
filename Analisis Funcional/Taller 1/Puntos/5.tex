%!TEX root = ../main.tex
Sean $E$ y $F$ espacios vectoriales normados. Suponga que $E$ es de dimensión finita (no se asume que $F$ sea de dimensión finita).

\begin{enumerate}
    \item[(i)] Muestre que todas las normas asignadas a $E$ son equivalentes.
    \begin{proof}
    Sea $E$ un espacio vectorial normado de dimensión finita $N$. Para esta demostación basta probar que $E$ es isomorfo al espacio $\ell_1^N$. Por consiguiente, dadas dos normas en $E$, estas serán isomorfas a $\ell_1^N$ para cada una de estas normas, y de esto deduciremos la equivalencia de las normas.\\

Sea $\mathcal{B} = \{e_1, e_2, \ldots, e_n\}$ la base canónica de $\ell_1^N$. Tomemos una base $\mathcal{B}' = \{v_1, v_2, \ldots, v_n\}$ para $E$. Definamos una aplicación  $T: \ell_1^N \to E$ tal que $T(e_i) = v_i$ para todo $1 \leq i \leq N$. Por su definición, tenemos que $T$ es transformación lineal.\\

Veamos ahora que $T$ es continua, para ello veremos que es acotada.

Sea $x = (x_1, x_2, \ldots, x_n) \in \mathbb{R}^n$, luego $x = \displaystyle\sum_{i=1}^N x_i e_i$. Por lo cual,
\begin{align*}
T(x) &= \sum_{i=1}^N x_i T(e_i) = \sum_{i=1}^N x_i v_i.
\end{align*}

Así,
\begin{align*}
\|T(x)\|_E 
&= \left\| \sum_{i=1}^N x_i v_i \right\|_E \\
&\leq \sum_{i=1}^N \|x_i v_i\|_E \\
&= \sum_{i=1}^N |x_i| \cdot \|v_i\|_E \\
&\leq \sum_{i=1}^N |x_i| \cdot \max_{1 \leq i \leq N} \|v_i\|_E
\end{align*}

si $K = \max_{1 \leq i \leq N} \|v_i\|_E$, entonces $T$ está acotada y por ende es continua. Adicionalmente, $T$ es biyectiva por su definición.

Ahora, razonando por reducción al absurdo suponga que $T^{-1}$ no es una transformación lineal continua, lo cual por las equivalencias mostradas en el ejercicio 2 la no continuidad debe fallar en 0. Esto nos indica que podemos encontrar una sucesión $\{y_n\}$ en $V$ y un número real $\varepsilon > 0$ tal que $\|T^{-1}(y_n)\|_1 > \varepsilon$ mientras que $y_n \to 0$. Definamos 
$$
z_n = \frac{y_n}{\|T^{-1}(y_n)\|_1},
$$
entonces cuando $z_n \to 0$  
\begin{align*}
\|T^{-1}(z_n)\|_1 
&= \left\|T^{-1}\left(\dfrac{y_n}{\|T^{-1}(y_n)\|_1}\right)\right\|_1 \\
&= \dfrac{1}{\|T^{-1}(y_n)\|_1} \cdot \|T^{-1}(y_n)\|_1 \\
&= 1.
\end{align*}

Ahora bien, mostremos que el siguiente conjunto es compacto, sea 
$$
B = \left\{ x \in \ell_1^N : \|x\|_1 \leq 1 \right\},
$$

note que $B$ es cerrado y acotado, y al ser la topología en $\ell_1^N$ es la misma que con la topología usual en $\mathbb{R}^N$. tenemos que  $B$ es también secuencialmente compacto, y entonces, por definición, existe una subsucesión $\{z_{n_k}\}$ tal que $\{T^{-1}(z_{n_k})\}$ es convergente.

Sea $T^{-1}(z_{n_k}) \to x$, teniendo en cuenta que $\|T^{-1}(z_{n_k})\|_1 = 1$ y $T^{-1}(z_{n_k}) \to x$, entonces $\|x\|_1 = 1$.

Dado que $T$ es continua,
\begin{align*}
\lim T(T^{-1}(z_{n_k})) &= T(x),
\end{align*}
es decir,
\begin{align*}
z_{n_k} \to T(x),
\end{align*}
lo cual implica que $T(x) = 0$. Pero $T$ es una aplicación 1-1 y $\|x\|_1 = 1$, por consiguiente $x \neq 0$, lo cual prueba que $T(x) \neq 0$.Esto nos lleva a una contradicción. Por ende, $T^{-1}$ debe ser continua.

Por lo cual, para cualquier norma en $E$, la aplicación $T$ es siempre un isomorfismo entre $\ell_1^N$ y $E$, lo cual implica que la aplicación de $(E, \|\cdot\|_{E_1})$ y $(E, \|\cdot\|_{E_2})$ también lo será, concluyendo así que las dos normas son equivalentes.
        
    \end{proof}
    
    \item[(ii)] Muestre que toda transformación lineal $T : E \to F$ es continua.
    \begin{proof}
    Sea $T: E\rightarrow F$ una transformación lineal, veamos que $T$ es continua, para esto de acuerdo con el ejercicio $2$ de este taller, sabemos que basta con mostrar que es acotada.\\

    Como $E$ es un espacio vectorial de dimensión finita, tomemos la base de $E$ como $\mathcal{B}=\{v_1,v_2,\dots, v_n\}$, luego si $x\in E$ tenemos que $x=\displaystyle\sum_{i=1}^{n}x_iv_i$, entonces la transformación lineal T es de la forma,
      \begin{align*}
           Tx&=T\left( \displaystyle\sum_{i=1}^{n} x_i v_i\right) \\
           &=\displaystyle\sum_{i=1}^{n} x_i T(v_i),\\
           &=\displaystyle\sum_{i=1}^{n} x_i v_i \max_{\substack{1\leq i\leq n}} T(v_i),
        \end{align*} 
    luego, tenemos que
    \begin{align*}
\|T\|_E 
&= \left\|\sum_{i=1}^{n} x_i T(v_i)\right\|_E\\
&\leq \sum_{i=1}^{n} \left\|x_i T(v_i)\right\|_E\\
&= \sum_{i=1}^{n} |x_i| \cdot \|T(v_i)\|_E,
\end{align*}
 por el punto anterior, como $E$ es un espacio de dimensión finita existen constantes positivas $C_1$ y $C_2$, tal que $C_1\|x\|_E\leq\|x\|_1\leq C_2\|x\|_E$ y si tomamos a $M_1=\max_{\substack{1\leq i\leq n}} \|T(v_i)\|_E$, entonces,
 \begin{align*}
     \sum_{i=1}^{n} |x_i| \cdot \|T(v_i)\|_E&= \|x\|_1 \max_{\substack{1\leq i \leq n}} \|T(v_i)\|_E,\\
     &\leq M_1 C_2\|x\|_E\\
     &=M \|x\|_E,
 \end{align*}
 por lo cual, T es acotado y continuo.
    \end{proof}
    \item[(iii)] Dé un ejemplo donde se verifique que (ii) puede ser falsa si $E$ es de dimensión infinita.
    \begin{sol}
    Sea $
T: (C^1([0,1]), \|\cdot\|_{\infty}) \longrightarrow (\mathbb{R}, |\cdot|)
$
donde por $f \mapsto f'(0)$. Note que $T$ es una tranformación lineal.

Como queremos mostrar que la tranformación no es continua, demostremos que $T$ no es acotada.
\begin{proof}

Supongamos que $T \in \mathcal{L}(E, \mathbb{R})$ con $\|\cdot\|_{\mathcal{L}^\infty}$, por lo cual,  existe $M > 0$ tal que 
$$
|T(x)| = |f'(0)| \leq M \|f\|_{\mathcal{L}^\infty }\quad \text{para todo } f \in C^1([0,1]).
$$

Sea $n \geq 2$ y $f(x) = (1 - x)^n$, entonces,
$$
\|f\|_{\mathcal{L}^\infty} = 1, \quad f'(x) = -n(1 - x)^{n - 1}, \quad f'(0) = -n.
$$

Reemplazando
\begin{align*}
n &= |f'(0)| \leq M \|f\|_\infty = M.
\end{align*}

Por tanto, cuando $n \to \infty$, se tendría $n \leq M$ para todo $n \geq 2$, lo cual es una contradicción. Entonces $T$ no es acotada.

\end{proof}
        
    \end{sol}
\end{enumerate}


