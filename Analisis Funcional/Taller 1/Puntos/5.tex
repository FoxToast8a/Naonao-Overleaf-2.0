%!TEX root = ../main.tex
Sean $E$ y $F$ espacios vectoriales normados. Suponga que $E$ es de dimensión finita (no se asume que $F$ sea de dimensión finita).

\begin{enumerate}
    \item[(i)] Muestre que todas las normas asignadas a $E$ son equivalentes.
    \begin{proof}
  Sea $E$ un espacio vectorial normado de dimensión finita $N$. Para esta demostación basta probar que $E$ es isomorfo al espacio $\ell_1^N$. Por consiguiente, dadas dos normas en $E$, estas serán isomorfas a $\ell_1^N$ para cada una de estas normas, y de esto deduciremos la equivalencia de las normas.\\

Sea $\mathcal{B} = \{e_1, e_2, \ldots, e_n\}$ la base canónica de $\ell_1^N$. Tomemos una base $\mathcal{B}' = \{v_1, v_2, \ldots, v_n\}$ para $E$. Definamos una aplicación  $T: \ell_1^N \to E$ tal que $T(e_i) = v_i$ para todo $1 \leq i \leq N$. Por su definición, tenemos que $T$ es transformación lineal.\\

Veamos ahora que $T$ es continua, para ello veremos que es acotada.

Sea $x = (x_1, x_2, \ldots, x_N) \in \mathbb{R}^N$, luego $x = \displaystyle\sum_{i=1}^N x_i e_i$. Por lo cual,
\begin{align*}
T(x) &= \sum_{i=1}^N x_i T(e_i) = \sum_{i=1}^N x_i v_i.
\end{align*}

Así,
\begin{align*}
\|T(x)\|_E 
&= \left\| \sum_{i=1}^N x_i v_i \right\|_E \\
&\leq \sum_{i=1}^N \|x_i v_i\|_E \\
&= \sum_{i=1}^N |x_i| \cdot \|v_i\|_E \\
&\leq \sum_{i=1}^N |x_i| \cdot \max_{1 \leq i \leq N} \|v_i\|_E
\end{align*}

si $K =$$ \max_{\substack{1 \leq i \leq N}} \|v_i\|_E$, entonces $T$ está acotada y por lo tanto es continua. Adicionalmente, $T$ es biyectiva por su definición.

Ahora, razonando por reducción al absurdo suponga que $T^{-1}$ no es una transformación lineal continua, lo cual por las equivalencias mostradas en el ejercicio 2 la no continuidad debe fallar en 0. Esto nos indica que podemos encontrar una sucesión $\{y_n\}$ en $V$ y un número real $\varepsilon > 0$ tal que $\|T^{-1}(y_n)\|_1 > \varepsilon$ mientras que $y_n \to 0$. Definamos 
$$
z_n = \frac{y_n}{\|T^{-1}(y_n)\|_1},
$$
entonces cuando $z_n \to 0$  
\begin{align*}
\|T^{-1}(z_n)\|_1 
&= \left\|T^{-1}\left(\dfrac{y_n}{\|T^{-1}(y_n)\|_1}\right)\right\|_1 \\
&= \dfrac{1}{\|T^{-1}(y_n)\|_1} \cdot \|T^{-1}(y_n)\|_1 \\
&= 1.
\end{align*}

Ahora bien, mostremos que el siguiente conjunto es compacto, sea 
$$
B = \left\{ x \in \ell_1^N : \|x\|_1 \leq 1 \right\},
$$

note que $B$ es cerrado y acotado, y al ser la topología en $\ell_1^N$ es la misma que con la topología usual en $\mathbb{R}^N$. tenemos que  $B$ es también secuencialmente compacto, y entonces, por definición, existe una subsucesión $\{z_{n_k}\}$ tal que $\{T^{-1}(z_{n_k})\}$ es convergente.

Sea $T^{-1}(z_{n_k}) \to x$, teniendo en cuenta que $\|T^{-1}(z_{n_k})\|_1 = 1$ y $T^{-1}(z_{n_k}) \to x$, entonces $\|x\|_1 = 1$.

Dado que $T$ es continua,
\begin{align*}
\lim T(T^{-1}(z_{n_k})) &= T(x),
\end{align*}
es decir,
\begin{align*}
z_{n_k} \to T(x),
\end{align*}
lo cual implica que $T(x) = 0$. Pero $T$ es una aplicación 1-1 y $\|x\|_1 = 1$, por consiguiente $x \neq 0$, lo cual prueba que $T(x) \neq 0$.Esto nos lleva a una contradicción. Por ende, $T^{-1}$ debe ser continua.

Por lo cual, para cualquier norma en $E$, la aplicación $T$ es siempre un isomorfismo entre $\ell_1^N$ y $E$, lo cual implica que la aplicación de $(E, \|\cdot\|_{E_1})$ y $(E, \|\cdot\|_{E_2})$ también lo será, concluyendo así que las dos normas son equivalentes.


Ahora veamos cuáles son las constantes positivas con las que podemos establecer que las normas son equivalentes, de acuerdo con la definición dada en el taller.

Puesto que $(E,\| \cdot \|_1)$ es isomorfo a $(E, \|\cdot\|_2)$ existe una aplicación lineal, la cual tomaremos de la siguiente manera

\[
I : (E, \| \cdot \|_1) \longrightarrow (E, \| \cdot \|_2), \quad x \mapsto I(x) = x.
\]

luego, tenemos que $I$ es acotada, es decir, existe $C_2>0$ tal que 

\begin{align*}
    \|x\|_2 \leq C_2 \|x\|_1
.\end{align*}

Ahora consideremos la aplicación lineal inversa a $I$
\[
I^{-1} : (E, \| \cdot \|_1) \longrightarrow (E, \| \cdot \|_2),
\]
de igual forma, tenemos que la aplicación inversa es acotada, por lo cual, existe $C_1^{-1}>0$, tal que 
\begin{align*}
    \|x\|_1 \leq C_1^{-1} \|x\|_2
,\end{align*}
así, tenemos que 
\[
C_1 \|x\|_1 \leq \|x\|_2 \leq C_2 \|x\|_1 \quad \text{para todo } x \in E,
\]
lo cual demuestra que las normas \( \| \cdot \|_1 \) y \( \| \cdot \|_2 \) son equivalentes.
%Para mostrar que todas las normas definidas en $E$ son equivalentes, primero vamos a mostrar que se tiene la transitividad entre normas equivalentes, además es importante resaltar que en este caso estamos tomando como norma base, la norma $1$ definida como,
%\begin{align*}
 %   \|x\|_1 = \sum_{i=0}^{n} |\alpha_i|,
%\end{align*}
%donde $x = \sum_{i=0}^{n} \alpha_i e_i \in E$.

%Ahora, supongamos que $\|\cdot\|_a$ y $\|\cdot\|_b$ son dos normas definidas sobre $E$ que son equivalentes a la norma $\|\cdot\|_1$. Esto quiere decir que existen constantes positivas $C_1, C_2, C_1', C_2'$ tales que, para todo $x \in E$, se cumple
%\begin{align*}
 %   C_1 \|x\|_1 \leq \|x\|_a \leq C_2 \|x\|_1, \\
  %  C_1' \|x\|_1 \leq \|x\|_b \leq C_2' \|x\|_1.
%\end{align*}

%A partir de estas desigualdades, queremos establecer la equivalencia directa entre $\|\cdot\|_a$ y $\|\cdot\|_b$. Combinando las desigualdades anteriores podemos decir que, 
%\[
%\left\{
%\begin{aligned}
%& C_1 \|x\|_1 \leq \|x\|_a \leq C_2 \|x\|_1 && \Rightarrow && \frac{C_1}{C_2} \|x\|_1 \leq \frac{1}{C_2} \|x\|_a \leq \|x\|_1 && \Rightarrow && \frac{1}{C_2} \|x\|_a \leq \frac{1}{C_1'} \|x\|_b \\
%& C_1' \|x\|_1 \leq \|x\|_b \leq C_2' \|x\|_1 && \Rightarrow && \|x\|_1 \leq \frac{1}{C_1'} \|x\|_b \leq \frac{C_2'}{C_1'} \|x\|_1 && \Rightarrow && \|x\|_a \leq \frac{C_2}{C_1'} \|x\|_b \\
%& \|x\|_1 \leq \frac{1}{C_1} \|x\|_a \leq \frac{C_2}{C_1} \|x\|_1 && \Rightarrow && \|x\|_1 \leq \frac{1}{C_1} \|x\|_a && \Rightarrow && \frac{1}{C_2'} \|x\|_b \leq \frac{1}{C_1} \|x\|_a \\
%& \frac{C_1'}{C_2'} \leq \frac{1}{C_2'} \|x\|_b \leq \|x\|_1 && \Rightarrow && \frac{1}{C_2'} \|x\|_b \leq \|x\|_1 && \Rightarrow && \frac{C_1'}{C_2'} \|x\|_b \leq \|x\|_a
%\end{aligned}
%\right.
%\]

%Luego,
%\[
%\frac{C_1}{C_2'} \|x\|_b \leq \|x\|_a \leq \frac{C_2}{C_1'} \|x\|_b
%\]

%como \( C_1, C_2, C_1', C_2' \) son constantes positivas, \( \dfrac{C_1'}{C_2'} \) y \( \dfrac{C_2}{C_1'} \) también lo son. Lo que nos indica que \( \|\cdot\|_a \) y \( \|\cdot\|_b \) son equivalentes.\\

%Ahora, queremos probar que 
%\[
%C_1 \|x\|_1 \leq \|x\|_a \leq C_2 \|x\|_1
%\]
%se cumple para todo $x \in E$, donde $C_1, C_2$ son constantes positivas. 

%Para el caso $x = 0$, la proposición se cumple trivialmente. Entonces, tomemos $x \neq 0$ y veamos que, dividiendo entre $\|x\|_1$
%\[
%C_1 \leq \frac{\|x\|_a}{\|x\|_1} \leq C_2 
%\text{ por lo que, }
%C_1 \leq \left\|\frac{x}{\|x\|_1}\right\|_a \leq C_2 
%\text{ , así  } 
%C_1 \leq \|u\|_a \leq C_2 \quad \text{(1)}
%\]
%si tomamos $u = \dfrac{x}{\|x\|_1}$, donde $\|u\|_1 = 1$. Esto nos indica que basta con probar $(1)$ para $u \in E$, con $\|u\|_1 = 1$.\\

%Ahora, veamos que cualquier norma $\|\cdot\|_a$ en el espacio $E$, es continua en $\|\cdot\|_1$

%Sea $\|\cdot\|_a : (E, \|\cdot\|_1) \to (\mathbb{R}, |\cdot|)$ definida por $x \mapsto \|x\|_a$.

%Queremos mostrar que esta función es continua, es decir, que para todo $\varepsilon > 0$, existe $\delta > 0$ tal que si $\|x - x'\|_1 < \delta$, entonces 
%\[
%|\|x\|_a - \|x'\|_a| < \varepsilon.
%\]

%Por la desigualdad triangular para la norma $\|\cdot\|_a$ se tiene

%\begin{align*}
%\left| \|x\|_a - \|x'\|_a \right| 
%&\leq \|x - x'\|_a,
%\end{align*}

%ahora, escribamos $x = \sum_{i=1}^n \alpha_i e_i$ y $x' = \sum_{i=1}^n \alpha_i' e_i$, con $\{e_i\}_{i=1}^n$ una base finita de $E$, luego,
%\begin{align*}
%\|x - x'\|_a &= \left\| \sum_{i=1}^n (\alpha_i - \alpha_i') e_i \right\|_a \\
%&\leq \sum_{i=1}^n |\alpha_i - \alpha_i'| \cdot \|e_i\|_a \\
%&\leq \left( \max_{1 \leq i \leq n} \|e_i\|_a \right) \sum_{i=1}^n |\alpha_i - \alpha_i'| \\
%&= \left( \max_{1 \leq i \leq n} \|e_i\|_a \right) \|x - x'\|_1.
%\end{align*}

%Por lo tanto, si tomamos 
%\[
%\delta = \frac{\varepsilon}{\max_{1 \leq i \leq n} \|e_i\|_a},
%\]
%entonces, si $\|x - x'\|_1 < \delta$, se sigue que,

%\[
%|\|x\|_a - \|x'\|_a| \leq \|x - x'\|_a < \varepsilon.
%\]

%Esto prueba que $\|\cdot\|_a$ es continua en $(E, \|\cdot\|_1)$.


%Ahora veamos cuales serían las constantes para decir que las normas son equivalentes para eso vamos a probar que si $B(0,1)$ es compacto y acotado entonces en ese conjunto se alcanza máximo y mínimo.

%Por lo demostrado anteriormente sabemos que $\|\cdot\|_a$ es una función continua y como $E$ es un espacio vectorial de dimensión finita, luego, el conjunto
%\[
%B = \left\{ x \in E : \|x\|_1 = 1 \right\}
%\]
%es compacto por ser cerrado y acotado, por lo que existe un isomorfismo lineal $T : \mathbb{R}^n \to E$, por el teorema del valor extremo, la función continua definida sobre $B$ alcanza su máximo y su mínimo. Definamos
%\[
%C_1 = \min_{\|x\|_1 = 1} \|x\|_a, \quad C_2 = \max_{\|x\|_1 = 1} \|x\|_a.
%\]

%Como $x \neq 0$ y $\|x\|_1 = 1$, entonces $C_1$ y $C_2$ son constantes positivas, con $C_2 \geq C_1$, y se cumple que
%\[
%C_1 \leq \|x\|_a \leq C_2.
%\]

%Luego, por la definición de equivalencia de normas, $\|\cdot\|_a$ y $\|\cdot\|_1$ son equivalentes.


        
   \end{proof}
  
   \item[(ii)] Muestre que toda transformación lineal $T : E \to F$ es continua.
    \begin{proof}
    Sea $T: E\rightarrow F$ una transformación lineal, veamos que $T$ es continua, para esto de acuerdo con el ejercicio $2$ de este taller, sabemos que basta con mostrar que es acotada.\\

    Como por hipótesis $E$ es un espacio vectorial de dimensión finita, tomemos la base de $E$ como $\mathcal{B}=\{v_1,v_2,\dots, v_n\}$, luego si $x\in E$ tenemos que $x=\displaystyle\sum_{i=1}^{n}x_iv_i$, entonces la transformación lineal T es de la forma,
      \begin{align*}
           Tx&=T\left( \displaystyle\sum_{i=1}^{n} x_i v_i\right) \\
           &=\displaystyle\sum_{i=1}^{n} x_i T(v_i),
        \end{align*} 
    así, tenemos que
    \begin{align*}
\|T\|_E 
&= \left\|\sum_{i=1}^{n} x_i T(v_i)\right\|_E\\
&\leq \sum_{i=1}^{n} \left\|x_i T(v_i)\right\|_E\\
    & =\sum_{i=1}^{n} |x_i| \cdot \|T(v_i)\|_E,
\end{align*}
 por el punto anterior, como $E$ es un espacio de dimensión finita existen constantes positivas $C_1$ y $C_2$, tal que $C_1\|x\|_E\leq\|x\|_1\leq C_2\|x\|_E$ y si tomamos a $M_1=\max_{\substack{1\leq i\leq n}} \|T(v_i)\|_E$, entonces,
 \begin{align*}
     \sum_{i=1}^{n} |x_i| \cdot \|T(v_i)\|_E &\leq\|x\|_1\max_{\substack{1\leq i \leq n}} \|T(v_i)\|_E,\\
     &\leq M_1 C_2\|x\|_E\\
     &=M \|x\|_E,
 \end{align*}
 por lo cual, T es acotado y continuo.
    \end{proof}
    \item[(iii)] Dé un ejemplo donde se verifique que (ii) puede ser falsa si $E$ es de dimensión infinita.
\begin{sol}
    Sea $
T: (C^1([0,1]), \|\cdot\|_{\infty}) \longrightarrow (\mathbb{R}, |\cdot|)
$
donde por $f \mapsto f'(0)$. Note que $T$ es una transformación lineal.

Queremos mostrar por la linealidad de la derivada que la transformación no es continua,  por lo tanto, demostremos que $T$ no es acotada.
\begin{proof}

Supongamos que $T \in \mathcal{L}(E, \mathbb{R})$ con $\|\cdot\|_{\mathcal{L}^\infty}$, por lo cual,  existe $M > 0$ tal que 
$$
|T(x)| = |f'(0)| \leq M \|f\|_{\mathcal{L}^\infty }\quad \text{para todo } f \in C^1([0,1]).
$$

Sea $n \geq 2$ y $f(x) = (1 - x)^n$,  entonces, por la forma en la que está definida $f$ tenemos que su máximo es $1$ y se alcanza cuando $f$ se evalúa en $0$, luego $
\|f\|_{\mathcal{L}^\infty} = 1,$ tomando la derivada de $f$ tenemos que

$$
f'(x) = -n(1 - x)^{n - 1}, \quad f'(0) = -n.
$$

Reemplazando
\begin{align*}
n &= |f'(0)| \leq M \|f\|_\infty = M.
\end{align*}

Por tanto, cuando $n \to \infty$, se tendría $n \leq M$ para todo $n \geq 2$, lo cual es una contradicción. Entonces $T$ no es acotada.

\end{proof}
        
\end{sol}
\end{enumerate}


