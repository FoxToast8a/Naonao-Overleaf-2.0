%!TEX root = ../main.tex

Considere $E = c_0$, donde
\[
c_0 = \left\{ u = \{u_n\}_{n \geq 1} :\text{tales que } u_n \in \mathbb{R}, \ \lim_{n \to \infty} u_n = 0 \right\}.
\]
Es decir, $c_0$ es el conjunto de las secuencias reales que tienden a cero. Dotamos a este espacio con la norma
$
\|u\|_{\ell^\infty} = \sup_{n \in \mathbb{Z}^+} |u_n|.
$
Considere el funcional $f : E \to \mathbb{R}$ dado por
\[
f(u) = \sum_{n=1}^{\infty} \frac{1}{2^n} u_n.
\]

\begin{enumerate}
    \item[(i)] Muestre que $f \in E^*$ y calcule $\|f\|_{E^*}$.
    \begin{sols}
    Primero, observemos que el funcional está bien definido, esto ya que  las sucesiones convergentes son acotadas, el supremo existe y como
    $$\left|\frac{1}{2^n}u_n\right|\leq\|u\|_{\ell^\infty}\frac{1}{2^n},$$ 
    por el criterio de comparación converge absolutamente, ya que el lado derecho de la desigualdad es una serie geométrica. Luego, dadas $u,v\in E$ y $\lambda\in\mathbb{R},$ tenemos que $u+\lambda v\in E$, donde $u+\lambda v=\{u_n+\lambda v_n\}_{n\geq 1}.$ Así, por la convergencia absoluta tenemos que $f$ es lineal, ya que
    \begin{align*}
       f(u+\lambda v)&=\sum_{n=1}^\infty\frac{1}{2^n}(u_n+\lambda v_n)\\
       &=\sum_{n=1}^\infty\left(\frac{1}{2^n}u_n+\frac{1}{2^n}\lambda v_n\right),\\
       &=\sum_{n=1}^\infty\frac{1}{2^n}u_n+\lambda\sum_{n=1}^\infty\frac{1}{2^n} v_n\\
       &=f(u)+\lambda f(v)
    .\end{align*}
        Ahora, mostremos que $f$ es acotado. Observe que para una suma parcial se tiene que
        \begin{align*}
            \left|\sum_{n=1}^m\frac{1}{2^n}u_n\right|&\leq\sum_{n=1}^m\frac{1}{2^n}|u_n|\\
            &\leq \sup_{n\in\mathbb{Z}^+}|u_n|\sum_{n=1}^m\frac{1}{2^n}\\
            &=\|u\|_{\ell^\infty}\sum_{n=1}^m\frac{1}{2^n}
        .\end{align*}
    Note que, si hacemos $m\to\infty$ al lado derecho tenemos una serie geométrica que converge a $1$, por lo cual, tenemos que
    $$|f(u)|\leq \|u\|_{\ell^\infty}.$$
    Mostrando así que $f$ es acotada.\\
    Faltaría simplemente calcular $\|f\|_{E^*}.$ Por la cota hallada previamente si tomamos el supremo a ambos lados tenemos que
    $$\|f(u)\|_{E^*}\sup_{\|u\|_{\ell^\infty}\leq 1}|f(u)|\leq\sup_{\|u\|_{\ell^\infty}\leq 1}\|u\|_{\ell^\infty}\leq 1.$$
    Ahora considere la sucesión $u^N$, donde $N\in \mathbb{Z}^+$ y está definida de la siguiente manera
    $$u_n=\begin{cases}
        1 & \text{Si }n\leq N,\\
        0 & \text{Si }n>N.
    \end{cases}$$
    Claramente $\|u^N\|_{\ell^\infty}=1$, luego por la desigualdad mostrada en el ejercicio $3$ numeral $(i)$ tenemos
    \begin{align*}
        \sum_{i=1}^N\frac{1}{2^n}&=|f(u^N)|\\
        &\leq\|f\|_{E^*}\|u^N\|_{\ell^\infty}\\
        &=\leq\|f\|_{E^*}
    .\end{align*}
    Así como el lado derecho de la desigualdad no depende de $N$, si tomamos $N\to \infty$ tenemos que 
    $$1\leq\|f\|_{E^*}.$$
    Por lo que concluimos que $\|f\|_{E^*}=1.$
    
    \end{sols}
    \item[(ii)] ¿Es posible encontrar $u \in E$ tal que $\|u\| = 1$ y $f(u) = \|f\|_{E^*}$?
    \begin{sols}
        En el numeral anterior vimos que $\|f\|_{E^*}=1$, ahora queremos ver si existe una sucesión $u\in E$ de norma $1$ tal que $f(u)=1.$ Supongamos que existe tal sucesión y veamos como esto nos lleva a una contradicción. Por hipótesis
        $$u_n\leq |u_n|\leq \|u\|_{\ell^\infty}=1,$$
        luego $u_n-1\leq 0$ para todo $n\in \mathbb{Z}^+.$ Así podemos notar que
        \begin{align*}
        \left|\sum_{n=1}^m\frac{1}{2^n}(u_n-1)\right|&\leq \sum_{n=1}^m\frac{1}{2^n}|u_n-1|\\
        &=\sum_{n=1}^m\frac{1}{2^n}(1-u_n)\\
        &=\sum_{n=1}^m\frac{1}{2^n}-\sum_{n=1}^m\frac{1}{2^n}u_n,
        \end{align*}
        Luego si $n\to \infty$ tenemos que 
        $$\left|\sum_{n=1}^\infty\frac{1}{2^n}(u_n-1)\right|\leq 1-f(u)=0,$$
        por lo tanto  
        $$\sum_{n=1}^\infty\frac{1}{2^n}(u_n-1)=0.$$
        Ahora, note que si existe algún $u_n<1$ la suma de arriba sería negativa, no igual a 0. Por lo que $u_n=1$ para todo $n\in \mathbb{Z}^+,$ pero esto implicaria que $u\notin E$, ya que esa sucesión no converge a 0. Luego no puede existir una sucesión $u$ que cumpla lo mencionado.

    \end{sols}
\end{enumerate}




