%!TEX root = ../main.tex

Considere $E = c_0$, donde
\[
c_0 = \left\{ u = \{u_n\}_{n \geq 1} :\text{tales que } u_n \in \mathbb{R}, \ \lim_{n \to \infty} u_n = 0 \right\}.
\]
Es decir, $c_0$ es el conjunto de las secuencias reales que tienden a cero. Dotamos a este espacio con la norma
$
\|u\|_{\ell^\infty} = \sup_{n \in \mathbb{Z}^+} |u_n|.
$
Considere el funcional $f : E \to \mathbb{R}$ dado por
\[
f(u) = \sum_{n=1}^{\infty} \frac{1}{2^n} u_n.
\]

\begin{enumerate}
    \item[(i)] Muestre que $f \in E^*$ y calcule $\|f\|_{E^*}$.
    \item[(ii)] ¿Es posible encontrar $u \in E$ tal que $\|u\| = 1$ y $f(u) = \|f\|_{E^*}$?
\end{enumerate}



Sean $(E, \| \cdot \|_E)$ y $(F, \| \cdot \|_F)$ espacios vectoriales normados. Suponga que $F$ es un espacio de Banach.

\begin{itemize}
    \item[•] Muestre que $\mathcal{L}(E, F)$ es un espacio de Banach con la norma usual de $\mathcal{L}(E, F)$.
    \item[•] En particular, concluya que $E^* = \mathcal{L}(E, \mathbb{R})$ y $E^{**} = \mathcal{L}(E^*, \mathbb{R})$ son espacios de Banach.
\end{itemize}
