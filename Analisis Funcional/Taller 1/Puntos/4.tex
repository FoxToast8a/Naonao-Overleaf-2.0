%!TEX root = ../main.tex
 Sean $(E, \| \cdot \|_E)$ y $(F, \| \cdot \|_F)$ espacios vectoriales normados. Suponga que $F$ es un espacio de Banach. Muestre que $\mathcal{L}(E, F)$ es un espacio de Banach con la norma usual de $\mathcal{L}(E, F)$. En particular, concluya que $E^* = \mathcal{L}(E, \mathbb{R})$ y $E^{**} = \mathcal{L}(E^*, \mathbb{R})$ son espacios de Banach.
\hfill

\begin{proof}
\hfill

Sea \( (E, \|\cdot\|_E) \) un espacio normado y sea \( (F, \|\cdot\|_F) \) un espacio de Banach. Consideremos el conjunto \( \mathcal{L}(E, F) \) de todas las aplicaciones lineales y continuas de \( E \) en \( F \), provisto de la norma definida por
\[
\|T\| := \sup_{\|x\|_E \leq 1} \|T(x)\|_F.
\]
Queremos demostrar que \( \mathcal{L}(E, F) \), con esta norma, es un espacio de Banach.

Sea \( (T_n)_{n \in \mathbb{N}} \subseteq \mathcal{L}(E, F) \) una sucesión de Cauchy. Por definición, para todo \( \varepsilon > 0 \), existe \( N \in \mathbb{N} \) tal que para todo \( n, m \geq N \),
\[
\|T_n - T_m\| < \varepsilon,
\]
es decir, para todo \( x \in E \) con \( \|x\|_E \leq 1 \), se tiene que,
\[
\|T_n(x) - T_m(x)\|_F < \varepsilon.
\]

Ahora, sea \( x \in E \) arbitrario (no necesariamente de norma menor o igual que uno), tenemos que para todo \( n, m \geq N \), se cumple que,
\[
\|T_n(x) - T_m(x)\|_F = \|(T_n - T_m)(x)\|_F \leq \|T_n - T_m\| \cdot \|x\|_E.
\]
Dado \( \varepsilon > 0 \), si \( x \neq 0 \), se puede tomar \( \delta := \dfrac{\varepsilon}{ \|x\|_E} \), y por ser \( (T_n) \) de Cauchy, existe \( N \in \mathbb{N} \) tal que para todo \( n, m \geq N \),
\[
\|T_n - T_m\| < \delta = \frac{\varepsilon}{\|x\|_E},
\]
lo que implica
\[
\|T_n(x) - T_m(x)\|_F < \varepsilon.
\]
En el caso \( x = 0 \), se tiene trivialmente que \( T_n(0) = 0 \) para todo \( n \), por lo que la sucesión es constante y, en particular, de Cauchy. Así, se concluye que para todo \( x \in E \), la sucesión \( (T_n(x))_{n \in \mathbb{N}} \subseteq F \) es de Cauchy.

Como \( F \) es un espacio de Banach, existe un elemento \( T(x) \in F \) tal que
\[
T_n(x) \to T(x) \in F,
\]
esto define una aplicación \( T: E \to F \) mediante
\[
T(x) := \lim_{n \to \infty} T_n(x).
\]

Veamos que \( T \) es lineal. Sean \( x, y \in E \) y \( \lambda \in \mathbb{K} \) (donde \( \mathbb{K} = \mathbb{R} \) o \( \mathbb{C} \)). Como cada \( T_n \) es lineal, se tiene
\[
T_n(\lambda x + y) = \lambda T_n(x) + T_n(y),
\]
y como los límites existen en \( F \), se concluye que
\[
T(\lambda x + y) = \lim_{n \to \infty} T_n(\lambda x + y) = \lim_{n \to \infty} (\lambda T_n(x) + T_n(y)) = \lambda T(x) + T(y),
\]
es decir, \( T \) es lineal.

Mostremos ahora que \( T \) es acotada. Como \( (T_n) \) es Cauchy en \( \mathcal{L}(E, F) \), existe una constante \( M > 0 \) tal que \( \|T_n\| \leq M \) para todo \( n \in \mathbb{N} \). Entonces, para todo \( x \in E \),
\[
\|T_n(x)\|_F \leq \|T_n\| \cdot \|x\|_E \leq M \|x\|_E,
\]
y pasando al límite cuando \( n \to \infty \),
\[
\|T(x)\|_F \leq M \|x\|_E.
\]
Esto demuestra que \( T \in \mathcal{L}(E, F) \), es decir, \( T \) es lineal y continua.

Finalmente, veamos que \( T_n \to T \) en \( \mathcal{L}(E, F) \). Dado \( \varepsilon > 0 \), existe \( N \in \mathbb{N} \) tal que para todo \( n, m \geq N \),
\[
\|T_n - T_m\| < \varepsilon.
\]
Fijado \( n \geq N \), y tomando el límite cuando \( m \to \infty \), se obtiene
\[
\|T_n - T\| = \sup_{\|x\|_E \leq 1} \|T_n(x) - T(x)\|_F \leq \varepsilon.
\]
Por tanto, \( \|T_n - T\| \to 0 \), lo que implica que \( T_n \to T \) en \( \mathcal{L}(E, F) \).

\medskip

Concluimos que \( \mathcal{L}(E, F) \), con la norma \( \|\cdot\| \), es un espacio de Banach. Además, al ser $\mathbb{R}$ un espacio normado y de Banach con la norma usual, entonces \( \mathcal{L}(E, \mathbb{R}) \) y \( \mathcal{L}(E^{*}, \mathbb{R}) \) son espacios de Banach.
\end{proof}
