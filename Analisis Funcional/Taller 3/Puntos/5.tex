%!TEX root = ../main.tex
 Sea $E$ un espacio de Banach
 \begin{itemize}
     \item[(a)] Demuestre que existe un espacio topológico compacto $K$ y una isometría de $E$ en $\left(C(K),\|\cdot\|_{\infty}\right)$.
     \begin{proof}
         Considere el conjunto $K=\mathcal{B}_{E^{\star}}=\{f \in E^{\star}:\|f\|_{E^{\star}}\leq 1\}$. Luego $K$ es un espacio topológico compacto en la topología débil$^{\star}$. Definamos la función $$T \colon (E, \|\cdot\|_E) \to(C(K), \|\cdot\|_\infty)$$ donde $x \mapsto Tx \colon \mathcal{B}_{E^{\star}} \to \mathbb{R}$
y $f \mapsto (Tx)(f) =f(x)=J_x(f)$, por definición de las funciones $J_x$ sabemos que $Tx$ es continua en la topología débil $\star$, ahora veamos que $T$ es lineal, sean $x, y \in E$ y $\alpha, \beta \in \mathbb{R}$, entonces,
\[
T(\alpha x + \beta y)(f) = \langle f, \alpha x + \beta y \rangle = \alpha \langle f, x \rangle + \beta \langle f, y \rangle = \alpha (T(x)(f)) + \beta (T(y)(f)).
\]

Por último, veamos que $T$ es una isometría.

\[
\|T(x)\|_\infty = \sup_{f \in K} |T(x)(f)| = \sup_{\substack{f \in E^{\star} \\ \|f\|_{E^{\star}} \leq 1}} |\langle f, x \rangle|\leq \|x\|,
\]

como $T$ es lineal, acotada tenemos que $T$ es continua. Además, por un corolario de la forma analítica de Hahn-Banach, tenemos que para cada $x \in E$ existe $f \in E^{\star}$ tal que
\[
\langle f, x \rangle = \|x\|, \quad \text{con } \|f\| = 1,
\]
por lo que tenemos $\|T(x)\|_\infty = \|x\|$, así $T$ es isometría.

     \end{proof}




     \item[(b)] Asuma que $E$ es separable. Entonces muestre que existe una isometría de $E$ en $l^{\infty}$ (vea el Ejercicio 14 para la definición del espacio).
     \begin{proof}
         
    
Como \( E \) es separable, la bola unitaria es cerrada \( K = B_{E^{\star}} \) del dual \( E^{\star} \) es compacta y metrizable en la topología débil \(\star\) por el Teorema de Banach-Alaoglu y la separabilidad de \( E \). Además, sabemos que todo espacio métrico compacto es segundo-contable y, por tanto, separable. Así, existe un subconjunto denso numerable \( \{f_n\}_{n=1}^\infty \subseteq K \).

Definamos
\[
T : E \to \ell^\infty, \quad x \mapsto T(x) := \{ f_n(x) \}_{n=1}^\infty.
\]

Veamos que $T$ está bien definido, como $\|f_n\| \leq 1$ para todo $n \in \mathbb{N}$, tomando un $x$ fijo, tenemos que
\[
\sup_{n \in \mathbb{N}} |f_n(x)| \leq \|x\| \sup_{n \in \mathbb{N}} \|f_n\| \leq M \|x\| \quad \text{donde } M \in \mathbb{R}, M \geq 1.
\]

Ahora veamos que $T$ es lineal. Sean $x, y \in E$, y $\alpha, \beta \in \mathbb{R}$
\[
T(\alpha x + \beta y) = \{ f_n(\alpha x + \beta y) \} = \alpha \{ f_n(x) \} + \beta \{ f_n(y) \},
\]
luego $T$ es lineal. Por lo que $T$ es continua ya que es lineal y acotada. Ahora, sólo nos falta ver que $T$ es una isometría, por el corolario de Hahn-Banach en forma analítica, existe $f \in K \subseteq E^{\star}$ tal que $|f(x)| = \|x\|$, al ser $f_n$ denso en $K $, existe una subsucesión $\{f_{n_k}\}$ tal que $f_{n_k} \to f$ en $\sigma(E^{\star}, E)$, y en particular $f_{n_k}(x) \to f(x)$, así,
\[
|f(x)| = \limsup_{k \to \infty} |f_{n_k}(x)| \leq \sup_{n \in \mathbb{N}} |f_n(x)| = \|T(x)\|_\infty \leq \|x\|.
\]

Con lo que 
\[
\|T(x)\|_\infty = \|x\|,
\]
concluyendo así que $T(x)$ es una isometría.

 \end{proof}
 \end{itemize}
 