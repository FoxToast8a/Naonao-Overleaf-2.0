%!TEX root = ../main.tex
  Sea $E$ un espacio de Banach reflexivo. Sea  $a:E \times E \to \mathbb{R}$ una forma bilineal que es continua, es decir, existe $M>0$ tal que $|a(x, y)| \leq M\|x\|\|y\|$, para todo $x, y \in E$. Asuma que a es coerciva, esto es, existe $\alpha>0$ tal que para todo $x \in E$
$$
a(x, x) \geq \alpha\|x\|^2
$$
\begin{itemize}
    \item[(a)]Dado $x \in E$, defina $A_x(y)=a(x, y)$, para todo $y \in E$. Muestre que $A_x \in E^{\star}$, para cada $x \in E$. Además, concluya que la función $x \mapsto A(x)=A_x$ satisface $A \in \mathcal{L}\left(E, E^{\star}\right)$. 
    \begin{sproof}
      Para ver que $A_x\in E^*$ tenemos que ver que sea lineal y acotada. Sean $y_1,y_2\in E$ y $\lambda\in \mathbb{R},$ tenemos que
      \begin{align*}
        A_x(y_1+\lambda y_2)&=a(x,y_1+\lambda y_2)\\
        &=a(x,y_1)+\lambda a(x,y_2)\\
        &=A_x(y_1)+\lambda A_x(y_2).
      \end{align*}
      Note que esto se sigue del hecho de que $a$ es bilineal, por lo que es lineal en la segunda componente. Ahora note que por la continuidad de $a$ tenemos que
      \begin{align*}
        |A_x(y)|&=|a(x,y)|\\
        &\leq M\|x\|\|y\|\\
        &=M_1\|y\|.
      \end{align*}
      Como para cada $x\in E$ se define el $A_x$, en cada operador $\|x\|$ es un número, por lo que es correcto tomar $M_1=M\|x\|>0$ para cada $A_x,$ así tenemos que $A_x\in E^*.$\\

      Ahora queremos ver que la función
      \begin{align*}
        A:&E\to E^*\\
        &x\mapsto A(x)=A_x
      \end{align*}
      pertenece a $\mathcal{L}\left(E, E^{\star}\right).$ Claramente esta función está bien definida ya que $A_x\in E^*$. Luego nuevamente tenemos que ver que es lineal y acotada, para la linealidad, sean $x_1,x_2\in E$ y $\lambda\in \mathbb{R},$ luego para $y\in E$
      \begin{align*}
        A_{x_1+\lambda x_2}(y)&=a(x_1+\lambda x_2,y)\\
        &=a(x_1,y)+\lambda a(x_2,y)\\
        &=A_{x_1}(y)+\lambda A_{x_2}(y).
      \end{align*}
      Nuevamente esto se tiene del hecho de que $a$ es bilineal. Por lo que $A_{x_1+\lambda x_2}=A_{x_1}+\lambda A_{x_2}$ pero esto es $A(x_1+\lambda x_2)=A(x_1)+\lambda A(x_2)$, obteniendo así que  $A$ es lineal. Ahora veamos que $A$ está acotada
      \begin{align*}
        \|A(x)\|&=\|A_x\|\\
        &=\sup_{\substack{y\in E\\\|y\|\leq 1}}|A_x(y)|\\
        &=\sup_{\substack{y\in E\\\|y\|\leq 1}}|a(x,y)|\\
        &\leq \sup_{\substack{y\in E\\\|y\|\leq 1}}M\|x\|\|y\|\\
        &=M\|x\|\sup_{\substack{y\in E\\\|y\|\leq 1}}\|y\|\\
        &\leq M\|x\|.
      \end{align*}
      Esto es gracias a la continuidad de $a.$ Así concluimos que $A\in\mathcal{L}\left(E, E^{\star}\right).$ 

    \end{sproof}
    \item[(b)] Muestre que A como en (a) es una función sobreyectiva.
    \begin{sproof}
      Primero veamos que la desigualdad $\|A_x\|\geq \alpha\|x\|$ se tiene para todo $x\in E$, es claro que cuando $x=0$ la desigualdad es cierta ya que $\|x\|=0$. Ahora si $x\neq 0$ tenemos que
      \begin{align*}
        \|A_x\|&=\sup_{\substack{y\in E\\y\neq 0}}\dfrac{|A_x(y)|}{\|y\|}\\
        &\geq \dfrac{|A_x(x)|}{\|x\|}\\
        &= \dfrac{|a(x,x)|}{\|x\|}\\
        &\geq \dfrac{\alpha\|x\|^2}{\|x\|}\\
        &=\alpha\|x\|.
      \end{align*}
       Ahora con ayuda de esta desigualdad probemos que $R(A)$ es cerrado. Sea $f\in\overline{R(A)},$ luego existe una sucesión $\{A_{x_n}\}_{n\in\mathbb{N}}\subseteq R(A)$ tales que $A_{x_n}\to f$, note que tomamos la sucesión de esa forma ya que son elementos del rango de la aplicación $A.$ Como esta aplicación es lineal y por la desigualdad anterior tenemos que
       \begin{align*}
         \|A_{x_n}-A_{x_m}\|&=\|A_{x_n-x_m}\|\\
         &\geq \alpha\|x_n-x_m\|.
       \end{align*}
       Así, $\|x_n-x_m\|\leq\dfrac{1}{\alpha}\|A_{x_n}-A_{x_m}\|$, como $\{A_{x_n}\}$ es convergente y estamos en un espacio normado, esta secuencia es Cauchy. Dado $\varepsilon>0$ existe $N\in \mathbb{N}$ tal que si $n,m\geq M$, $\|A_{x_n}-A_{x_m}\|<\varepsilon$, esto implica que $\|x_n-x_m\|<\dfrac{\varepsilon}{\alpha}$. Así la sucesión $\{x_n\}$ es Cauchy en $E$, pero este espacio es Banach, por lo que $x_n\to x\in E $. Luego podemos considerar la función $A(x)=A_x.$ Note que
       \begin{align*}
         \|A_x-f\|&=\|A_x-A_{x_n}+A_{x_n}-f\|\\
         &\leq \|A_x-A_{x_n}\|+\|A_{x_n}-f\|\\
         &=\|A_{x-x_n}\|+\|A_{x_n}-f\|\\
         &\leq\|A\|\|x-x_n\|+\|A_{x_n}-f\|,
       \end{align*}
       por lo que cuando $n\to\infty$ tenemos por la convergencia que $\|x-x_n\|\to 0$ y $\|A_{x_n}-f\|\to 0$. Concluyendo así que $A_x=f$, por lo que $f\in R(A)$, mostrando que el rango de la aplicación $A$ es cerrado.

       Ahora veamos por contradicción que la función es sobreyectiva. Como $R(A)=\overline{R(A)}$, si no es sobreyectiva, $\overline{R(A)}\neq E^*$, luego por Hahn-Banach existe $f\in E^{**}$ tal que $f\not\equiv 0$ y $f|_{R(A)}=0.$ Esto quiere decir que para cualquier $A_x\in R(A)$, con $x\in E$, tenemos que 
       $$\langle f,A_x\rangle=0.$$
       Pero como $E$ es reflexivo sabemos que la aplicación canónica $J:E\to E^{**}$ es sobreyectiva, como $f\in E^{**}$, existe un $x_0\in E$, tal que $f=J_{x_0}.$ Luego
       \begin{align*}
         0&=\langle f,A_x\rangle\\
         &=\langle J_{x_0},A_x\rangle\\
         &=\langle A_x,x_0\rangle\\
         &=a(x,x_0).
       \end{align*}
       Por lo que si tomamos $x=x_0$, como $a$ es coerciva tenemos que $0=a(x_0,x_0)\geq \alpha\|x_0\|^2.$ de eso concluimos que $x_0=0,$ pero esto implicaría que $f=J_{x_0}=J_0=0,$ una contradicción ya que $f$ era no nulo. Así concluimos que $R(A)=\overline{R(A)}=E^*$ mostrando que $A$ es sobreyectiva.

    \end{sproof}
    \item[(c)] Deduzca que para cada $f \in E^{\star}$, existe un único $x \in E$ tal que $a(x, y)=\langle f, y\rangle$, $\forall y \in E$. Esto es, la forma bilineal coerciva a representa todo funcional lineal continuo.
    \begin{sproof}
      Tomemos $f\in E^*$, por $(b)$, como $A$ es sobreyectiva, existe $x_1\in E$ tal que $A(x_1)=A_{x_1}=f$, es decir que para todo $y\in E$, $a(x_1,y)=A_{x_1}(y)=f(y).$ Si no existe otro además de $x_1$ hemos acabado. En caso contrario, suponga que existe un $x_2\in E$ tal que $a(x_2,y)=f(y)$ para todo $y$, luego como $a$ es bilineal
      \begin{align*}
        0&=f(y)-f(y)\\
        &=a(x_1,y)-a(x_2,y)\\
        &=a(x_1-x_2,y).
      \end{align*}
      Como es para todo $y\in E$, si tomamos $y=x_1-x_2$, tenemos que $a(x_1-x_2,x_1-x_2)=0$, pero como $a$ es coerciva, existe $\alpha>0$ tal que
      $$0=a(x_1-x_2,x_1-x_2)\geq\alpha\|x_1-x_2\|^2.$$
      Así tenemos que $\|x_1-x_2\|^2=0.$ Por lo que $x_1-x_2=0$, así $x_1=x_2$, concluyendo así la unicidad. 
    \end{sproof}
\end{itemize}
