%!TEX root = ../main.tex
  Considere el espacio $L^p\left(\mathbb{R}^n\right), 1 \leq p \leq \infty$. Sean
$$
f_0(x)=\left\{\begin{array}{ll}
|x|^{-\alpha}, & \text { si }|x| \leq 1, \\
0, & \text { si }|x|>1 .
\end{array} \quad f_1(x)= \begin{cases}0, & \text { si }|x| \leq 1, \\
|x|^\alpha, & \text { si }|x|>1 .\end{cases}\right.
$$
\begin{itemize}
  \item[(I)] ¿Para qué valores de $\alpha \in \mathbb{R}, f_0 \in L^p\left(\mathbb{R}^n\right)$ ?\\
  \begin{sol}
Sea,
  \[
\| f_c \|_{L^p} = \left( \int_{\mathbb{R}^n} |f_c|^p \, d\overline{x} \right)^{1/p} = \left( \int_{\frac{1}{2} \times 1+1} |x_1^{-p}|^q \, d\overline{x} \right)^{1/p}
\]
Usando coordenadas polares \( x = r\theta \), \( \theta \in S^{n-1} \) donde \( r = |x| \in [0, \infty) \)

\begin{align*}
  \| f_c \|_{L^p} &= \left( \int_{\frac{1}{2} \times 1+1} |x_1^{-p}|^q \, dx \right)^{1/p}\\
  & = \left( \int_0^1 \int_{S^{n-1}} r^{-p\alpha} \, dr \, dv(\theta) \, dt \right)^{1/p}\\ 
  &= \left( \int_{S^{n-1}} dv(\theta) \int_0^1 r^{n-1-p\alpha} \, dr \right)^{1/p}
.\end{align*}



Sabemos que
\[
\int_\varepsilon^1 r^a \, dr = \lim_{\varepsilon \to 0} \frac{r^{a+1}}{a+1} \bigg|_\varepsilon^1 = \lim_{\varepsilon \to 0} \left( \frac{1}{a+1} - \frac{\varepsilon^{a+1}}{a+1} \right)
\]
si \( a + 1 > 0 \) converge, por lo que cuando \( a + 1 < 0 \) diverge. Luego, para que 
\[
\int_0^1 r^{n-1-p\alpha} \, dr\]
converja se tiene que cumplir que $ n-1-p\alpha > -1 $ es decir, $n - p\alpha > 0$ por lo que, para \( \alpha < \frac{n}{p} \) la integral es finita, es decir, \( f_c \in L^p(\mathbb{R}^n) \).
\end{sol}
\item[(II)] ¿Para qué valores de $\alpha \in \mathbb{R}, f_1 \in L^p\left(\mathbb{R}^n\right)$ ?\\
\begin{sol}
Sea,

\[
\| f \|_{L^p} = \left( \int_{\mathbb{R}^n} |f|^p \, dx \right)^{1/p} = \left( \int_{|x|>1} |x|^{\alpha p} \, dx \right)^{1/p}
\]

haciendo el cambio a coordenadas polares \( x = r\theta \) con \( r = |x| \in [1, \infty) \) y \( \theta \in S^{n-1} \)

\[\| f \|_{L^p} =\left( \int_1^\infty \int_{S^{n-1}} r^{\alpha p} r^{n-1} \, dv(\theta) \, dr \right)^{1/p} = \left( \int_{S^{n-1}} dv(\theta) \int_1^\infty r^{\alpha p + n - 1} \, dr \right)^{1/p}
\]

La integral \( \int_1^\infty r^{\alpha p + n - 1} \, dr \) converge si y sólo si,

\[
\alpha p + n - 1 < -1 \quad \text{ por lo que } \quad \alpha p + n < 0 \quad \text{ así }\quad \alpha < -\frac{n}{p}
\]

Por lo tanto, para \( \alpha < -\dfrac{n}{p} \) la integral es finita, es decir, \( f \in L^p(\mathbb{R}^n) \).
\end{sol}

\item[(III)] ¿Para qué valores de $\alpha \in \mathbb{R}, \frac{1}{1+|x|^\alpha} \in L^p\left(\mathbb{R}^n\right)$ ?
\begin{sol}
Tenemos que 
\begin{align*}
  \| f \|_{L^p}^{p} &= \int_{\mathbb{R}^n} \left|\frac{1}{1 + |x|^\alpha}\right|^{p}\, dx \\
   &= \int_{\mathbb{R}^n} \dfrac{1}{(1 + |x|^\alpha)^p} \, dx \\
   &= \int_{\|x\|\leq 1} \dfrac{1}{(1 + |x|^\alpha)^p} \, dx + \int_{\|x\|\geq 1} \dfrac{1}{(1 + |x|^\alpha)^p} \, dx
\end{align*}

Sabemos que cuando $\|x\|\leq 1$ la integral converge entonces debemos preocuparnos por la condición cuando $\|x\| \geq 1$.

Tenemos que 
\begin{align*}
  \int_{\|x\|\geq 1} \dfrac{1}{(1 + |x|^\alpha)^p} \, dx \leq  \int_{\|x\|\geq 1} \dfrac{1}{(|x|^\alpha)^p} \, dx
\end{align*}

luego, tenemos que cambiando a coordenadas polares la última integral,
\[ 
 \int_{0}^{\infty} \left( \frac{1}{r^{\alpha}} \right)^{p} r^{n-1} \, dr = \int_{0}^{\infty} \frac{r^{n-1}}{(r^{\alpha})^{p}} \, dr 
\]

por lo que se tiene la convergencia con las siguientes condiciones
\[ 
n - 1 < \alpha p 
\]
\[ 
n \leq \alpha p 
\]
\[ 
\frac{n}{p} \leq \alpha. 
\]

Por lo que tenemos una aproximación de la cota que deseamos para $\alpha$, ahora acotemos inferiormente la integral

\begin{align*}
    \int_{\|x\|\geq 1} \dfrac{1}{(1 + |x|^\alpha)^p} \, dx,
  \end{align*} 
  luego como $\|x\|\geq 1$  entonces 
  \begin{align*}
    \int_{\|x\|\geq 1} \dfrac{1}{(1 + |x|^\alpha)^p} \, dx
    &\geq \int_{\|x\|\geq 1} \dfrac{1}{(2 |x|^\alpha)^p} \, dx
  \end{align*} 
  cambiando a coordenadas polares,
\[ 
 \int_{0}^{\infty} \left( \frac{1}{2r^{\alpha}} \right)^{p} r^{n-1} \, dr = \int_{0}^{\infty} \frac{r^{n-1}}{(2r^{\alpha})^{p}} \, dr 
\]
como $p$ es fijo entonces solo debemos poner condiciones sobre $\alpha p$, así, 
\[ 
n - 1 < \alpha p 
\]
\[ 
n \leq \alpha p 
\]
\[ 
\frac{n}{p} \leq \alpha. 
\]
por lo tanto, para \( \alpha \geq \dfrac{n}{p} \) la integral es finita, es decir, \( f \in L^p(\mathbb{R}^n) \). 
\end{sol}
\end{itemize}
