%!TEX root = ../main.tex
  Considere el espacio $L^p\left(\mathbb{R}^n\right), 1 \leq p \leq \infty$. Sean
$$
f_0(x)=\left\{\begin{array}{ll}
|x|^{-\alpha}, & \text { si }|x| \leq 1, \\
0, & \text { si }|x|>1 .
\end{array} \quad f_1(x)= \begin{cases}0, & \text { si }|x| \leq 1, \\
|x|^\alpha, & \text { si }|x|>1 .\end{cases}\right.
$$
\begin{itemize}
  \item[(I)] ¿Para qué valores de $\alpha \in \mathbb{R}, f_0 \in L^p\left(\mathbb{R}^n\right)$ ?
\item[(II)] ¿Para qué valores de $\alpha \in \mathbb{R}, f_1 \in L^p\left(\mathbb{R}^n\right)$ ?
\item[(III)] ¿Para qué valores de $\alpha \in \mathbb{R}, \frac{1}{1+|x|^\alpha} \in L^p\left(\mathbb{R}^n\right)$ ?
\end{itemize}
