%!TEX root = ../main.tex
 \begin{itemize}
  
\item[(I)] Sea $\mathbb{R}$ con la $\sigma$-álgebra de Borel $\mathcal{B}(\mathbb{R})$.
\begin{itemize}
\item[(a)] Dado $x_0 \in \mathbb{R}$, considere $\delta_{x_0}$ la medida de Dirac centrada en $x_0$ dada por: $\delta_{x_0}(A)=1$ si $x_0 \in A$, y $\delta_{x_0}(A)=0$ si $x_0 \notin A$, para cada $A \in \mathcal{B}(\mathbb{R})$. Muestre que $\delta_{x_0}$ es una medida.
\begin{sproof}
    Es claro que $\delta_{x_0}:\mathcal{B}(\mathbb{R})\to [0,\infty]$ ya que las únicas imágenes de la función son 0 y 1. Ahora comprobemos las dos condiciones.\\

    Si $A=\varnothing$, claramente $x_0\notin A$, luego por la definición $\delta_{x_0}(A)=0,$ como queríamos.
Ahora, si \(x_0 \in \bigcup_{i=1}^{\infty} A_i\), entonces \(x_0 \in A_j\) para algún \(j\) único, puesto que los conjuntos son disjuntos. Luego, 
\[
\delta_{\chi_0}(A_i) = 
\begin{cases}
1 & \text{si } i = j, \\
0 & \text{si } i \neq j.
\end{cases}
\]
así,
\[
\sum_{i=1}^{\infty} \delta_{\chi_0}(A_i) = 1 = \delta_{\chi_0}\left(\bigcup_{i=1}^{\infty} A_i\right).
\]
por lo que $\delta$ es una medida.
\end{sproof}
\item[(b)] Sea $f: \mathbb{R} \rightarrow \mathbb{R}$ una función medible. Muestre que
$$
\int_{\mathbb{R}} f(x) d \delta_{x_0}=f\left(x_0\right)
$$
\begin{proof}
Para probar esto primero veamos que esto se tiene para funciones simples positivas, sea \( f(x) = \sum_{i=1}^n a_i \chi_{A_i}(x) \), donde 
\[
\chi_{A_i}(x) = \begin{cases} 
1, & \text{si } x \in A_i, \text{ con } a_i \geq 0, A_i \in \mathcal{B}(\mathbb{R}) \\
0, & \text{si } x \notin A_i
\end{cases}
\]
para todo \( i = 1,\dots,n \) y \( A_i \cap A_j = \emptyset \) si \( i \neq j \).

Por definición,
\[
\int_{\mathbb{R}} f(x) \, d\delta_x = \int_{\mathbb{R}} \left( \sum_{i=1}^n a_i \chi_{A_i}(x) \right) d\delta_x = \sum_{i=1}^n a_i \delta_x(A_i).
\]
Como \( A_i \cap A_j = \emptyset \) para \( j \neq i \), entonces para algún \( j \), \( \delta_x(A_j) = 1 \) y en el resto es \( 0 \), por lo  tanto,
\[
\int_{\mathbb{R}} f(x) \, d\delta_x = \sum_{i=1}^n a_i \chi_{A_i}(x)= a_j  = f(x).
\]

Por lo que la proposición es cierta para funciones simples positivas, ahora veamos que se cumple para funciones medibles no negativas. Tomemos \( f: \mathbb{R} \to [0, \infty] \) una función medible no negativa. Entonces, por un teorema visto en clase, existe \( \{f_n\}_{n=1}^{\infty} \) una sucesión de funciones simples tales que
\begin{align*}
0\leq f_n(x)\leq f_{n+1} \leq f(x) \text{ y }  \lim_{n \to \infty} f_n(x) = f(x)  
\end{align*}


para cada \( x \in \mathbb{R} \). Luego, por el teorema de la convergencia monótona,
\[
\int_{\mathbb{R}} f(x) \, d\delta_x = \lim_{n \to \infty} \int_{\mathbb{R}} f_n(x) \, d\delta_x = \lim_{n \to \infty} f_n(x_0) = f(x_0),
\]
por lo que se tiene para funciones medibles no negativas.

Por último, veamos el caso general. Para \( f: \mathbb{R} \to \mathbb{R} \) una función medible, recordamos que \( f = f^+ - f^- \), donde \( f^+ \), \( f^- \) son funciones medibles no negativas. Luego,
\[
\int_{\mathbb{R}} f(x) \, d\delta_x = \int_{\mathbb{R}} f^+(x) \, d\delta_x - \int_{\mathbb{R}} f^-(x) \, d\delta_x.
\]
Por lo anteriormente probado, como la parte negativa y la parte positiva son funciones medibles no negativas,
\[
\int_{\mathbb{R}} f(x) \, d\delta_x = f^+(x_0) - f^-(x_0) = f(x_0).
\]

\end{proof}

\item[(c)] De un ejemplo de una función que sea integrable con la medida $\delta_{x_0}$ para algún $x_0$, pero que no sea integrable con la medida de Lebesgue.\\
\begin{sol}
Sea $f: \mathbb{R} \to \mathbb{R}$ tal que $f(x) = 1$ para todo $x \in \mathbb{R}$. Entonces
$$
\displaystyle \int_{\mathbb{R}} 1 \, d\lambda(x) = \lambda(\mathbb{R})
$$
mientras que

\[ 
\int_{\mathbb{R}} 1 \, d\delta_0(x) = 1 < \infty 
\]

por lo cual, $f$ es integrable respecto a la medida de Dirac centrada en 0, pero no lo es respecto a la medida de Lebesgue.
\end{sol}

\end{itemize}
\item[(II)] Sea $\mathbb{N}=\{1,2,3, \ldots\}$ con la $\sigma$-álgebra $\mathcal{P}(\mathbb{N})$.
\begin{itemize}
  

\item[(a)] Considere la medida contadora $\mu$ dada por: $\mu(A)=\operatorname{cardinal}(A)$ si $A$ es finito y $\mu(A)=\infty$ caso contrario, para cada $A \in \mathcal{P}(\mathbb{N})$. Muestre que $\mu$ es una medida.
\begin{proof}
Para ver que es medida, primero veamos que la medida del vacío es nula, claramente $\mu(\emptyset) = \text{card}(\emptyset) = 0.$ Sean \( A_i \subseteq \mathcal{P}(\mathbb{N}) \) con \( A_i \cap A_j = \emptyset \) para \( i \neq j \), si algún \( A_i \) es infinito, entonces \( \bigcup_{i=1}^{\infty} A_i \) es infinito y así,
\[ \mu\left(\bigcup_{i=1}^{\infty} A_i\right) = \infty = \sum_{i=1}^{\infty} \mu(A_i), \]
ya que \( \mu(A_i) = \infty \) para algún \( i \). Si  todos los \( A_i \) son finitos, entonces \( \mu(A_i) = \text{card}(A_i) \), y como son disjuntos
\[ \mu\left(\bigcup_{i=1}^{\infty} A_i\right) = \text{card}\left(\bigcup_{i=1}^{\infty} A_i\right) = \sum_{i=1}^{\infty} \text{card}(A_i). \]

Para algún \( N \in \mathbb{Z}^+ \), si \( n \geq N \) entonces \( A_n = \emptyset \), por lo que
\[ \mu\left(\bigcup_{i=1}^{\infty} A_i\right)= \sum_{i=1}^{\infty} \mu(A_i). \]
              
\end{proof}
                

\item[(b)] Dada $f: \mathbb{N} \rightarrow \mathbb{R}$ una función medible, es decir, $f$ es una secuencia, $f=\left\{a_j\right\}_{j \in \mathbb{N}}$, para algunos $a_j \in \mathbb{R}, j \in \mathbb{N}$. Muestre que si $f$ es integrable (es decir, $\int_{\mathbb{N}}|f| d \mu<\infty$ ), entonces
$$
\int_{\mathbb{N}} f d x=\sum_{j=1}^{\infty} a_j
$$
\begin{proof}

Teniendo en cuenta que \( N = \bigcup^{\infty}_{i=1} \{n\} \), donde para cada \( n \in \mathbb{N} \), \( \{n\} \) es un conjunto medible. Por lo que, podemos definir las funciones simples \( \chi_{\{i\}} \) para todo \( j \in \mathbb{N} \), como toda función medible se puede escribir como \( f = \{a_j\}_{j \in \mathbb{N}} \), si tomamos \( f \) no negativa entonces
\[
f(x) = \sum_{j=1}^{\infty} a_j \chi_{i_j}(x), \quad \text{con } a_j \geq 0, \text{ para todo } j \in \mathbb{N},
\]
así,

\[
\int_N f(x) \, d\mu = \int_N \left( \sum_{j=1}^{\infty} a_j \chi_{i_j}(x) \right) d\mu = \sum_{j=1}^{\infty} a_j \mu(i_j) = \sum_{j=1}^{\infty} a_j.
\]

Ahora, si tomamos \( f \) una función medible e integrable, podemos escribirla como \( f = f^+ - f^- \), donde \( f^+ \) y \( f^- \) son medibles, integrables y no negativas. Por lo que,
\begin{align*}
 \int_N f(x) \, d\mu &= \int_N f^+(x) \, d\mu - \int_N f^-(x) \, d\mu = \sum_{j=1}^{\infty} a_j - \sum_{j=1}^{\infty} b_j = \sum_{j=1}^{\infty} c_j 
.\end{align*}
donde $c_j \in \mathbb{R}$, por lo que  $f$ es integrable.


    
\end{proof}

\end{itemize}
\end{itemize}