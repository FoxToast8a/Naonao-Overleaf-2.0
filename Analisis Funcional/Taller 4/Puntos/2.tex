%!TEX root = ../main.tex
  Sea $\Omega \subseteq \mathbb{R}^n$ abierto. Sea $1 \leq p \leq \infty$. Entonces $L^p(\Omega)$ es un espacio de Banach.

  \begin{proof}
Distinguimos los casos \( p = \infty \) y \( 1 \leq p < \infty \).

\subsubsection*{Caso 1: \( p = \infty \)}

Sea \( (f_n) \) una sucesión de Cauchy en \( L^\infty \). Dado un entero \( k \geq 1 \), existe un entero \( N_k \) tal que
\[
\operatorname*{ess\,sup}_{x \in \Omega} |f_n(x) - f_m(x)| = \|f_m - f_n\|_{L^\infty} \leq \frac{1}{k} \quad \text{para todo } m, n \geq N_k,
\]
entonces, por la definición del supremo esencial, para cada \( k \) existe un conjunto de medida nula \( E_k \) tal que,
\[
|f_m(x) - f_n(x)| \leq \frac{1}{k} \quad \text{ para todo } x \in \Omega \setminus E_k, \text{ para todo } m, n \geq N_k.
\]
como cada conjunto \( E_k \) tiene medida nula y la medida de Lebesgue es subaditiva para uniones numerables, se tiene que
\[
\mu\left( \bigcup_{k=1}^\infty E_k \right) \leq \sum_{k=1}^\infty \mu(E_k) = 0,
\]
y por ser no negativa, concluimos que
\[
\mu\left( \bigcup_{k=1}^\infty E_k \right) = 0.
\]
Sea entonces \( E := \bigcup_{k=1}^\infty E_k \), un conjunto de medida nula. Como \( \Omega \setminus E \subseteq \Omega \setminus E_k \) para todo \( k \), se tiene que para todo \( x \in \Omega \setminus E \),
\[
|f_n(x) - f_m(x)| \leq \frac{1}{k} \quad \text{ para todo } m, n \geq N_k.
\]
Esto muestra que \( (f_n(x)) \) es de Cauchy en \( \mathbb{R} \), espacio completo, por lo que converge a un límite que denotamos por \( f(x) \). Pasando al límite cuando \( m \to \infty \), se obtiene
\[
|f(x) - f_n(x)| \leq \frac{1}{k} \quad \text{ para todo } x \in \Omega \setminus E, \text{ para todo } n \geq N_k.
\]
Entonces,
\[
\operatorname*{ess\,sup}_{x \in \Omega} |f(x) - f_n(x)| = \sup_{x \in \Omega \setminus E} |f(x) - f_n(x)| \leq \frac{1}{k},
\]
y por lo tanto,
\[
\|f - f_n\|_{L^\infty} \leq \frac{1}{k} \quad \text{ para todo } n \geq N_k.
\]

Para ver que \( f \in L^\infty \), notamos que para todo \( x \in \Omega \setminus E \),
\[
|f(x)| \leq |f(x) - f_n(x)| + |f_n(x)| \leq \frac{1}{k} + \|f_n\|_{L^\infty},
\]
de modo que \( f \) es esencialmente acotada, es decir, \( f \in L^\infty \).

En consecuencia, \( f_n \to f \) en \( L^\infty \).

\subsubsection*{Caso 2: \( 1 \leq p < \infty \)}

Sea \( (f_n) \) una sucesión de Cauchy en \( L^p \). Como \( L^p \) es un espacio métrico, basta demostrar que existe una subsucesión que converge en \( L^p \).

Extraemos una subsucesión \( (f_{n_k}) \) tal que
\[
\|f_{n_{k+1}} - f_{n_k}\|_p \leq \frac{1}{2^k} \quad \text{ para todo } k \geq 1.
\]
Esto se construye eligiendo inductivamente \( n_1, n_2, \dots \) con \( n_{k+1} \geq n_k \) tales que \( \|f_m - f_n\|_p \leq \frac{1}{2^{k+1}} \) para todo \( m, n \geq n_{k+1} \), lo cual es posible por ser \( (f_n) \) Cauchy.

Definimos \( f_k := f_{n_k} \), y consideramos:
\[
g_n(x) := \sum_{k=1}^n |f_{k+1}(x) - f_k(x)|.
\]
Cada \( g_n \) es medible, no decreciente y cumple
\[
\|g_n\|_p \leq \sum_{k=1}^n \|f_{k+1} - f_k\|_p \leq \sum_{k=1}^\infty \frac{1}{2^k} = 1.
\]
Por el Teorema de Convergencia Monótona, existe una función medible \( g \in L^p \), finita casi en todas partes, tal que
\[
g_n(x) \to g(x) \quad \text{casi en todas partes}.
\]

Por otro lado, para \( m \geq n \geq 2 \), se tiene:
\[
|f_m(x) - f_n(x)| \leq \sum_{k=n}^{m-1} |f_{k+1}(x) - f_k(x)| = g_m(x) - g_{n-1}(x) \leq g(x) - g_{n-1}(x).
\]
Entonces, \( (f_n(x)) \) es de Cauchy en \( \mathbb{R} \) casi en todas partes, y por ser \( \mathbb{R} \) completo, converge a una función \( f(x) \), finita casi en todas partes.

Además, para \( n \geq 2 \),
\[
|f(x) - f_n(x)| \leq g(x) - g_{n-1}(x)\leq g(x) \quad \text{casi en todas partes}.
\]
Elevando a la potencia \( p \),
\[
|f(x) - f_n(x)|^p \leq g(x)^p, \quad \text{con } g^p \in L^1.
\]
Como \( |f_n(x) - f(x)|^p \to 0 \) c.t.p., por el Teorema de Convergencia Dominada se concluye que
\[
\|f_n - f\|_p^p = \int_\Omega |f_n(x) - f(x)|^p \, dx \to 0.
\]
Es decir,
\[
\|f_n - f\|_p \to 0,
\]
Como consecuencia, la función \( f \) pertenece a \( L^p \) por ser el límite en norma de funciones de \( L^p \), es decir,
\[
f \in L^p.
\]

Esto demuestra que \( f_n \to f \) en \( L^p \), y que \( L^p \) es completo.





\end{proof}
