%!TEX root = ../main.tex
Sean $(E,\|\cdot\|)$ y $(F,\|\cdot\|)$ espacios de Banach.
\begin{itemize}
    \item[(i)] Sea $K\subset E$ un subespacio cerrado de $E.$ Definimos la relación sobre $E$ dada por $x\thicksim_Ky$ si y solo si $x-y\in K.$
    \begin{itemize}
        \item[(a)] Muestre que $\thicksim_K$ es una relacion de equivalencia sobre $E.$\\
        \begin{sproof}
            Dado $x\in E,$ como $K$ es subespacio, $x-x=0\in K$, esto implica que $x\thicksim_K x$, mostrando así la reflexividad.\\

            Dados $x,y\in E$, suponga que tenemos que $x\thicksim_K y,$ luego $x-y\in K$, nuevamente como $K$ es subespacio, es cerrado por multiplicacion escalar, así $-(x-y)\in K$, pero $-(x-y)=y-x$, así por definición de la relación tenemos que $y\thicksim_K x,$ mostrando así la simetría.\\

            Por ultimo sean $x,y,z\in E$, con $x\thicksim_K y$ y $y\thicksim_K z,$ por definición $x-y\in K$ y $y-z\in K$, y como $K$ es subespacio es cerrado para la suma, así tenemos que $(x-y)+(y-z)\in K$, pero $(x-y)+(y-z)=x-z$, así tenemos que $x\thicksim_K z,$ así la relación es transitiva. Con esto podemos concluir que la relación es de equivalencia. 

        \end{sproof}
        \item[(b)] Muestre que el espacio cociente $E/K$ es un espacio de Banach con la norma
        $$\|x+K\|_{E/K}=\inf_{k\in K}\|x-k\|,\quad x\in E.$$ 
        Es decir, debe verificar que el espacio cociente es un espacio vectorial, normado, cuya norma lo hace completo.
        \begin{sproof}
            Primero notemos que las operaciones definidas sobre el conjunto $E/K$ son las siguientes
            \begin{align*}
                (x+K)+(y+K)&=(x+y)+K,\\
                \lambda(x+K)&=\lambda x+K.
            \end{align*}
            Las propiedades de espacio vectorial para $E/K$, se heredan del hecho de que $E$ ya es espacio vectorial, solo bastaría verificar que si están bien definidas estas operaciones. Si $x_1+K=x_2+K$ y $y_1+K=y_2+K$, tenemos que $x_1-x_2\in K$ y $y_1-y_2\in K$, pero como es subespacio la suma esta, así $(x_1+y_1)-(x_2+y_2)\in K$, así $(x_1+y_1)+K=(x_2+y_2)+ K$, luego la suma esta bien definida. De manera similar, si $x_1-x_2\in K$, tenemos que al multiplicar por un escalar también esta en $K$, esto es $\lambda x_1-\lambda x_2\in K$, así $\lambda x_1+K=\lambda x_2+K.$\\

            Ahora veamos que la norma definida en el enunciado, efectivamente es norma del espacio $E/K.$ Primero esta norma esta bien definida ya que si $x+K=y+K$, eso quiere decir que $x-y\in K$, luego
            \begin{align*}
                 \|x+K\|&=\inf_{k\in K}\|x-k\|\\
                 &=\inf_{k_1\in K}\|x-(k_1+x-y)\|\\
                 &=\inf_{k_1\in K}\|y-k_1\|\\
                 &=\|y+K\|.
             \end{align*}
             Luego como $\|x-k\|\geq 0$, para todo $k\in K$, es claro que 
            $$\|x+K\|=\inf_{k\in K}\|x-k\|\geq 0,$$
            ya que estamos tomando el ínfimo de un conjunto que esta acotado interiormente por $0$ y $x\in E$ fue tomado arbitrariamente.\\

            Ahora supongamos que $x+K=0+W$, luego $x\in W$, así tenemos que 
            $$0\leq \|x+K\|=\inf_{k\in K}\|x-k\|\leq \|x-x\|=0,$$
            Mostrando que el neutro tiene norma $0.$ Ahora si suponemos que $\|x+K\|=0,$ como la norma es un ínfimo tenemos que existe una sucesión de puntos $k_n\in K$ tal que $\|x-k_n\|\to 0$, esto quiere decir que la sucesión $k_n$ converge a $x$, pero $K$ es cerrado por hipótesis, así $x\in K$, esto quiere decir que $x+K=0+K$. Ahora si $\lambda=0$, es claro que $\|\lambda(x+K)\|=\|0+K\|=0=0\cdot\|x+K\|=|\lambda|\|x+K\|$

        \end{sproof}
    \end{itemize}
    \item[(ii)] Sea $T\in L(E,F)$ tal que existe $c>0$ para el cual
    $$\|Tx\|_F\geq c\|x\|_E,$$
    para todo $x\in E$. Si $K$ denota el espacio nulo de $T$ y $R(T)$ el rango de $T$, muestre que $\overline{T}:E/K\to R(T)$ dada por $\overline{T}(x+K)=T(x),$ $x\in E,$ esta bien definida y es un isomorfismo. Esto es $\overline{T}\in L(E/K,R(T))$ y $\overline{T}^{-1}\in L(R(T),E/K).$
    \begin{sproof}
        Como el dominio de $\overline{T}$ son clases de equivalencia tenemos que mostrar que la aplicación esta bien definida. Consideremos $x+K=y+K$, es decir $x,y\in E$ son dos representantes distintos de la misma clase. Por definición $\overline{T}(x+K)=T(x)$ y $\overline{T}(y+K)=T(y)$, luego como $T$ es lineal
        \begin{align*}
            \overline{T}(x+K)-\overline{T}(y+K)&=T(x)-T(y)\\
            &=T(x-y).
        \end{align*}
        Como supusimos que clases son iguales, eso quiere decir que $x-y\in K$, pero $K$ es el espacio nulo de $T$, así $T(x-y)=0$, mostrando así que
        $$\overline{T}(x+K)=\overline{T}(y+K),$$
        por lo tanto esta bien definida. Ahora la aplicación claramente es sobreyectiva ya que dado $y\in R(T)$, existe $x\in E$ tal que $y=T(x)=\overline{T}(x+K)$, así cada $y$ tiene preimagen. Para la inyectividad se sigue un argumento muy parecido a mostrar que esta bien definida. Dados $x+K,y+K$, si $\overline{T}(x+K)=\overline{T}(y+K)$, esto quiere decir que $T(x)=T(y),$ por linealidad $T(x-y)=0$, así $x-y\in K$, concluyendo que $x+K=y+K.$ Para ver que es isomorfismo faltaría mostrar que la aplicación y su inversa son lineales y acotadas. Primero $\overline{T}$ es lineal ya que
        \begin{align*}
              \overline{T}((x+K)+\lambda(y+K))&=\overline{T}((x+\lambda y)+K)\\
              &=T(x+\lambda y)\\
              &=T(x)+\lambda T(y)\\
              &=\overline{T}(x+K)+\lambda \overline{T}(y+K).
          \end{align*}  
          Note que se tiene por la linealidad de $T.$ Ademas es acotada ya que
          \begin{align*}
              \|\overline{T}(x+K)\|&=\|T(x)\|\\
              &\leq M\|x\|\\
              &=M\|x-k+k\|\\
              \leq M\|x-k\|+M\|k\|.
          \end{align*}
          Donde $M>0$, es una constante que depende de $x$. Esto se tiene ya que $T$ es acotada, ahora por la monoticidad del ínfimo, podemos tomarlo sobre los $k\in K,$ como el lado izquierdo no depende de $k$ tenemos que
          \begin{align*}
              \|\overline{T}(x+K)\|&\leq\inf_{k\in K}\{M\|x-k\|+M\|k\|\}\\
              &= M\inf_{k\in K}\|x-k\|+M\inf_{k\in K}\|k\|\\
              &=M\inf_{k\in K}\|x-k\|\\
              &=M\|x+K\|.
          \end{align*}
          Así hemos mostrado que es acotado. Ahora probemos las mismas dos propiedades para la aplicación $\overline{T}^{-1}.$ Para la linealidad tenemos que dados $y_1,y_2\in R(T)$, existen $x_1,x_2\in E$, tales que $y_i=T(x_i)=\overline{T}(x_i+K)$, para $i=1,2.$ Luego por la linealidad ya probada tenemos que $$y_1+\lambda y_2=\overline{T}((x_1+K)+\lambda(x_2+K)),$$ 
          como es biyectiva, aplicando $\overline{T}^{-1}$ a ambos lados obtenemos $$\overline{T}^{-1}(y_1+\lambda y_2)=(x_1+K)+\lambda(x_2+K).$$
          Pero por la manera en que tomamos $y_1,y_2$, y por la biyectividad sabemos que $\overline{T}^{-1}(y_i)=x_i+K,$ para $i=1,2.$ Así concluimos que 
          $$\overline{T}^{-1}(y_1+\lambda y_2)=\overline{T}^{-1}(y_1)+\lambda\overline{T}^{-1}(y_2).$$
          Para mostrar que es acotada usaremos una idea similar, si $y\in R(T)$, existe un $x\in E$ tal que $y=T(x)=\overline{T}(x+K)$, luego $\overline{T}^{-1}(y)=x+K,$ asi tomando la norma de $E/K$ obtenemos
          \begin{align*}
              \|\overline{T}^{-1}(y)\|&=\|x+K\|\\
              &=\inf_{k\in K}\|x-k\|\\
              &\leq \|x\|.
          \end{align*}
          Por hipotesis existe un $c>0$, tal que $c\|x\|\leq\|T(x)\|$, es decir $\|x\|\leq \dfrac{1}{c}\|T(x)\|$ así tenemos que
          $$\|\overline{T}^{-1}(y)\|\leq \dfrac{1}{c}\|T(x)\|.$$
          Pero $y=T(x),$ luego 
          $$\|\overline{T}^{-1}(y)\|\leq \dfrac{1}{c}\|y\|,$$
          así hemos mostrado que el operador es acotado y por tanto hemos concluido que $\overline{T}$ es un isomorfismo.

    \end{sproof}
    
\end{itemize}