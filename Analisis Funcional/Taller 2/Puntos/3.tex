%!TEX root = ../main.tex
Sean $(E,\|\cdot\|)$ y $(F,\|\cdot\|)$ espacios de Banach.
\begin{itemize}
    \item[(i)] Sea $K\subset E$ un subespacio cerrado de $E.$ Definimos la relacion sobre $E$ dada por $x\thicksim_Ky$ si y solo si $x-y\in K.$
    \begin{itemize}
        \item[(a)] Muestre que $\thicksim_K$ es una relacion de equicalencia sobre $E.$\\
        \item[(b)] Muestre que el espacio cociente $E/K$ es un espacio de Banach con la norma
        $$\|x+K\|_{E/K}=\inf\|x-k\|,\quad x\in E.$$ 
        Es decir, debe verificar que el espacio cociente es un espacio vectorial, normado, cuya norma lo hace completo.
    \end{itemize}
    \item[(ii)] Sea $T\in L(E,F)$ tal que existe $c>0$ para el cual
    $$\|Tx\|_F\geq c\|x\|_E,$$
    para todo $x\in E$. Si $K$ denota el espacio nulo de $T$ y $R(T)$ el rango de $T$, muestre que $\overline{T}:E/K\to R(T)$ dada por $\overline{T}(x+K)=T(x),$ $x\in E,$ esta bien definida y es un isomorfismo. Esto es $\overline{T}\in L(E/K,R(T))$ y $\overline{T}^{-1}\in L(R(T),E/K).$
\end{itemize}