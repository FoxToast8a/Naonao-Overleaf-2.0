%!TEX root = ../main.tex
Sean $(E,\|\cdot\|)$ y $(F,\|\cdot\|)$ espacios de Banach.
\begin{itemize}
    \item[(i)] Sea $K\subset E$ un subespacio cerrado de $E.$ Definimos la relación sobre $E$ dada por $x\thicksim_Ky$ si y solo si $x-y\in K.$
    \begin{itemize}
        \item[(a)] Muestre que $\thicksim_K$ es una relación de equivalencia sobre $E.$\\
        \begin{sproof}
            Dado $x\in E,$ como $K$ es subespacio, $x-x=0\in K$, esto implica que $x\thicksim_K x$, mostrando así la reflexividad.\\

            Dados $x,y\in E$, suponga que tenemos que $x\thicksim_K y,$ luego $x-y\in K$, nuevamente como $K$ es subespacio, es cerrado bajo la multiplicación por escalar, así $-(x-y)\in K$, pero $-(x-y)=y-x$, por definición de la relación tenemos que $y\thicksim_K x,$ mostrando que se cumple la simetría.\\

            Por último sean $x,y,z\in E$, con $x\thicksim_K y$ y $y\thicksim_K z,$ por definición $x-y\in K$ y $y-z\in K$, como $K$ es subespacio, es cerrado para la suma, así tenemos que $(x-y)+(y-z)\in K$, pero $(x-y)+(y-z)=x-z$, así tenemos que $x\thicksim_K z,$ luego, la relación es transitiva. Con esto podemos concluir que la relación es de equivalencia. 

        \end{sproof}
        \item[(b)] Muestre que el espacio cociente $E/K$ es un espacio de Banach con la norma
        $$\|x+K\|_{E/K}=\inf_{k\in K}\|x-k\|,\quad x\in E.$$ 
        Es decir, debe verificar que el espacio cociente es un espacio vectorial, normado, cuya norma lo hace completo.
        \begin{sproof}
            Primero notemos que las operaciones definidas sobre el conjunto $E/K$ son las siguientes
            \begin{align*}
                (x+K)+(y+K)&=(x+y)+K,\\
                \lambda(x+K)&=\lambda x+K.
            \end{align*}
            Puesto que $E$ es un espacio vectorial, las propiedades de $E$ se heredan a $E/K$, solo bastaría verificar que estas operaciones están bien definidas. Si $x_1+K=x_2+K$ y $y_1+K=y_2+K$, tenemos que $x_1-x_2\in K$ y $y_1-y_2\in K$, pero como $K$ es subespacio  $(x_1+y_1)-(x_2+y_2)\in K$, así $(x_1+y_1)+K=(x_2+y_2)+ K$, luego la suma está bien definida. De manera similar, si $x_1-x_2\in K$, tenemos que al multiplicar por un escalar también está en $K$, esto es $\lambda x_1-\lambda x_2\in K$, así $\lambda x_1+K=\lambda x_2+K.$\\

            Ahora veamos que la norma definida en el enunciado, efectivamente es norma del espacio $E/K.$ Primero esta norma está bien definida ya que si $x+K=y+K$, eso quiere decir que $x-y\in K$, luego
            \begin{align*}
                 \|x+K\|&=\inf_{k\in K}\|x-k\|\\
                 &=\inf_{k_1\in K}\|x-(k_1+x-y)\|\\
                 &=\inf_{k_1\in K}\|y-k_1\|\\
                 &=\|y+K\|.
             \end{align*}
             Como $\|x-k\|\geq 0$, para todo $k\in K$, es claro que 
            $$\|x+K\|=\inf_{k\in K}\|x-k\|\geq 0,$$
            ya que estamos tomando el ínfimo de un conjunto que está acotado inferiormente por $0$ y $x\in E$ fue tomado arbitrariamente.\\

            Ahora supongamos que $x+K=0+W$, luego $x\in W$, así tenemos que 
            $$0\leq \|x+K\|=\inf_{k\in K}\|x-k\|\leq \|x-x\|=0,$$
            Mostrando que el neutro tiene norma $0.$ Ahora si suponemos que $\|x+K\|=0,$ como la norma es un ínfimo tenemos que existe una sucesión de puntos $k_n\in K$ tal que $\|x-k_n\|\to 0$, es decir que la sucesión $k_n$ converge a $x$, pero $K$ es cerrado por hipótesis, así $x\in K$, por lo cual $x+K=0+K$. 

            Si $\lambda=0$, es claro que 
            $$\|\lambda(x+K)\|=\|0+K\|=0=0\cdot\|x+K\|=|\lambda|\|x+K\|.$$
            Ahora, si $\lambda\neq 0$,
            \begin{align*}
                \|\lambda(x+K)\|&=\|\lambda x+K\|\\
                &=\inf_{k\in K}\|\lambda x-k\|\\
                &=\inf_{k\in K}\|\lambda(x-\lambda^{-1}k)\|\\
                &=|\lambda|\inf_{k\in K}\|x-\lambda^{-1}k\|\\
                &=|\lambda|\inf_{k_1\in K}\|x-k_1\|\quad (k_1=\lambda^{-1}k\in K)\\
                &=|\lambda|\|x+K\|.
            \end{align*}
             Esto lo podemos hacer ya que $K$ es un subespacio. Por último veamos la desigualdad triangular,
            \begin{align*}
                \|(x+K)+(y+K)\|&=\|(x+y)+K\|\\
                &=\inf_{k\in K}\|x+y-k\|\\
                &=\inf_{k_1,k_2\in K}\|x+y-(k_1+k_2)\|\\
                &\leq \inf_{k_1,k_2\in K}\{\|x-k_1\|+\|y-k_2\|\}\\
                &=\inf_{k_1\in K}\|x-k_1\|+\inf_{k_2\in K}\|y-k_2\|\\
                &=\|x+K\|+\|y+K\|.
            \end{align*}
            Note que nuevamente usamos el hecho de que $K$ es subespacio, para escribir a $k=k_1+k_2.$ Así concluimos que $E/K$ es normado. Faltaría ver que el espacio es Banach.\\

            Consideremos $\{x_n+K\}\subset E/K,$ una sucesión de Cauchy, observe que podemos construir una subsucesión tal que 
            $$\|(x_{n_k}-x_{n_{k+1}})+K\|.$$
            Esto lo podemos hacer ya que si $\varepsilon=\dfrac{1}{2},$ como la sucesión es Cauchy
            $$\|(x_n-x_m)+K\|<\dfrac{1}{2},$$
            para $n,m\geq n_1\in \mathbb{Z}^+$. De manera similar para $\varepsilon=\dfrac{1}{4},$ existe $n_2\in \mathbb{Z}^+$ tal que si $n,m\geq n_2$,
            $$\|(x_n-x_m)+K\|<\dfrac{1}{4},$$
            Note que esta cantidad es menor a $\dfrac{1}{2}$, entonces podemos asumir $n_2>n_1$, por un argumento inductivo, escogemos $n_1<n_2<\dots<n_k$ tal que si $n,m\geq n_k$, tenemos que
            $$\|(x_n-x_m)+K\|<\dfrac{1}{2^k}.$$
            En particular para $n_k,n_{k+1}$ 
            $$\|(x_{n_k}-x_{n_{k+1}})+K\|<\dfrac{1}{2^k}.$$
            Ahora, a partir de esta subsucesión $\{x_{n_k}+K\}$ podemos construir una sucesión $\{y_k\}$ tal que cada $y_k\in x_{n_k}+K$. Por caracterización del ínfimo para $k=1$, si tomamos $\delta=\dfrac{1}{2}-\|(x_{n_1}-x_{n_2})+K\|>0$, tenemos que existe $w_2\in K$ tal que
            $$\|x_{n_1}-x_{n_2}-w_2\|<\|(x_{n_1}-x_{n_2})+K\|+\delta=\dfrac{1}{2}.$$


            Así, tomando $y_1=x_{n_1}$ y $y_2=x_{n_2}+w_2$, se cumple que $y_1\in x_{n_1}+K$ y $y_2\in x_{n_2}+K$. Luego, para el caso de $k=2$
            \begin{align*}
               \dfrac{1}{2^2}&>\|(x_{n_2}-x_{n_3})+K\|\\
               &=\inf_{w\in K}\|x_{n_2}-x_{n_3}-w\|\\
               &=\inf_{w\in K}\|x_{n_2}+w_2-w_2-x_{n_3}-w\|\\
               &=\inf_{\overline{w}\in K}\|x_{n_2}+w_2-x_{n_3}-\overline{w}\|\\
               &=\inf_{\overline{w}\in K}\|y_2-x_{n_3}-\overline{w}\|.
            \end{align*}
            Note que esto lo podemos hacer ya que $w_2\in K$ y este es un subespacio. De manera análoga podemos concluir la existencia de un $w_3\in K$ tal que 
            $$\|y_2-x_{n_3}-w_3\|<\dfrac{1}{2^2}.$$
             Tomando $y_3=x_{n_3}+w_3\in x_{n_3}+K$, nos damos cuenta que podemos tomar la sucesión $\{y_k\}$ deseada.\\

             Como la serie geométrica converge absolutamente, sabemos que dado un $\varepsilon>0,$ existe un $N$ para el cual
             $$\sum_{i=N}^\infty\dfrac{1}{2^i}<\varepsilon.$$ 
             Tomando $n,m\geq N$, asumimos sin pérdida de generalidad que $n>m$, note que
             \begin{align*}
                 \|y_n-y_m\|&\leq \sum_{i=m}^{n-1}\|y_{i+1}-y_i\|\\
                &<\sum_{i=m}^{n-1}\dfrac{1}{2^i}\\
                &<\sum_{i=N}^\infty\dfrac{1}{2^i}<\varepsilon.
             \end{align*}
             Por lo tanto $\{y_k\}$ es de Cauchy y como esta es una sucesión de términos en $E$ que es de Banach, sabemos que $y_k\to y$, luego, debido a que la norma es un ínfimo
             $$\|x_{n-k}+K-(y+K)\|=\|(x_{n_k}-y)+K\|\leq \|y_k-y\|\to 0.$$
             Así, concluimos que la subsucesión es convergente y como la secuencia original era de Cauchy podemos decir que $x_n+K\to y+K.$ Mostrando así que $E/K$ es de Banach.

        \end{sproof}
    \end{itemize}
    \item[(ii)] Sea $T\in L(E,F)$ tal que existe $c>0$ para el cual
    $$\|Tx\|_F\geq c\|x\|_E,$$
    para todo $x\in E$. Si $K$ denota el espacio nulo de $T$ y $R(T)$ el rango de $T$, muestre que $\overline{T}:E/K\to R(T)$ dada por $\overline{T}(x+K)=T(x),$ $x\in E,$ esta bien definida y es un isomorfismo. Esto es $\overline{T}\in L(E/K,R(T))$ y $\overline{T}^{-1}\in L(R(T),E/K).$

    \begin{sproof}
        Como el dominio de $\overline{T}$ son clases de equivalencia tenemos que mostrar que la aplicación está bien definida. Consideremos $x+K=y+K$, es decir $x,y\in E$ son dos representantes distintos de la misma clase. Por definición $\overline{T}(x+K)=T(x)$ y $\overline{T}(y+K)=T(y)$, luego como $T$ es lineal
        \begin{align*}
            \overline{T}(x+K)-\overline{T}(y+K)&=T(x)-T(y)\\
            &=T(x-y).
        \end{align*}
        Como las clases a las que pertenecen x y y son iguales,  $x-y\in K$, pero $K$ es el espacio nulo de $T$, así $T(x-y)=0$ y
        $$\overline{T}(x+K)=\overline{T}(y+K),$$
        entonces $\overline{T}$ está bien definida. 


        La aplicación claramente es sobreyectiva ya que dado $y\in R(T)$, existe $x\in E$ tal que $y=T(x)=\overline{T}(x+K)$, así cada $y$ tiene una preimagen. Para la inyectividad se sigue un argumento análogo a mostrar que está bien definida. Dados $x+K,y+K$, si $\overline{T}(x+K)=\overline{T}(y+K)$, esto quiere decir que $T(x)=T(y),$ por linealidad $T(x-y)=0$, así $x-y\in K$, concluyendo que $x+K=y+K.$ Para ver que es isomorfismo faltaría mostrar que la aplicación y su inversa son lineales y acotadas. Primero veamos que $\overline{T}$ es lineal
        \begin{align*}
              \overline{T}((x+K)+\lambda(y+K))&=\overline{T}((x+\lambda y)+K)\\
              &=T(x+\lambda y)\\
              &=T(x)+\lambda T(y)\\
              &=\overline{T}(x+K)+\lambda \overline{T}(y+K).
          \end{align*}  
          Esto se tiene por la linealidad de $T.$ Además es acotada ya que
          \begin{align*}
              \|\overline{T}(x+K)\|&=\|T(x)\|\\
              &\leq M\|x\|\\
              &=M\|x-k+k\|\\
              &\leq M\|x-k\|+M\|k\|.
          \end{align*}
          Donde $M>0$, es una constante que depende de $x$ puesto que $T$ es acotada, ahora por la monoticidad del ínfimo, podemos tomarlo sobre los $k\in K,$ como el lado izquierdo no depende de $k$ tenemos que
          \begin{align*}
              \|\overline{T}(x+K)\|&\leq\inf_{k\in K}\{M\|x-k\|+M\|k\|\}\\
              &= M\inf_{k\in K}\|x-k\|+M\inf_{k\in K}\|k\|\\
              &=M\inf_{k\in K}\|x-k\|\\
              &=M\|x+K\|.
          \end{align*}
          Así, hemos mostrado que es acotado. Ahora probemos las mismas dos propiedades para la aplicación $\overline{T}^{-1}.$ Para la linealidad tenemos que dados $y_1,y_2\in R(T)$, existen $x_1,x_2\in E$, tales que $y_i=T(x_i)=\overline{T}(x_i+K)$, para $i=1,2.$ Luego por la linealidad ya probada tenemos que $$y_1+\lambda y_2=\overline{T}((x_1+K)+\lambda(x_2+K)),$$ 
          como es biyectiva, aplicando $\overline{T}^{-1}$ a ambos lados
           $$\overline{T}^{-1}(y_1+\lambda y_2)=(x_1+K)+\lambda(x_2+K).$$

          Pero, por la manera en que tomamos $y_1,y_2$, y por la biyectividad sabemos que $\overline{T}^{-1}(y_i)=x_i+K,$ para $i=1,2.$ Así,
          $$\overline{T}^{-1}(y_1+\lambda y_2)=\overline{T}^{-1}(y_1)+\lambda\overline{T}^{-1}(y_2).$$
          Para mostrar que es acotada usaremos una idea similar, si $y\in R(T)$, existe un $x\in E$ tal que $y=T(x)=\overline{T}(x+K)$, luego $\overline{T}^{-1}(y)=x+K,$ tomando la norma de $E/K$ obtenemos que,
          \begin{align*}
              \|\overline{T}^{-1}(y)\|&=\|x+K\|\\
              &=\inf_{k\in K}\|x-k\|\\
              &\leq \|x\|.
          \end{align*}
          Por hipótesis existe un $c>0$, tal que $c\|x\|\leq\|T(x)\|$, es decir $\|x\|\leq \dfrac{1}{c}\|T(x)\|$ lo que nos indica que
          $$\|\overline{T}^{-1}(y)\|\leq \dfrac{1}{c}\|T(x)\|.$$
          Pero, $y=T(x),$ luego,
          $$\|\overline{T}^{-1}(y)\|\leq \dfrac{1}{c}\|y\|,$$
          así hemos mostrado que el operador es acotado y por tanto hemos concluido que $\overline{T}$ es un isomorfismo.

    \end{sproof}
    
\end{itemize}