%!TEX root = ../main.tex
 Considere los espacios $C([0,1])$ y $C^1([0,1])$ ambos equipados con la norma del supremo $\|f\|_{L^\infty}=\displaystyle\sup_{x\in[0,1]}|f(x)|.$ Definimos el operador derivada $D:C^1([0,1])\to C([0,1])$ dado por $f\mapsto f^\prime.$ Muestre que $D$ es un operador no acotado, pero su grafico $G(D)$ es cerrado.

\begin{proof}
\hfill

Supongamos, por contradicción, que $D$ es un operador acotado. Entonces existe una constante $M > 0$ tal que,
\begin{align*}
    \|f'\| = \|Df\| \leq M\|f\| \quad \text{para todo } f \in C^{1}([0,1]).
\end{align*}

Definimos una sucesión de funciones $\{f_n\}_{n \in \mathbb{N}}$ dada por,
\begin{align*}
    f_n : [0,1] \to \mathbb{R}, \quad f_n(x) = x^n.
\end{align*}

Claramente $f_n \in C^1([0,1])$ para todo $n \in \mathbb{N}$. Además, se cumple que,
\begin{align*}
    \|f_n\| &= \sup_{x \in [0,1]} |x^n| = 1,\\
    \|Df_n\| &= \sup_{x \in [0,1]} |n x^{n-1}| = n.
\end{align*}

Entonces:
\begin{align*}
    \|Df_n\| = n \leq M \|f_n\| = M.
\end{align*}

Esto implica que $n \leq M$ para todo $n$, lo cual es una contradicción, ya que siempre existe un $n \in \mathbb{N}$ tal que $n > M$. Por lo tanto, el operador $D$ no es acotado.\\

Ahora veamos que, aunque el operador \( D \) no es acotado, su gráfico  
\[
G(D) = \left\{ (f, f') : f \in C^1([0,1]) \text{ y } f' \in C([0,1]) \right\}
\]  
sí es un conjunto cerrado.

Para demostrarlo, tomemos una sucesión \(\{(f_n, f_n')\}_{n \in \mathbb{N}} \subset G(D)\) tal que  
\[
(f_n, f_n') \to (f, g)
\]  
\text{en la norma del gráfico}, es decir, en la norma  
\[
\|(f_n, f_n') - (f, g)\|_{G(D)} = \|f_n - f\|_{L^\infty} + \|f_n' - g\|_{L^\infty}.
\]

Esto significa que, para todo \(\varepsilon > 0\), existe un \(N \in \mathbb{N}\) tal que si \(n > N\), entonces  
\[
\|f_n - f\|_{L^\infty} + \|f_n' - g\|_{L^\infty} < \varepsilon.
\]  
Por lo tanto,  
\[
f_n \to f \quad \text{uniformemente}, \quad \text{y} \quad f_n' \to g \quad \text{uniformemente}.
\]

Ahora, dado que cada \(f_n\) es de clase \(C^1([0,1])\), y que tanto \(f_n\) como \(f_n'\) convergen uniformemente, se sigue que \(f \in C^1([0,1])\) y que \( f' = g \), con \( g \in C([0,1]) \). Es decir, la función límite \( f \) es derivable y su derivada es \( g \), que es continua. Esto implica que \((f, f') \in G(D)\).  

Por lo tanto, el gráfico \(G(D)\) es un conjunto cerrado.


\end{proof}