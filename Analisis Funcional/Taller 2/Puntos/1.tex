%!TEX root = ../main.tex
Sea $(E,\|\cdot\|)$ un espacio vectorial normado. Dado $r>0,$ considere $C=B(0,r)=\{y\in E:\|y\|<r\}.$ Determine el funcional de Minkowski de $C.$


\textbf{Solución:}

Dado que $(E, \| \cdot \|)$ es un espacio vectorial normado, se deduce que el conjunto $C = B(0,r)$ es abierto, convexo y  $0 \in C$.

Por consiguiente, el funcional de Minkowski asociado a $C$ se define como,
\[
\rho(x) = \text{ínf} \left\{ \alpha > 0 : \alpha^{-1}x \in C \right\}, \qquad x \in E.
\]

Ahora, sea $x \in B(0,r)$. Entonces, para todo $\alpha > 0$ tal que $\alpha^{-1}x \in C$, 
\[
\| \alpha^{-1}x \| < r,
\]

lo que implica que,
\[
\alpha^{-1} \| x \| < r,
\]

luego, despejando $\alpha$, 
\[
\frac{\| x \|}{r} < \alpha.
\]


Supongamos por contradicción que $\rho(x) \neq \dfrac{\| x \|}{r}$, por lo que se tiene que,
\[
\frac{\| x \|}{r} < \rho(x).
\]

Tomemos el promedio entre $\dfrac{\| x \|}{r}$ y $\rho(x)$
\[
\beta = \dfrac{\dfrac{\| x \|}{r} + \rho(x)}{2},
\]

este valor cumple que
\[
\frac{\| x \|}{r} < \beta < \rho(x).
\]

Pero entonces,
\[
\| \beta^{-1}x \| = \beta^{-1} \| x \| < \frac{r}{\| x \|} \cdot \| x \| = r,
\]
lo cual implica que $\beta^{-1}x \in B(0,r)=C$. Por tanto, $\beta \in \left\{ \alpha > 0 : \alpha^{-1}x \in C \right\}$.

Esto contradice el hecho de que $\rho(x)$ es la ínfimo de ese conjunto, ya que $\beta < \rho(x)$.

Por lo tanto, concluimos que,
\[
\rho(x) = \frac{\| x \|}{r}.
\]