%!TEX root = ../main.tex

Considere los operadores de desplazamiento $S_r, S_l \in L\left(l^2\right)$, donde si $x=\left(x_1, x_2, \ldots, x_n, \ldots\right) \in l^2$, estos se definen como
$$
S_r x=\left(0, x_1, x_2, \ldots, x_{n-1}, \ldots\right)
$$
$y$
$$
S_l x=\left(x_2, x_3, x_4, \ldots, x_{n+1}, \ldots\right) .
$$
$S_r$ se conoce como desplazamiento a derecha y $S_l$ como desplazamiento a izquierda.
\begin{enumerate}
    \item[(a)] Determinar las normas \( \|S_{r}\| \) y \( \|S_{l}\| \).
    \begin{proof}
        Sea \( x = (x_1, x_2, \ldots) \in \ell^2 \). Por definición de la norma de operador, se tiene
\[
\|S_r\|_{\mathcal{L}(\ell^2)} = \sup_{\|x\|_{\ell^2} \leq 1} \|S_r x\|_{\ell^2}.
\]
Dado que \( S_r x = (0, x_1, x_2, \ldots) \), si denotamos \( y = S_r x = (y_1, y_2, \ldots) \), entonces
\[
\|S_r x\|_{\ell^2} = \left( \sum_{k=1}^{\infty} |y_k|^2 \right)^{1/2} = \left( \sum_{k=2}^{\infty} |x_{k-1}|^2 \right)^{1/2} = \left( \sum_{k=1}^{\infty} |x_k|^2 \right)^{1/2} = \|x\|_{\ell^2}.
\]
Por lo tanto,
\[
\|S_r\|_{\mathcal{L}(\ell^2)} = \sup_{\|x\|_{\ell^2} \leq 1} \|S_r x\|_{\ell^2} \leq \sup_{\|x\|_{\ell^2} \leq 1} \|x\|_{\ell^2} = 1.
\]
Como esta cota se alcanza, por ejemplo al tomar \( x = e_1 = (1, 0, 0, \ldots) \), para el cual \( \|x\|_{\ell^2} = 1 \) y
\[
S_r e_1 = (0, 1, 0, 0, \ldots), \quad \|S_r e_1\|_{\ell^2} = 1,
\]
concluimos que
\[
\|S_r\|_{\mathcal{L}(\ell^2)} = 1.
\]

De forma análoga, el operador \( S_l \) está definido por \( S_l x = (x_2, x_3, \ldots) \). Si denotamos \( y = S_l x \), entonces
\[
\|S_l x\|_{\ell^2} = \left( \sum_{k=1}^{\infty} |y_k|^2 \right)^{1/2} = \left( \sum_{k=2}^{\infty} |x_k|^2 \right)^{1/2} \leq \|x\|_{\ell^2},
\]
por lo que
\[
\|S_l\|_{\mathcal{L}(\ell^2)} \leq 1.
\]
Esta cota también se alcanza, por ejemplo, con \( x = e_2 = (0, 1, 0, \ldots) \), ya que
\[
S_l e_2 = (1, 0, 0, \ldots), \quad \|S_l e_2\|_{\ell^2} = 1,
\]
lo que implica que
\[
\|S_l\|_{\mathcal{L}(\ell^2)} = 1.
\]

    \end{proof}
    
    \item[(b)] Muestre que \( EV(S_{r}) = \emptyset \).
    \begin{proof}
        Sea \( x = (x_1, x_2, \ldots) \in \ell^2 \) y sea \( \lambda \in \mathbb{R} \). Supongamos que \( S_r x = \lambda x \). Entonces, por la definición del operador \( S_r \), se tiene:
\[
S_r x = (0, x_1, x_2, \ldots) = (\lambda x_1, \lambda x_2, \ldots).
\]
Comparando componente a componente, se obtiene:
\begin{itemize}
    \item  En la primera coordenada: \( 0 = \lambda x_1 \), por lo que si \( \lambda \neq 0 \), se deduce que \( x_1 = 0 \).
\item  En la segunda coordenada: \( x_1 = \lambda x_2 \), lo cual implica \( x_2 = 0 \) ya que \( x_1 = 0 \).
\item  En general, para \( i \geq 2 \), se cumple que \( x_{i-1} = \lambda x_i \), lo que por recurrencia implica que \( x_i = 0 \) para todo \( i \geq 1 \).
\end{itemize}



Por tanto, \( x = (0, 0, 0, \ldots) \).

Ahora, si \( \lambda = 0 \), se tiene que
\[
S_r x = (0, x_1, x_2, \ldots) = (0, 0, 0, \ldots),
\]
lo cual implica que \( x_i = 0 \) para todo \( i \geq 1 \), es decir, nuevamente \( x = (0, 0, 0, \ldots) \).

En ambos casos, el único vector que satisface \( S_r x = \lambda x \) es el vector nulo. Por lo tanto, el núcleo de \( S_r - \lambda I \) es trivial:
\[
N(S_r - \lambda I) = \{0\},
\]
y concluimos que \( \lambda \) no es valor propio de \( S_r \) para ningún \( \lambda \in \mathbb{R} \). Así,
\[
\sigma_p(S_r) = \emptyset.
\]

    \end{proof}
    
    
     \item[(e)] Muestre que \( \sigma(S_{l}) = [-1, 1] \).
    \begin{proof}
        Por el ítem $a)$ sabemos que $\|S_l\|_{\ell^2} = 1$. Entonces, por un teorema visto en clase, se tiene que
\[
\sigma(S_l) \subseteq \{ \lambda \in \mathbb{C} : |\lambda| \leq \|S_l\| = 1 \}.
\]

Supongamos que $\lambda \neq 0$ y que existe $x \in \ell^2$ tal que \( S_l x = \lambda x \). Esto implica que
\[
(x_2, x_3, x_4, \dots) = (\lambda x_1, \lambda x_2, \lambda x_3, \dots).
\]
Por lo tanto, para todo \( i \geq 1 \), se cumple que
\[
x_{i+1} = \lambda x_i.
\]
Este tipo de recurrencia implica que
\[
x_i = \lambda^{i-1} x_1 \quad \text{para todo } i \geq 1.
\]
Es decir, el vector \( x \) tiene la forma
\[
x = (x_1, \lambda x_1, \lambda^2 x_1, \lambda^3 x_1, \dots) = x_1 (1, \lambda, \lambda^2, \lambda^3, \dots).
\]

Ahora bien, para que \( x \in \ell^2 \), debe cumplirse que
\begin{align*}
\sum_{n=1}^{\infty} |x_n|^2 
&= \sum_{n=1}^{\infty} |x_1 \lambda^{n-1}|^2 
= |x_1|^2 \sum_{n=0}^{\infty} |\lambda|^{2n}.
\end{align*}

La serie geométrica \( \sum_{n=0}^{\infty} |\lambda|^{2n} \) converge si y solo si \( |\lambda| < 1 \). Por lo tanto, para que \( x \in \ell^2 \), se debe tener \( |\lambda| < 1 \).

Esto muestra que si \( |\lambda| < 1 \), entonces \( \lambda \) es valor propio de \( S_l \), y así
\[
A := \{ \lambda \in \mathbb{R} : |\lambda| < 1 \} \subseteq \sigma(S_l).
\]

Como \( \sigma(S_l) \) es un conjunto cerrado en \( \mathbb{R} \) , se sigue que
\[
\overline{A} \subseteq \sigma(S_l) \subseteq \{ \lambda \in \mathbb{R} : |\lambda| \leq 1 \}.
\]

Pero \( \overline{A} = \{ \lambda \in \mathbb{R} : |\lambda| \leq 1 \} \), así que se concluye
\[
\sigma(S_l) = \{ \lambda \in \mathbb{R} : |\lambda| \leq 1 \}.
\]

    \end{proof}
    \item[(d)] Muestre que \( EV(S_{l}) = (-1, 1) \). Encuentre el espacio propio correspondiente.
   
\begin{proof}
Ya se ha visto que si \( |\lambda| < 1 \), entonces existe un vector no nulo \( x \in \ell^2 \) tal que
\[
x = x_1(1, \lambda, \lambda^2, \lambda^3, \dots),
\]
y este pertenece a \( \ell^2 \) si y solo si \( \sum_{n=0}^\infty |\lambda|^{2n} < \infty \), lo cual ocurre si y solo si \( |\lambda| < 1 \). Por lo tanto, todos los \( \lambda \in (-1,1) \) son valores propios de \( S_l \), y el espacio propio correspondiente está dado por
\[
\text{span}\{(1, \lambda, \lambda^2, \lambda^3, \ldots)\}.
\]

Veamos ahora que \( \lambda = 1 \) y \( \lambda = -1 \) no son valores propios de \( S_l \). Supongamos que \( S_l x = \lambda x \) con \( \lambda = \pm 1 \), y \( x \in \ell^2 \setminus \{0\} \). Entonces, como antes,
\[
x = x_1 (1, \lambda, \lambda^2, \lambda^3, \ldots).
\]

Para \( \lambda = 1 \), tendríamos \( x = x_1 (1,1,1,1,\dots) \), y la norma cuadrada sería
\[
\|x\|^2 = |x_1|^2 \sum_{n=0}^{\infty} 1 = \infty,
\]
lo cual contradice que \( x \in \ell^2 \).

Análogamente, para \( \lambda = -1 \), tendríamos \( x = x_1(1, -1, 1, -1, \dots) \), y nuevamente:
\[
\|x\|^2 = |x_1|^2 \sum_{n=0}^{\infty} 1 = \infty,
\]
por lo que \( x \notin \ell^2 \) tampoco en este caso.

Así, ni \( \lambda = 1 \) ni \( \lambda = -1 \) son valores propios de \( S_l \).

En conclusión,
\[
EV(S_l) = (-1,1).
\]
\end{proof}

   
    \item[(f)] Determine los adjuntos \( S_{r}^* \) y \( S_{l}^* \).
    
    \textbf{Solución:}
    Recordemos que, en un espacio de Hilbert real como \( \ell^2(\mathbb{R}) \), el adjunto \( T^* \) de un operador \( T \in \mathcal{L}(\ell^2) \) es el único operador tal que
\[
(Tx, y) = (x, T^*y) \quad \text{para todo } x, y \in \ell^2.
\]

Sea \( x = (x_1, x_2, x_3, \ldots) \in \ell^2 \) y \( y = (y_1, y_2, y_3, \ldots) \in \ell^2 \). 

\textbf{Para \( S_r \):} recordemos que
\[
S_r x = (0, x_1, x_2, x_3, \ldots).
\]
Entonces
\begin{align*}
(S_r x, y) &= \sum_{n=1}^\infty (S_r x)_n y_n = \sum_{n=1}^\infty x_{n-1} y_n \quad \text{(tomando \( x_0 = 0 \))} \\
&= \sum_{n=2}^\infty x_{n-1} y_n = \sum_{m=1}^\infty x_m y_{m+1} = (x, z),
\end{align*}
donde \( z = (y_2, y_3, y_4, \ldots) \). Por tanto,
\[
S_r^* y = (y_2, y_3, y_4, \ldots),
\]
es decir,
\[
S_r^* y = S_l y.
\]

\textbf{Para \( S_l \):} recordemos que
\[
S_l x = (x_2, x_3, x_4, \ldots).
\]
Entonces
\begin{align*}
(S_l x, y) &= \sum_{n=1}^\infty (S_l x)_n y_n = \sum_{n=1}^\infty x_{n+1} y_n \\
&= \sum_{m=2}^\infty x_m y_{m-1} = (x, z),
\end{align*}
donde \( z = (0, y_1, y_2, y_3, \ldots) \). Por tanto,
\[
S_l^* y = (0, y_1, y_2, \ldots),
\]
es decir,
\[
S_l^* y = S_r y.
\]

Por lo tanto, se cumple que
\[
S_r^* = S_l, \quad \text{y} \quad S_l^* = S_r.
\]
\item[(c)] Muestre que \( \sigma(S_{r}) = [-1, 1] \).
    \begin{proof}
        Primero, notamos que $S_r$ es un operador lineal y acotado sobre el espacio de Hilbert real $\ell^2$, lo cual ya fue demostrado previamente. Además, en una prueba anterior también se encontró su adjunto, y se verificó que
\[
S_r^* = S_\ell.
\]

Ahora, por la \textbf{Proposición 8.3.1}, la cual establece que si $T \in \mathcal{L}(H)$ para un espacio de Hilbert $H$, entonces se tiene que
\[
\sigma(T) = \sigma(T^*),
\]
concluimos que
\[
\sigma(S_r) = \sigma(S_r^*) = \sigma(S_\ell)=[-1,1].
\]

Esto demuestra que el espectro de $S_r=[-1,1]$.
    \end{proof}
    \begin{proof}
        Queremos mostrar que si \( \lambda \in \rho(S_r) \), entonces \( |\lambda| > 1 \).

Recordemos que \( S_r \in \mathcal{L}(\ell^2) \) y que si \( \lambda \in \rho(S_r) \), entonces \( R_\lambda = (\lambda I - S_r)^{-1} \in \mathcal{L}(\ell^2) \), es decir, \( R_\lambda \) es un operador acotado.

Sea \( x = (x_1, x_2, \dots) \in \ell^2 \), y supongamos que \( y = R_\lambda x \). Entonces
\[
(\lambda I - S_r)y = x \quad \text{es decir,} \quad \lambda y_n - y_{n-1} = x_n, \quad \text{con } y_0 = 0.
\]
Despejando recursivamente obtenemos:
\[
y_1 = \frac{x_1}{\lambda}, \quad y_2 = \frac{x_2 + y_1}{\lambda} = \frac{x_2}{\lambda} + \frac{x_1}{\lambda^2}, \quad \dots, \quad y_n = \sum_{k=1}^n \frac{x_k}{\lambda^{n - k + 1}}.
\]
Entonces, 
\[
y_n = \sum_{k=1}^n \lambda^{-(n - k + 1)} x_k.
\]

Supongamos ahora que \( |\lambda| \leq 1 \).Consideremos 
\[
x^{(N)} = (\underbrace{1, 1, \dots, 1}_N, 0, 0, \dots),
\]
que claramente están en \( \ell^2 \). Evaluamos \( y^{(N)} = R_\lambda x^{(N)} \), y calculamos su norma:

\[
y_n^{(N)} = \begin{cases}
\sum_{k=1}^n \lambda^{-(n - k + 1)} = \lambda^{-n} \sum_{j=1}^{n} \lambda^j = \lambda^{-n} \cdot \frac{\lambda(1 - \lambda^n)}{1 - \lambda}, & \text{si } n \leq N, \\
\sum_{k=1}^N \lambda^{-(n - k + 1)} = \lambda^{-n} \sum_{j=n - N + 1}^{n} \lambda^j, & \text{si } n > N.
\end{cases}
\]

En particular, para \( n = N \), se tiene:
\[
y_N^{(N)} = \sum_{k=1}^{N} \lambda^{-(N - k + 1)} = \sum_{j=1}^{N} \lambda^{-j}.
\]

Si \( |\lambda| \leq 1 \), entonces \( |\lambda^{-1}| \geq 1 \), y por tanto \( \sum_{j=1}^{N} |\lambda^{-j}| \geq N \). Esto implica que \( \| y^{(N)} \| \to \infty \) cuando \( N \to \infty \), mientras que \( \| x^{(N)} \| = \sqrt{N} \), así que

\[
\frac{\| y^{(N)} \|}{\| x^{(N)} \|} \to \infty.
\]

Esto contradice que \( R_\lambda \) sea un operador acotado. Por tanto, si \( \lambda \in \rho(S_r) \), entonces necesariamente \( |\lambda| > 1 \). Como el espectro debe estar contenido en \( \{ \lambda \in \mathbb{C} : |\lambda| \leq \|S_r\|=1 \} \), y como ya probamos en un paso anterior que todo \( |\lambda| < 1 \) está en el resolvente, concluimos que:

\[
\sigma(S_r) = \{ \lambda \in \mathbb{C} : |\lambda| \leq 1 \}.
\]

    \end{proof}
\end{enumerate}




