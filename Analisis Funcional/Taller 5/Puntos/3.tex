%!TEX root = ../main.tex
 Sea $H$ un espacio de Hilbert $y M \subseteq H$ un subespacio cerrado. Considera la proyección ortogonal $P_M$. Muestre que
 \begin{itemize}
     \item[(I)] $P_M$ es lineal.
\item[(II)] $P_M^2=P_M$ (esto es, aplicar dos veces el operador proyección da el mismo resultado).
\item[(III)] $P_M^{\star}=P_M$, donde $P_M^{\star}$ denota el adjunto de $P_M$ (vea el Ejercicio 14).
\item[(IV)] $\operatorname{Rango}\left(P_M\right)=M$ y $\operatorname{Kernel}\left(P_M\right)=M^{\perp}$.
\item[(V)] Suponga que $P \in L(H)$. Entonces $P$ es una proyección ortogonal sobre un subespacio cerrado de $H$ si, y solo si, $P=P^2=P^{\star}$. 
 \end{itemize}
