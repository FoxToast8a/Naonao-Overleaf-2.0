%!TEX root = ../main.tex
 (I) Muestre que los siguientes conjuntos $M$ son subespacios cerrados no vacíos de $L^2((-1,1))$ y determine explícitamente la proyección $P_M$ en cada caso.
 \begin{itemize}
     

\item[(a)] $M=\left\{f \in L^2((-1,1))\right.$ : $f(x)=f(-x)$ para casi todo $\left.x \in(-1,1)\right\}$.
\item[(b)] $M=\left\{f \in L^2((-1,1)): \int_{-1}^1 f(x) d x=0\right\}$.
\item[(c)] $M=\left\{f \in L^2((-1,1)): f(x)=0\right.$ para casi todo $\left.x \in(-1,0)\right\}$.
 \end{itemize}
(II) Sea $\Omega \subset \mathbb{R}^n$ un abierto acotado. Considere
$$
K=\left\{f \in L^2(\Omega): \int_{\Omega} f(x) d x \geq 1\right\}
$$
\begin{itemize}
\item[(a)] Muestre que $K$ es un conjunto cerrado convexo de $L^2(\Omega)$.
\item[(b)] Determine la proyección sobre $K$, es decir, el operador $P_K$.
\end{itemize}


