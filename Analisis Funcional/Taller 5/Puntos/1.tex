%!TEX root = ../main.tex
 (I) Muestre que los siguientes conjuntos $M$ son subespacios cerrados no vacíos de $L^2((-1,1))$ y determine explícitamente la proyección $P_M$ en cada caso.
 \begin{itemize}

     

\item[(a)] $M=\left\{f \in L^2((-1,1))\right.$ : $f(x)=f(-x)$ para casi todo $\left.x \in(-1,1)\right\}$.
\begin{proof}
\hfill
\begin{enumerate}
    \item[I)] \textbf{M es no vacio:} 
    Veamos que $f(x)=0\in M$, pues 
    \begin{align*}
        \|f(x)\|_{L_2}=\left( \int_{-1}^{1}|0|^2 dx\right)^{\frac{1}{2}}=0<\infty\quad \text{ y }\quad f(-x)=0=f(x).
    \end{align*}
    \item[II)]\textbf{M es cerrado:} Sea \( (f_n)_{n=1}^{\infty} \subseteq M \) tal que \( f_n \to f \) en \( L^2(-1,1) \). Entonces, para todo \( \varepsilon > 0 \) existe \( N \in \mathbb{Z}^+ \) tal que si \( n \geq N \), se cumple

\begin{align}
    \|f_n - f\|_{L^2(-1,1)} < \frac{\varepsilon}{2}. \label{eq:convergencia}
\end{align}

A continuación, definamos \( g(x) := f(-x) \) para \( x \in (-1,1) \), y veamos que

\begin{align}
    \|g - f\|_{L^2(-1,1)} &= \|g - f_n + f_n - f\|_{L^2(-1,1)} \nonumber \\
    &\leq \underbrace{\|g - f_n\|_{L^2(-1,1)}}_{\text{A}} + \|f_n - f\|_{L^2(-1,1)}. \label{eq:desigualdad}
\end{align}

Para estimar el término \( A \), notamos que como \( f_n(x) = f_n(-x) \) para casi todo \( x \in (-1,1) \), se tiene

\begin{align*}
    A^2 = \int_{-1}^1 |g(x) - f_n(x)|^2\,dx &= \int_{-1}^1 |f(-x) - f_n(x)|^2\,dx \\
    &= \int_{-1}^1 |f(-x) - f_n(-x)|^2\,dx.
\end{align*}

Haciendo el cambio de variable \( u = -x \), obtenemos

\begin{align}
    A^2 = -\int_{1}^{-1} |f(u) - f_n(u)|^2\,du &= \int_{-1}^{1} |f(u) - f_n(u)|^2\,du \nonumber \\
    &= \|f - f_n\|_{L^2(-1,1)}^2. \label{eq:romano}
\end{align}

Entonces, para \( n \geq N \), usando las ecuaciones \eqref{eq:convergencia}, \eqref{eq:romano} y \eqref{eq:desigualdad}, obtenemos:

\begin{align*}
    \|g - f\|_{L^2(-1,1)} &\leq \|g - f_n\|_{L^2(-1,1)} + \|f_n - f\|_{L^2(-1,1)} \\
    &= \|f - f_n\|_{L^2(-1,1)} + \|f_n - f\|_{L^2(-1,1)} \\
    &< \frac{\varepsilon}{2} + \frac{\varepsilon}{2} = \varepsilon.
\end{align*}

Como \( \varepsilon > 0 \) es arbitrario, concluimos que \( \|g - f\|_{L^2(-1,1)} = 0 \), es decir, \( f(x) = f(-x) \) para casi todo \( x \in (-1,1) \), lo cual implica que \( f \in M \). Por tanto, \( M \) es cerrado.

    \item[III)] \textbf{ M es un subespacio:} Sea \( f, g \in M \), veamos que \(  f + g \in M \).

Por la definición del conjunto \( M \), se cumple que \( f(x) = f(-x) \) y \( g(x) = g(-x) \) para casi todo \( x \in (-1,1) \). Esto significa que existen conjuntos de medida nula \( F \) y \( G \) tales que

\[
f(x) = f(-x) \quad \forall x \in (-1,1) \setminus F,
\]
\[
g(x) = g(-x) \quad \forall x \in (-1,1) \setminus G.
\]

Además, por la subaditividad de la medida de Lebesgue, se tiene

\[
0 \leq \mu(F \cup G) \leq \mu(F) + \mu(G) = 0,
\]

por lo tanto, \( F \cup G \) también es un conjunto de medida nula. Como

\[
(-1,1) \setminus (F \cup G) \subseteq (-1,1) \setminus F, \quad (-1,1) \setminus G,
\]

se concluye que para casi todo \( x \in (-1,1) \setminus (F \cup G) \) se cumple

\[
(f + g)(x) = f(x) + g(x) = f(-x) + g(-x) = (f + g)(-x).
\]

Así, \( f + g \in M \).

Ahora, para \( f \), sabemos que \( f(x) = f(-x) \) para casi todo \( x \in (-1,1) \setminus F \). Sea \( \lambda \in \mathbb{R} \), entonces
\[
(\lambda f)(x) = \lambda f(x) = \lambda f(-x) = (\lambda f)(-x) \quad,
\]
para casi todo \( x \in (-1,1) \setminus F \), lo cual implica que \( \lambda f \in M \). Por lo tanto, \( M \) es un subespacio de \( H \).
\item[IV)] \textbf{Proyección $P_M$:} Como \( M \) es un subespacio cerrado de \( L^2((-1,1)) \), para todo \( f \in L^2((-1,1)) \) la proyección ortogonal \( P_M f \) está bien definida como la única \( g \in M \) tal que
\[
(f - g, h) = 0 \quad \text{para toda } h \in M.
\]

Definimos \( g(x) := \dfrac{f(x) + f(-x)}{2} \). Notamos que \( g \in L^2((-1,1)) \), ya que \( L^2((-1,1)) \) es un espacio vectorial, y si \( f \in L^2 \), entonces también lo están \( f(-x) \) y su suma.

Además,

\[
g(-x) = \frac{f(-x) + f(-(-x))}{2} = \frac{f(-x) + f(x)}{2} = g(x),
\]

por lo tanto, \( g \in M \).

Ahora calculemos \( (f - g, h) \) para \( h \in M \):

\begin{align*}
(f - g, h) &= \int_{-1}^{1} \left[ f(x) - \frac{f(x) + f(-x)}{2} \right] h(x)\,dx \\
&= \int_{-1}^{1} \left[ \frac{f(x)}{2} - \frac{f(-x)}{2} \right] h(x)\,dx \\
&= \frac{1}{2} \int_{-1}^{1} f(x) h(x)\,dx - \frac{1}{2} \int_{-1}^{1} f(-x) h(x)\,dx \\
&= B - A,
\end{align*}

donde:
\begin{itemize}
    \item \( B = \dfrac{1}{2} \int_{-1}^{1} f(x) h(x)\,dx \),
    \item \( A = \dfrac{1}{2} \int_{-1}^{1} f(-x) h(x)\,dx \).
    
\end{itemize}


Como \( h \in M \), entonces \( h(x) = h(-x) \) para casi todo \( x \in (-1,1) \). Usando el cambio de variable \( u = -x \), tenemos:

\begin{align*}
A &= \frac{1}{2} \int_{-1}^{1} f(-x) h(x)\,dx 
= \frac{1}{2} \int_{-1}^{1} f(-x) h(-x)\,dx \\
&= -\frac{1}{2} \int_{1}^{-1} f(u) h(u)\,du 
= \frac{1}{2} \int_{-1}^{1} f(u) h(u)\,du = B.
\end{align*}

Luego, como \( A = B \), se concluye que:

\[
(f - g, h) = B - A = 0.
\]

Por lo tanto, \( g = P_M f \) es la proyección ortogonal de \( f \) sobre \( M \), y está dada por:

\[
(P_M f)(x) = \frac{f(x) + f(-x)}{2}.
\]

\end{enumerate}

\end{proof}
\item[(b)] $M=\left\{f \in L^2((-1,1)): \int_{-1}^1 f(x) d x=0\right\}$.

 \begin{proof}
    \hfill
        \begin{enumerate}
            \item[I)]\textbf{M es no vacío:} Consideremos la función \( f(x) = x \). Notamos que \( f \in L^2((-1,1)) \) porque

\[
\int_{-1}^1 |x|^2\, dx = \int_{-1}^1 x^2\, dx = \frac{2}{3} < \infty.
\]

Además,

\[
\int_{-1}^1 f(x)\, dx = \int_{-1}^1 x\, dx = 0.
\]

Por lo tanto, \( f(x) = x \in M \), y así concluimos que \( M \neq \emptyset \).
\item[II)] \textbf{\( M \) es cerrado:} En primera instancia, notamos que podemos expresar \( M \) usando el producto interno de \( L^2((-1,1)) \):

\[
M = \left\{ f \in L^2((-1,1)) : \int_{-1}^1 f(x)\, dx = 0 \right\} = \left\{ f \in L^2((-1,1)) : (f, 1) = 0 \right\},
\]

donde \( 1 \) es la función constante \( g(x) = 1 \).

Es importante mencionar que, para \( f \in L^2((-1,1)) \), tiene sentido calcular la integral

\[
\int_{-1}^1 f(x)\, dx,
\]

porque \( \mu((-1,1)) = 2 < \infty \). Además, usando la desigualdad de Cauchy–Schwarz, se tiene:

\[
\left| \int_{-1}^1 f(x)\, dx \right| \leq \left( \int_{-1}^1 1^2\, dx \right)^{1/2} \left( \int_{-1}^1 |f(x)|^2\, dx \right)^{1/2} = \sqrt{2} \, \|f\|_{L^2((-1,1))},
\]

por lo que la integral es finita.

Notemos ahora que \( L^2((-1,1)) \setminus M \neq \varnothing \), pues, por ejemplo, la función constante \( g(x) = 1 \) pertenece a \( L^2((-1,1)) \), pero no a \( M \), ya que

\[
(g, 1) = \int_{-1}^1 1 \cdot 1\, dx = 2 \neq 0.
\]

Así, \( g \notin M \). Sea ahora \( f \in L^2((-1,1)) \setminus M \), entonces \( (f,1) \neq 0 \). Definimos:

\[
\alpha := \left| \left( f, \frac{1}{\sqrt{2}} \right) \right| = \frac{1}{\sqrt{2}} |(f,1)| > 0.
\]

Sea \( B := B(f, \frac{\alpha}{2}) \) la bola abierta en \( L^2((-1,1)) \) centrada en \( f \) y de radio \( \frac{\alpha}{2} \). Si \( g \in B \), entonces

\[
\|g - f\|_{L^2} < \frac{\alpha}{2}.
\]

Aplicando nuevamente la desigualdad de Cauchy–Schwarz, se obtiene:

\[
\left| \left( g - f, \frac{1}{\sqrt{2}} \right) \right| \leq \|g - f\|_{L^2} \cdot \left\| \frac{1}{\sqrt{2}} \right\|_{L^2} = \|g - f\|_{L^2} < \frac{\alpha}{2}.
\]


Entonces,

\begin{align*}
\left| \left( g, \frac{1}{\sqrt{2}} \right) \right| &= \left| \left( f, \frac{1}{\sqrt{2}} \right) + \left( g - f, \frac{1}{\sqrt{2}} \right) \right| \\
&\geq \left| \left( f, \frac{1}{\sqrt{2}} \right) \right| - \left| \left( g - f, \frac{1}{\sqrt{2}} \right) \right| \\
&> \alpha - \frac{\alpha}{2} = \frac{\alpha}{2} > 0.
\end{align*}

Por lo tanto,

\[
|(g,1)| = \sqrt{2} \cdot \left| \left( g, \frac{1}{\sqrt{2}} \right) \right| > \sqrt{2} \cdot \frac{\alpha}{2} > 0,
\]

lo cual implica que \( (g,1) \neq 0 \), es decir, \( g \notin M \). Así, \( B \subseteq L^2((-1,1)) \setminus M \), y por lo tanto, \( L^2((-1,1)) \setminus M \) es abierto.

Concluimos que \( M \) es cerrado en \( L^2((-1,1)) \).
\item[III)] \textbf{M es subespacio:} Recordemos que
\[
M = \left\{ f \in L^2((-1,1)) : \int_{-1}^{1} f(x)\,dx = 0 \right\}.
\]

Sea \( f, g \in M \) y \( \lambda \in \mathbb{R} \). Entonces:

\[
\int_{-1}^{1} (\lambda f(x) + g(x))\, dx = \lambda \int_{-1}^{1} f(x)\, dx + \int_{-1}^{1} g(x)\, dx = \lambda \cdot 0 + 0 = 0.
\]

Por tanto, \( \lambda f + g \in M \), y así \( M \) es un subespacio vectorial de \( L^2((-1,1)) \).

\item[IV)] \textbf{Proyección ortogonal sobre \( M \):}  Sea \( f \in L^2((-1,1)) \), queremos encontrar \( g \in M \) tal que
\[
(f - g, h) = 0 \quad \text{para toda } h \in M,
\]
es decir,
\[
\int_{-1}^{1} (f(x) - g(x)) h(x)\, dx = 0.
\]

Proponemos
\[
g(x) := f(x) - \frac{1}{2} \int_{-1}^{1} f(t)\,dt,
\]
y veremos que \( g = P_M f \).

\vspace{0.5em}
Primero, verifiquemos que \( g \in M \):

\begin{align*}
\int_{-1}^{1} g(x)\, dx &= \int_{-1}^{1} \left( f(x) - \frac{1}{2} \int_{-1}^{1} f(t)\, dt \right)\, dx \\
&= \int_{-1}^{1} f(x)\, dx - \frac{1}{2} \int_{-1}^{1} f(t)\, dt \cdot \int_{-1}^{1} dx \\
&= \int_{-1}^{1} f(x)\, dx - \frac{1}{2} \int_{-1}^{1} f(t)\, dt \cdot 2 \\
&= \int_{-1}^{1} f(x)\, dx - \int_{-1}^{1} f(x)\, dx = 0.
\end{align*}

Así, \( g \in M \).

\vspace{0.5em}
Ahora, tomemos \( h \in M \) y calculemos \( (f - g, h) \):

\begin{align*}
(f - g, h) &= \int_{-1}^{1} \left[ f(x) - \left( f(x) - \frac{1}{2} \int_{-1}^{1} f(t)\, dt \right) \right] h(x)\, dx \\
&= \int_{-1}^{1} \left( \frac{1}{2} \int_{-1}^{1} f(t)\, dt \right) h(x)\, dx \\
&= \left( \frac{1}{2} \int_{-1}^{1} f(t)\, dt \right) \left( \int_{-1}^{1} h(x)\, dx \right).
\end{align*}

Como \( h \in M \), se tiene que \( \int_{-1}^{1} h(x)\, dx = 0 \), y por lo tanto:
\[
(f - g, h) = 0.
\]

Así, \( g = f(x) - \frac{1}{2} \int_{-1}^{1} f(t)\, dt \) es la proyección ortogonal de \( f \) sobre \( M \), es decir,

\[
(P_M f)(x) = f(x) - \frac{1}{2} \int_{-1}^{1} f(t)\,dt.
\]

        \end{enumerate}
    \end{proof}
    \item \( M = \{ f \in L^2((-1,1)) : f(x) = 0 \text{ para casi todo } x \in (-1,0) \} \).
    \begin{proof}
        \hfill
        \begin{enumerate}
            \item[I)] \textbf{M es no vacío:} Consideremos la función característica

\[
f(x) := \chi_{[0,1)}(x).
\]

Entonces \( f(x) = 0 \) para todo \( x \in (-1,0) \), por lo tanto también se anula para casi todo \( x \in (-1,0) \), y claramente es medible. Verificamos que \( f \in L^2((-1,1)) \):

\[
\|f\|_{L^2((-1,1))}^2 = \int_{-1}^{1} |f(x)|^2\, dx = \int_{-1}^{1} \chi_{[0,1)}(x)\, dx = \int_{0}^{1} 1\, dx = 1.
\]

Luego \( f \in L^2((-1,1)) \) y \( f \in M \), de modo que
$M \neq \varnothing$.
\item[II)] \textbf{M es cerrado:} Notemos que
\[
f \in M \iff \int_{-1}^{0} |f(x)|^2\, dx = 0 \iff \| \chi_{(-1,0)} f \|_{L^2((-1,1))} = 0,
\]
donde \( \chi_{(-1,0)} \) denota la función característica del intervalo \( (-1,0) \).

Sea \( (f_n)_{n=1}^\infty \subseteq M \) tal que \( f_n \to f \) en \( L^2((-1,1)) \). Queremos probar que \( f \in M \), es decir, que
\[
\| \chi_{(-1,0)} f \|_{L^2((-1,1))} = 0.
\]

Para ello, primero observamos que
\[
\| \chi_{(-1,0)} f_n - \chi_{(-1,0)} f \|_{L^2((-1,1))} = \| \chi_{(-1,0)} (f_n - f) \|_{L^2((-1,1))}.
\]

Usando la definición de la norma \( L^2 \):
\begin{align*}
\| \chi_{(-1,0)} (f_n - f) \|_{L^2((-1,1))}^2 &= \int_{-1}^{1} \left| \chi_{(-1,0)}(x) (f_n(x) - f(x)) \right|^2 dx \\
&= \int_{-1}^{0} |f_n(x) - f(x)|^2 dx.
\end{align*}

Como \( f_n \to f \) en \( L^2((-1,1)) \), se tiene que
\[
\int_{-1}^{0} |f_n(x) - f(x)|^2 dx \leq \int_{-1}^{1} |f_n(x) - f(x)|^2 dx \to 0,
\]
por lo que
\[
\| \chi_{(-1,0)} f_n - \chi_{(-1,0)} f \|_{L^2((-1,1))} \to 0,
\]
es decir,
\[
\chi_{(-1,0)} f_n \to \chi_{(-1,0)} f \quad \text{en } L^2((-1,1)).
\]

Sea ahora \( \varepsilon > 0 \). Como \( \chi_{(-1,0)} f_n \to \chi_{(-1,0)} f \), existe \( N \in \mathbb{N} \) tal que si \( n \geq N \), entonces
\[
\| \chi_{(-1,0)} (f_n - f) \|_{L^2((-1,1))} < \varepsilon.
\]

Para \( n \geq N \), se cumple:
\begin{align*}
\| \chi_{(-1,0)} f \|_{L^2((-1,1))} 
&= \| \chi_{(-1,0)} f - \chi_{(-1,0)} f_n + \chi_{(-1,0)} f_n \|_{L^2((-1,1))} \\
&\leq \| \chi_{(-1,0)} (f - f_n) \|_{L^2((-1,1))} + \| \chi_{(-1,0)} f_n \|_{L^2((-1,1))}.
\end{align*}

Como \( f_n \in M \), se tiene \( \chi_{(-1,0)} f_n(x) = 0 \) para casi todo $x\in (-1,0)$, por lo que:
\[
\| \chi_{(-1,0)} f \|_{L^2((-1,1))} < \varepsilon.
\]

Como \( \varepsilon > 0 \) es arbitrario, se deduce que:
\[
\| \chi_{(-1,0)} f \|_{L^2((-1,1))} = 0,
\]
es decir, \( f(x) = 0 \) para casi todo \( x \in (-1,0) \), por lo tanto \( f \in M \).
\item[III)] \textbf{M es subespacio:} Sean \( f, g \in M \) y \( \lambda \in \mathbb{R} \). Por definición de \( M \), existen conjuntos de medida nula \( F \) y \( G \) tales que
\[
f(x) = 0 \quad \text{para todo } x \in (-1,0) \setminus F,
\]
\[
g(x) = 0 \quad \text{para todo } x \in (-1,0) \setminus G.
\]

Como \( 0 \leq \mu(F \cup G) \leq \mu(F) + \mu(G) = 0 \), se sigue que \( F \cup G \) es también de medida nula.

Además, como
\[
(-1,0) \setminus (F \cup G) \subseteq (-1,0) \setminus F \quad \text{y} \quad (-1,0) \setminus (F \cup G) \subseteq (-1,0) \setminus G,
\]
entonces para todo \( x \in (-1,0) \setminus (F \cup G) \),
\[
(f + g)(x) = f(x) + g(x) = 0 + 0 = 0,
\]
es decir, \( f + g \in M \).

Por otro lado, si \( x \in (-1,0) \setminus F \), entonces \( f(x) = 0 \), por lo tanto \( \lambda f(x) = \lambda \cdot 0 = 0 \), así que \( \lambda f \in M \).Concluimos que \( M \) es un subespacio vectorial de \( L^2((-1,1)) \).
\item[IV)] \textbf{Proyección ortogonal sobre \( M \):} Sea \( f \in L^2((-1,1)) \), queremos encontrar \( g \in M \) tal que 
\[
(f - g, h) = 0 \quad \text{para toda } h \in M.
\]
Tomemos \( g = \chi_{[0,1)} f \) y veamos que \( g = P_M f \). Claramente, \( g(x) = 0 \) para todo \( x \in (-1,0) \), por lo cual \( g \in M \).

Sea \( h \in M \), entonces:
\begin{align*}
(f - g, h) &= \int_{-1}^1 \left(f(x) - \chi_{[0,1)}(x) f(x)\right) h(x) \, dx \\
&= \int_{-1}^1 \chi_{(-1,0)}(x) f(x) h(x) \, dx \\
&= \int_{-1}^0 f(x) h(x) \, dx = 0,
\end{align*}
dado que \( h(x) = 0 \) para casi todo \( x \in (-1,0) \), pues \( h \in M \).

Así concluimos que \( g = \chi_{[0,1)} f = P_M f \).



        \end{enumerate}
    \end{proof}


\item[(c)] $M=\left\{f \in L^2((-1,1)): f(x)=0\right.$ para casi todo $\left.x \in(-1,0)\right\}$.
\begin{proof}
        \hfill
        \begin{enumerate}
            \item[I)] \textbf{M es no vacío:} Consideremos la función característica

\[
f(x) := \chi_{[0,1)}(x).
\]

Entonces \( f(x) = 0 \) para todo \( x \in (-1,0) \), por lo tanto también se anula para casi todo \( x \in (-1,0) \), y claramente es medible. Verificamos que \( f \in L^2((-1,1)) \):

\[
\|f\|_{L^2((-1,1))}^2 = \int_{-1}^{1} |f(x)|^2\, dx = \int_{-1}^{1} \chi_{[0,1)}(x)\, dx = \int_{0}^{1} 1\, dx = 1.
\]

Luego \( f \in L^2((-1,1)) \) y \( f \in M \), de modo que
$M \neq \varnothing$.
\item[II)] \textbf{M es cerrado:} Notemos que
\[
f \in M \iff \int_{-1}^{0} |f(x)|^2\, dx = 0 \iff \| \chi_{(-1,0)} f \|_{L^2((-1,1))} = 0,
\]
donde \( \chi_{(-1,0)} \) denota la función característica del intervalo \( (-1,0) \).

Sea \( (f_n)_{n=1}^\infty \subseteq M \) tal que \( f_n \to f \) en \( L^2((-1,1)) \). Queremos probar que \( f \in M \), es decir, que
\[
\| \chi_{(-1,0)} f \|_{L^2((-1,1))} = 0.
\]

Para ello, primero observamos que
\[
\| \chi_{(-1,0)} f_n - \chi_{(-1,0)} f \|_{L^2((-1,1))} = \| \chi_{(-1,0)} (f_n - f) \|_{L^2((-1,1))}.
\]

Usando la definición de la norma \( L^2 \):
\begin{align*}
\| \chi_{(-1,0)} (f_n - f) \|_{L^2((-1,1))}^2 &= \int_{-1}^{1} \left| \chi_{(-1,0)}(x) (f_n(x) - f(x)) \right|^2 dx \\
&= \int_{-1}^{0} |f_n(x) - f(x)|^2 dx.
\end{align*}

Como \( f_n \to f \) en \( L^2((-1,1)) \), se tiene que
\[
\int_{-1}^{0} |f_n(x) - f(x)|^2 dx \leq \int_{-1}^{1} |f_n(x) - f(x)|^2 dx \to 0,
\]
por lo que
\[
\| \chi_{(-1,0)} f_n - \chi_{(-1,0)} f \|_{L^2((-1,1))} \to 0,
\]
es decir,
\[
\chi_{(-1,0)} f_n \to \chi_{(-1,0)} f \quad \text{en } L^2((-1,1)).
\]

Sea ahora \( \varepsilon > 0 \). Como \( \chi_{(-1,0)} f_n \to \chi_{(-1,0)} f \), existe \( N \in \mathbb{N} \) tal que si \( n \geq N \), entonces
\[
\| \chi_{(-1,0)} (f_n - f) \|_{L^2((-1,1))} < \varepsilon.
\]

Para \( n \geq N \), se cumple:
\begin{align*}
\| \chi_{(-1,0)} f \|_{L^2((-1,1))} 
&= \| \chi_{(-1,0)} f - \chi_{(-1,0)} f_n + \chi_{(-1,0)} f_n \|_{L^2((-1,1))} \\
&\leq \| \chi_{(-1,0)} (f - f_n) \|_{L^2((-1,1))} + \| \chi_{(-1,0)} f_n \|_{L^2((-1,1))}.
\end{align*}

Como \( f_n \in M \), se tiene \( \chi_{(-1,0)} f_n(x) = 0 \) para casi todo $x\in (-1,0)$, por lo que:
\[
\| \chi_{(-1,0)} f \|_{L^2((-1,1))} < \varepsilon.
\]

Como \( \varepsilon > 0 \) es arbitrario, se deduce que:
\[
\| \chi_{(-1,0)} f \|_{L^2((-1,1))} = 0,
\]
es decir, \( f(x) = 0 \) para casi todo \( x \in (-1,0) \), por lo tanto \( f \in M \).
\item[III)] \textbf{M es subespacio:} Sean \( f, g \in M \) y \( \lambda \in \mathbb{R} \). Por definición de \( M \), existen conjuntos de medida nula \( F \) y \( G \) tales que
\[
f(x) = 0 \quad \text{para todo } x \in (-1,0) \setminus F,
\]
\[
g(x) = 0 \quad \text{para todo } x \in (-1,0) \setminus G.
\]

Como \( 0 \leq \mu(F \cup G) \leq \mu(F) + \mu(G) = 0 \), se sigue que \( F \cup G \) es también de medida nula.

Además, como
\[
(-1,0) \setminus (F \cup G) \subseteq (-1,0) \setminus F \quad \text{y} \quad (-1,0) \setminus (F \cup G) \subseteq (-1,0) \setminus G,
\]
entonces para todo \( x \in (-1,0) \setminus (F \cup G) \),
\[
(f + g)(x) = f(x) + g(x) = 0 + 0 = 0,
\]
es decir, \( f + g \in M \).

Por otro lado, si \( x \in (-1,0) \setminus F \), entonces \( f(x) = 0 \), por lo tanto \( \lambda f(x) = \lambda \cdot 0 = 0 \), así que \( \lambda f \in M \).Concluimos que \( M \) es un subespacio vectorial de \( L^2((-1,1)) \).
\item[IV)] \textbf{Proyección ortogonal sobre \( M \):} Sea \( f \in L^2((-1,1)) \), queremos encontrar \( g \in M \) tal que 
\[
(f - g, h) = 0 \quad \text{para toda } h \in M.
\]
Tomemos \( g = \chi_{[0,1)} f \) y veamos que \( g = P_M f \). Claramente, \( g(x) = 0 \) para todo \( x \in (-1,0) \), por lo cual \( g \in M \).

Sea \( h \in M \), entonces:
\begin{align*}
(f - g, h) &= \int_{-1}^1 \left(f(x) - \chi_{[0,1)}(x) f(x)\right) h(x) \, dx \\
&= \int_{-1}^1 \chi_{(-1,0)}(x) f(x) h(x) \, dx \\
&= \int_{-1}^0 f(x) h(x) \, dx = 0,
\end{align*}
dado que \( h(x) = 0 \) para casi todo \( x \in (-1,0) \), pues \( h \in M \).

Así concluimos que \( g = \chi_{[0,1)} f = P_M f \).



        \end{enumerate}
    \end{proof}
 \end{itemize}
(II) Sea $\Omega \subset \mathbb{R}^n$ un abierto acotado. Considere
$$
K=\left\{f \in L^2(\Omega): \int_{\Omega} f(x) d x \geq 1\right\}
$$
\begin{itemize}
\item[(a)] Muestre que $K$ es un conjunto cerrado convexo de $L^2(\Omega)$.
  \begin{proof}
        Notemos que
\[
K = \left\{ f \in L^2(\Omega) : (f,1) \geq 1 \right\},
\]
donde \( \Omega \subseteq \mathbb{R}^n \) es un abierto acotado. Entonces \( 0 < \mu(\Omega) < \infty \), y además:
\[
\|1\|_{L^2(\Omega)} = \left( \int_{\Omega} 1^2 \, dx \right)^{1/2} = (\mu(\Omega))^{1/2}.
\]

Sea \( g \in L^2(\Omega) \setminus K \), es decir, \( a := (g,1) < 1 \). Tomemos \( \varepsilon > 0 \) tal que \( a + \varepsilon < 1 \), y definimos
\[
\delta = \frac{\varepsilon}{\|1\|_{L^2(\Omega)}} = \frac{\varepsilon}{(\mu(\Omega))^{1/2}}>0.
\]
Sea \( B = B(g, \delta) \) la bola abierta centrada en \( g \) de radio \( \delta \) en \( L^2(\Omega) \). Sea \( f \in B \), entonces \( \|f - g\|_{L^2(\Omega)} < \delta \). Usando la desigualdad de Cauchy-Schwarz, se tiene:
\begin{align*}
(f,1) &= (f - g, 1) + (g,1) \\
&\leq \|f - g\|_{L^2(\Omega)} \cdot \|1\|_{L^2(\Omega)} + a \\
&< \delta \cdot (\mu(\Omega))^{1/2} + a = \varepsilon + a < 1.
\end{align*}
Por tanto, \( f \notin K \), lo que implica que \( B \subseteq L^2(\Omega) \setminus K \). Es decir, el complemento de \( K \) es abierto, y por lo tanto \( K \) es cerrado en \( L^2(\Omega) \).

Ahora veamos que \( K \) es convexo. Sean \( f, g \in K \) y \( t \in [0,1] \). Entonces:
\[
(tf + (1 - t)g, 1) = t(f,1) + (1 - t)(g,1) \geq t \cdot 1 + (1 - t) \cdot 1 = 1.
\]
Por tanto, \( tf + (1 - t)g \in K \), lo cual prueba que \( K \) es convexo.

    \end{proof}
\item[(b)] Determine la proyección sobre $K$, es decir, el operador $P_K$.
    \begin{proof}
        Procedamos a calcular \( P_K f \). Sea \( f \in L^2(\Omega) \), queremos encontrar \( g \in K \) tal que  
\[
(f - g, h - g) \leq 0 \quad \text{para toda } h \in K.
\]

Proponemos:
\[
g(x) = f(x) + \chi_{(-\infty,1)}(C_f) \cdot \frac{1 - C_f}{\mu(\Omega)},
\]
donde \( C_f = \int_{\Omega} f(y) \, dy \).

Veamos que esta \( g \) cumple las condiciones requeridas. Consideramos dos casos:

\medskip
\noindent
\textbf{Caso 1:} Si \( f \in K \), entonces \( C_f \geq 1 \), por lo que \( \chi_{(-\infty,1)}(C_f) = 0 \). Así,
\[
g = f \in K, \quad \text{y } (f - g, h - g) = (0, h - f) = 0 \leq 0 \quad \forall h \in K.
\]
Entonces, \( P_K f = f \).

\medskip
\noindent
\textbf{Caso 2:} Si \( f \notin K \), entonces \( C_f < 1 \), así que \( \chi_{(-\infty,1)}(C_f) = 1 \) y:
\begin{align*}
\int_{\Omega} g(x) \, dx &= \int_{\Omega} \left( f(x) + \frac{1 - C_f}{\mu(\Omega)} \right) dx \\
&= \int_{\Omega} f(x) \, dx + \frac{1 - C_f}{\mu(\Omega)} \cdot \mu(\Omega) \\
&= C_f + (1 - C_f) = 1.
\end{align*}
Por tanto, \( g \in K \).

Sea \( h \in K \), definimos \( C_h = \int_{\Omega} h(x) \, dx \). Entonces \( C_h \geq 1 \). Calculamos:
\begin{align*}
(f - g, h - g) &= \left( -\frac{1 - C_f}{\mu(\Omega)}, h - f - \frac{1 - C_f}{\mu(\Omega)} \right) \\
&= -\frac{1 - C_f}{\mu(\Omega)} \cdot \int_{\Omega} \left( h(x) - f(x) - \frac{1 - C_f}{\mu(\Omega)} \right) dx \\
&= -\frac{1 - C_f}{\mu(\Omega)} \left( C_h - C_f - \frac{1 - C_f}{\mu(\Omega)} \cdot \mu(\Omega) \right) \\
&= -\frac{1 - C_f}{\mu(\Omega)} (C_h - C_f - (1 - C_f)) \\
&= -\frac{1 - C_f}{\mu(\Omega)} (C_h - 1).
\end{align*}

Como \( 1 - C_f > 0 \) y \( C_h - 1 \geq 0 \), se concluye que
\[
(f - g, h - g) \leq 0.
\]

De esta forma, \( g = P_K f \).

    \end{proof}
\end{itemize}


