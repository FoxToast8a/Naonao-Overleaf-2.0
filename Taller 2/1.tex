%!TEX root = main.tex
Considere que una fuente sin memoria emite los símbolos del alfabeto $S = \{a, b, c\}$. Se sabe que la distribución de probabilidad de los símbolos en la salida del canal es 
\[
P_Y = \{0\text{.}6, 0\text{.}3, 0\text{.}1\}.
\]
Los símbolos entran a un canal ruidoso que puede cambiar unos símbolos en otros de acuerdo a la distribución de probabilidad que muestra la siguiente tabla de probabilidades condicionales $p(x|y)$:

\[
\begin{array}{c|ccc}
\text{entra} \backslash \text{sale} & a & b & c \\
\hline
a & 0\text{.}95 & 0\text{.}03 & 0\text{.}02 \\
b & 0\text{.}03 & 0\text{.}95 & 0\text{.}02 \\
c & 0\text{.}02 & 0\text{.}02 & 0\text{.}96 \\
\end{array}
\]

Denote por $X$ y $Y$ las variables aleatorias que modelan los símbolos que salen de la fuente y del canal respectivamente. Resuelva las siguientes preguntas:

\begin{enumerate}
    \item[A]. Determine las entropías de la fuente y el canal.
    \item[B]. Determine la entropía del sistema.
    \item[C]. Determine e interprete las entropías condicionales $H(X|Y)$ y $H(Y|X)$.
    \item[D]. Calcule la información común entre la entrada y la salida del sistema.
    \item[E]. Determine cuánta información se perdió dentro del canal.
    \item[F]. Determine la pérdida de información por símbolo.
\end{enumerate}


