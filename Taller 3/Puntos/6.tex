%!TEX root = ../main.tex

Una fuente $F$ genera símbolos de un alfabeto $\{a, b, c\}$ y se modela como un proceso markoviano $\{X_i\}_{i \in \mathbb{N}}$ con matriz de transición
\begin{equation*}
    P = \begin{bmatrix} 0\text{.}6 & 0\text{.}3 & 0\text{.}1 \\ 0\text{.}2 & 0\text{.}5 & 0\text{.}3 \\ 0\text{.}1 & 0\text{.}4 & 0\text{.}5 \end{bmatrix}
\end{equation*}

\begin{itemize}
    \item[A)] Determine la distribución de probabilidad para las variables aleatorias $X_i$.
    \item[B)] ¿Cuál es la probabilidad de que la fuente emita la palabra \texttt{aabc} (considerando que las letras salen de derecha a izquierda, es decir, primero \texttt{c})?
    \item[C)]  Dibuje el grafo de estados que modela el proceso markoviano.
\end{itemize}