%!TEX root = ../main.tex

Construya una matriz generadora para un código lineal binario que transforma los mensajes (columna izquierda) escritos en binario en los códigos (columna derecha) mostrados en la siguiente tabla:

\begin{center}
\begin{tabular}{c|c}
$z$ & $zG$ \\
\hline
0 & 0000000 \\
1 & 0001110 \\
2 & 0010101 \\
3 & 0011011 \\
4 & 0100011 \\
5 & 0101101 \\
6 & 0110110 \\
7 & 0111000 \\
8 & 1000111 \\
9 & 1001001 \\
10 & 1010010 \\
11 & 1011100 \\
12 & 1100100 \\
13 & 1101010 \\
14 & 1110001 \\
15 & 1111111 \\
\end{tabular}
\end{center}
\begin{sols}
    Primero notemos que los mensaje de la columna izquierda en binario consumen 4 digitos, es decir
    $$\begin{array}{c|c}
    Decimal & Binario\\
    0 & 0000 \\
1 & 0001 \\
2 & 0010 \\
3 & 0011 \\
4 & 0100 \\
5 & 0101 \\
6 & 0110 \\
7 & 0111 \\
8 & 1000 \\
9 & 1001 \\
10 & 1010 \\
11 & 1011 \\
12 & 1100 \\
13 & 1101 \\
14 & 1110 \\
15 & 1111 \\     
    \end{array}$$
    Luego note que para que las dimensiones de la matriz cuadren, necesitamos cuatro vectores linealmente independientes de la segunda columna, es decir de $zG$, luego resulta natural escoger los que correspoden a $z\in{0001,0010,0100,1000}$, ya que como esta es la primera parte de cada $zG$ resultan linealmente independientes de manera inmediata, por lo que la matriz generadora $G$ es
    $$G=\begin{bmatrix}
        1&0&0&0&1&1&1\\
        0&1&0&0&0&1&1\\
        0&0&1&0&1&0&1\\
        0&0&0&1&1&1&0\\

    \end{bmatrix}$$ 
    Note que esta eleccion de vectores automaticamente nos da la forma sistematica de la matriz.

\end{sols}

\begin{itemize}
    \item[A)] Clasifique el código de acuerdo a la notación $[n,m,d]$.
    \begin{sols}
        Note que nuestra matriz generadora tiene $4$ filas y $7$ columnas, esto quiere decir que el codigo es $[7,4,d]$, faltaria determinar la distancia minima, pero esto lo podemos hacer simplemente viendo el peso de cada palabra y escogiendo el peso minimo entre ellas.
        \begin{align*}         
 w(0001110)&=3 \\
w(0010101)&=3 \\
 w(0011011)&=4 \\
w(0100011)&=3 \\
w(0101101)&=4 \\
 w(0110110)&=4 \\
w(0111000)&=3 \\
 w(1000111)&=4 \\
w(1001001)&=3 \\
w(1010010)&=3 \\
w(1011100)&=4 \\
w(1100100)&=3 \\
 w(1101010)&=4 \\
w(1110001)&=4 \\
 w(1111111)&=7 \\
        \end{align*}
        De esta manera la distancia minima del codigo es $d=3$, por lo que la clasificacion del codigo es $[7,4,3].$

    \end{sols}

    \item[B)] Determine si el mensaje $z= 1011111$ trae errores, en tal caso calcule el síndrome, la coclase y corrija siempre y cuando sea posible.
    \begin{sols}
        Por simple inspeccion nos damos cuenta que la palabra $z$ tiene errores ya que $w(z)=6$ y ninguna palabra tiene ese peso en nuestro codigo. Recordemos que al inicio ya dimos la matriz $G$ en su forma sistematica por lo que tenemos que $G=[\begin{array}{c|c}
            I_4&P
        \end{array}]$, donde
        $$P=\begin{bmatrix}
            1 & 1 & 1 \\
            0 & 1 & 1 \\
            1 & 0 & 1 \\
            1 & 1 & 0
        \end{bmatrix}$$
        Asi como en $GF(2)$ tenemos que $-1=1$, tenemos que la matriz de verificacion para el codigo es
    $$H=[\begin{array}{c|c}
        -P^T&I_{7-4}
    \end{array}]=[\begin{array}{c|c}
        P^T&I_{3}
    \end{array}]=\begin{bmatrix}
        1 & 0 & 1 & 1 & 1 & 0 & 0 \\
        1 & 1 & 0 & 1 & 0 & 1 & 0 \\
        1 & 1 & 1 & 0 & 0 & 0 & 1
    \end{bmatrix}$$
    Ahora determinemos el sindrome, que sabemos que sera no nulo ya que la palabra no pertenece al codigo
    $$Hz^T=\begin{bmatrix}
        1 & 0 & 1 & 1 & 1 & 0 & 0 \\
        1 & 1 & 0 & 1 & 0 & 1 & 0 \\
        1 & 1 & 1 & 0 & 0 & 0 & 1
    \end{bmatrix}\begin{bmatrix}
        1\\
        0\\
        1\\
        1\\
        1\\
        1\\
        1\\
    \end{bmatrix}=\begin{bmatrix}
        0\\
        1\\
        1\\
    \end{bmatrix}$$
    Esto nos confirma que efectivamente la palabra tiene errores, la coclase respectiva es
    \begin{align*}
        (C+z)_H=\{&1011111,1010001,1001010,1000100,1111100,1110010,1101001,1100111,\\&0011000,0010110,0001101,0000011,0111011,0110101,0101110,0100000\}
    \end{align*}
    Note que el lider de la coclase es la palabra $0100000,$ ya que es la unica que tiene peso 1, por lo que el mensaje corregido seria 
    $$1011111-0100000=1111111.$$
    \end{sols}
    
    \item[C)]¿Cuántos errores puede detectar el código? Justifique.
    \begin{sols}
      Dado que $d=3$, y esa es la distancia minima de nuestro codigo, por lo visto en clase nuestro codigo puede detectar hasta $d-1=3-1=2$ errores.  

    \end{sols}
    \item[D)]¿Cuántos errores puede corregir el código? Justifique.
    \begin{sols}
        De manera similar al anterior punto el codigo puede corregir hasta
        $$\left\lfloor\frac{d-1}{2}\right\rfloor=\left\lfloor\frac{3-1}{2}\right\rfloor=1,$$
        es decir que nuestro codigo puede corregir hasta 1 error.
    \end{sols}
\end{itemize}