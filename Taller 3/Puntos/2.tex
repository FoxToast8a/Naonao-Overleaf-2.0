%!TEX root = ../main.tex

Construya una matriz generadora para un código lineal binario que transforma los mensajes (columna izquierda) escritos en binario en los códigos (columna derecha) mostrados en la siguiente tabla:

\begin{center}
\begin{tabular}{c|c}
$z$ & $zG$ \\
\hline
0 & 0000000 \\
1 & 0001110 \\
2 & 0010101 \\
3 & 0011011 \\
4 & 0100011 \\
5 & 0101101 \\
6 & 0110110 \\
7 & 0111000 \\
8 & 1000111 \\
9 & 1001001 \\
10 & 1010010 \\
11 & 1011100 \\
12 & 1100100 \\
13 & 1101010 \\
14 & 1110001 \\
15 & 1111111 \\
\end{tabular}
\end{center}

\begin{itemize}
    \item[A)] Clasifique el código de acuerdo a la notación $[n,m,d]$.
    \item[B)] Determine si el mensaje $z= 1011111$ trae errores, en tal caso calcule el síndrome, la coclase y corrija siempre y cuando sea posible.
    \item[C)]¿Cuántos errores puede detectar el código? Justifique.
    \item[D)]¿Cuántos errores puede corregir el código? Justifique.
\end{itemize}