%!TEX root = ../main.tex

Resuelva el ejercicio $6\text{.}7\text{.}6$ de las notas de clase.\\
Considere una fuente binaria con una distribución de probabilidades $\{p(0) = 0\text{.}3, p(1) = 0\text{.}7\}$ y un canal con matriz de transición:
\begin{equation*}
    \begin{bmatrix} 0\text{.}1 & 0\text{.}9 \\ 0\text{.}8 & 0\text{.}2 \end{bmatrix}
\end{equation*}
Si los símbolos de la fuente son codificados mediante las asignaciones $0 \to 000$, $1 \to 111$, determine una función decodificadora $\delta: \{0,1\}^3 \to \{0,1\}$ con máxima probabilidad de corrección.
\begin{sol}

Por los datos dados por el enunciado sabemos que el dominio de la función $\delta$ está dado por combinaciones de $0$ y $1$ de longitud $3$. Luego, como tenemos la probabilidad de cada uno de los símbolos y la matriz de transición, podemos tomar a $\delta$ como una función de decodificación para una fuente que se codifica por repetición, y así tendríamos la máxima probabilidad de corrección.

$$
\delta(y_1 y_2 \dots y_n) =
\begin{cases}
0, & \text{si} \quad p(0) \prod_{i=1}^n p(y_i | 0) \geq p(1) \prod_{i=1}^n p(y_i | 1), \\
1, & \text{en otro caso}.
\end{cases}
$$
Por lo cual evaluamos en cada una de las cadenas y así encontramos la función decodificadora.
\begin{itemize}
    \item Cadena $000$
\end{itemize}
Calculamos las probabilidades ,
\begin{align*}
    p(0) p(0 | 0) p(0 | 0) p(0 | 0) &= (0.3)(0.1)(0.1)(0.1) = 0.0003\\
    p(1) p(0 | 1) p(0 | 1) p(0 | 1) &= (0.7)(0.8)(0.8)(0.8) = 0.3584
.\end{align*}
Como $0.003<0.5120$ entonces $\delta(000)=1$

\begin{itemize}
    \item Cadena $100$
\end{itemize}
Hallamos las probabilidades, 
\begin{align*}
    p(0) p(1 | 0) p(0 | 0) p(0 | 0) &= (0.3)(0.9)(0.1)(0.1) = 0.0027\\
    p(1) p(1 | 1) p(0 | 1) p(0 | 1) &= (0.7)(0.2)(0.8)(0.8) = 0.0896
.\end{align*}
Puesto que $0.0027<0.0896$, $\delta(001)=1$

\begin{itemize}
    \item Cadena $010$
\end{itemize}
Realizando los cálculos de las probabilidades
\begin{align*}
p(0) p(0 | 0) p(1 | 0) p(0 | 0) &= (0.3)(0.1)(0.9)(0.1) = 0.0027\\
    p(1) p(0 | 1) p(1 | 1) p(0 | 1) &= (0.7)(0.8)(0.2)(0.8) = 0.0896\\
.\end{align*}
Así, como $0.0027<0.0896$, $\delta(001)=1$

\begin{itemize}
    \item Cadena $001$
\end{itemize}
Las probabilidades para este caso nos dan 
\begin{align*}
    p(0) p(0 | 0) p(0 | 0) p(1 | 0) &= (0.3)(0.1)(0.1)(0.9) = 0.0027\\
    p(1) p(0 | 1) p(0 | 1) p(1 | 1) &= (0.7)(0.8)(0.8)(0.2) = 0.0896
.\end{align*}
Puesto que $0.0027<0.0896$, $\delta(001)=1$

\begin{itemize}
    \item Cadena $110$
\end{itemize}
Calculando las probabilidades,
\begin{align*}
    p(0) p(1 | 0) p(1 | 0) p(0 | 0) &= (0.3)(0.9)(0.9)(0.1) = 0.0243\\
    p(1) p(1 | 1) p(1 | 1) p(0 | 1) &= (0.7)(0.2)(0.2)(0.8) = 0.0224
.\end{align*}
Luego, como $0.0243> 0.224$ entonces $\delta(110)=0$


\begin{itemize}
    \item Cadena $101$
\end{itemize}
La probabilidad para obtener esta cadena dado cada uno de los símbolos es,
\begin{align*}
    p(0) p(1 | 0) p(0 | 0) p(1 | 0) &= (0.3)(0.9)(0.1)(0.9) = 0.0243\\
    p(1) p(1 | 1) p(0 | 1) p(1 | 1) &= (0.7)(0.2)(0.8)(0.2) = 0.0224
.\end{align*}
Como se tiene que $0.0243> 0.224$ entonces $\delta(101)=0$


\begin{itemize}
    \item Cadena $011$
\end{itemize}

Calculando la probabilidad,
\begin{align*}
p(0) p(0 | 0) p(1 | 0) p(1 | 0) &= (0.3)(0.1)(0.9)(0.9) = 0.0243\\
    p(1) p(0 | 1) p(1 | 1) p(1 | 1) &= (0.7)(0.8)(0.2)(0.2) = 0.0224
.\end{align*}
Se tiene que $\delta(011)=0$, puesto que $0.0243> 0.224$.

\begin{itemize}
    \item Cadena $111$
\end{itemize}
Por último calculamos la probabilidad de
\begin{align*}
p(0) p(1 | 0) p(1 | 0) p(1 | 0) &= (0.3)(0.9)(0.9)(0.9) = 0.2187\\
    p(1) p(1 | 1) p(1 | 1) p(1 | 1) &= (0.7)(0.2)(0.2)(0.2)=0.0056
.\end{align*}
Con lo que decimos que $\delta=0$, porque $0.2187>0.0056$

Por lo que concluimos que 
\[
\delta(y_1 y_2 y_3) =
\begin{cases}
1, & \text{si } y_1 y_2 y_3 = 000, \\
1, & \text{si } y_1 y_2 y_3 = 100, \\
1, & \text{si } y_1 y_2 y_3 = 001, \\
1, & \text{si } y_1 y_2 y_3 = 010, \\
0, & \text{si } y_1 y_2 y_3 = 110, \\
0, & \text{si } y_1 y_2 y_3 = 101, \\
0, & \text{si } y_1 y_2 y_3 = 011, \\
0, & \text{si } y_1 y_2 y_3 = 111.
\end{cases}
\]










    
\end{sol}