%!TEX root = main.tex

Una fuente genera símbolos con una distribución de probabilidad 
$$\{0.729, 0.081, 0.081, 0.081, 0.009, 0.009, 0.009, 0.001\}.$$
Determine un código binario que permita calcular la codificación de los símbolos de la fuente con menor longitud promedio. Cálcule la eficiencia del código.
\begin{sols}
    Como se nos pide hacerlo con la menor longitud promedio de palabra, podemos ir a lo seguro y aplicar el algoritmo de Huffman. Dado que $D=2,$ es decir, estamos trabajando en binario. Nuestro $r=0,$ por lo que no hay que añadir nodos nuevos y tenemos que el planteamiento inicial es
    \begin{center}
       \begin{tikzpicture}

            \node[](0){}
                child{node[red node, label=below:{$p_1$}]{}  edge from parent[draw=none]}
                child{node[red node, label=below:{$p_2$}]{}  edge from parent[draw=none]}
                child{node[red node, label=below:{$p_2$}]{}  edge from parent[draw=none]}
                child{node[red node, label=below:{$p_2$}]{}  edge from parent[draw=none]}
                child{node[red node, label=below:{$p_3$}]{}  edge from parent[draw=none]}
                child{node[red node, label=below:{$p_3$}]{}  edge from parent[draw=none]}
                child{node[red node, label=below:{$p_3$}]{}  edge from parent[draw=none]}
                child{node[red node, label=below:{$p_4$}]{}  edge from parent[draw=none]}

;
\end{tikzpicture}
    \end{center}
Donde $p_1=0,001,p_2=0,009,p_3=0,081$ y $p_4=0,729$, esto lo hacemos netamente por simpleza del árbol. Procedemos juntando los nodos de menor probabilidad por medio de un padre y sumamos sus probabilidades
\begin{center}
       \begin{tikzpicture}

            \node[](0){}
                child{node[red node, label=above:{$0,010$}]{}  edge from parent[draw=none]
                    child{node[solid, label=below:{$p_1$}]{}}
                    child{node[solid, label=below:{$p_2$}]{}}}
                child{node[red node, label=below:{$p_2$}]{}  edge from parent[draw=none]}
                child{node[red node, label=below:{$p_2$}]{}  edge from parent[draw=none]}
                child{node[red node, label=below:{$p_3$}]{}  edge from parent[draw=none]}
                child{node[red node, label=below:{$p_3$}]{}  edge from parent[draw=none]}
                child{node[red node, label=below:{$p_3$}]{}  edge from parent[draw=none]}
                child{node[red node, label=below:{$p_4$}]{}  edge from parent[draw=none]}

;
\end{tikzpicture}
    \end{center}
Luego reorganizamos nuevamente de orden menor a mayor
\begin{center}
       \begin{tikzpicture}

            \node[](0){}
                child{node[red node, label=below:{$p_2$}]{}  edge from parent[draw=none]}
                child{node[red node, label=below:{$p_2$}]{}  edge from parent[draw=none]}
                child{node[red node, label=above:{$0,010$}]{}  edge from parent[draw=none]
                    child{node[solid, label=below:{$p_1$}]{}}
                    child{node[solid, label=below:{$p_2$}]{}}}
                child{node[red node, label=below:{$p_3$}]{}  edge from parent[draw=none]}
                child{node[red node, label=below:{$p_3$}]{}  edge from parent[draw=none]}
                child{node[red node, label=below:{$p_3$}]{}  edge from parent[draw=none]}
                child{node[red node, label=below:{$p_4$}]{}  edge from parent[draw=none]}

;
\end{tikzpicture}
    \end{center}
    Volvemos a juntar los dos de menor probabilidad, es decir
    \begin{center}
       \begin{tikzpicture}[level 1/.style={sibling distance=25mm}, level 2/.style={sibling distance=15mm}]

            \node[](0){}
                child{node[red node, label=above:{$0,018$}]{}  edge from parent[draw=none]
                    child{node[solid, label=below:{$p_2$}]{}}
                    child{node[solid, label=below:{$p_2$}]{}}}
                child{node[red node, label=above:{$0,010$}]{}  edge from parent[draw=none]
                    child{node[solid, label=below:{$p_1$}]{}}
                    child{node[solid, label=below:{$p_2$}]{}}}
                child{node[red node, label=below:{$p_3$}]{}  edge from parent[draw=none]}
                child{node[red node, label=below:{$p_3$}]{}  edge from parent[draw=none]}
                child{node[red node, label=below:{$p_3$}]{}  edge from parent[draw=none]}
                child{node[red node, label=below:{$p_4$}]{}  edge from parent[draw=none]}

;
\end{tikzpicture}
    \end{center}
Nuevamente los reorganizamos
\begin{center}
       \begin{tikzpicture}[level 1/.style={sibling distance=25mm}, level 2/.style={sibling distance=15mm}]

            \node[](0){}
                child{node[red node, label=above:{$0,010$}]{}  edge from parent[draw=none]
                    child{node[solid, label=below:{$p_1$}]{}}
                    child{node[solid, label=below:{$p_2$}]{}}}
                child{node[red node, label=above:{$0,018$}]{}  edge from parent[draw=none]
                    child{node[solid, label=below:{$p_2$}]{}}
                    child{node[solid, label=below:{$p_2$}]{}}}
                child{node[red node, label=below:{$p_3$}]{}  edge from parent[draw=none]}
                child{node[red node, label=below:{$p_3$}]{}  edge from parent[draw=none]}
                child{node[red node, label=below:{$p_3$}]{}  edge from parent[draw=none]}
                child{node[red node, label=below:{$p_4$}]{}  edge from parent[draw=none]}

;
\end{tikzpicture}
    \end{center}
Ahora procedemos aunir los dos nodos de menor probabilidad
\begin{center}
       \begin{tikzpicture}[level 1/.style={sibling distance=25mm}, level 2/.style={sibling distance=20mm}, level 3/.style={sibling distance=15mm}]

            \node[](0){}
                child{node[red node, label=above:{$0,028$}]{}  edge from parent[draw=none]
                    child{node[solid]{}
                        child{node[solid, label=below:{$p_1$}]{}}
                        child{node[solid, label=below:{$p_2$}]{}}}
                    child{node[solid]{} 
                        child{node[solid, label=below:{$p_2$}]{}}
                        child{node[solid, label=below:{$p_2$}]{}}}}
                child{node[red node, label=below:{$p_3$}]{}  edge from parent[draw=none]}
                child{node[red node, label=below:{$p_3$}]{}  edge from parent[draw=none]}
                child{node[red node, label=below:{$p_3$}]{}  edge from parent[draw=none]}
                child{node[red node, label=below:{$p_4$}]{}  edge from parent[draw=none]}

;
\end{tikzpicture}
    \end{center}
Note que ya esta organizado, asi que juntamos los dos nodos de menor probabilidad
\begin{center}
       \begin{tikzpicture}[level 1/.style={sibling distance=30mm}, level 2/.style={sibling distance=25mm}, level 3/.style={sibling distance=20mm}, level 4/.style={sibling distance=15mm}]

            \node[](0){}
                child{node[red node, label=above:{$0,109$}]{}  edge from parent[draw=none]
                    child{node[solid]{}
                        child{node[solid]{}
                            child{node[solid, label=below:{$p_1$}]{}}
                            child{node[solid, label=below:{$p_2$}]{}}}
                        child{node[solid]{} 
                            child{node[solid, label=below:{$p_2$}]{}}
                            child{node[solid, label=below:{$p_2$}]{}}}}
                    child{node[solid, label=below:{$p_3$}]{}}}
                child{node[red node, label=below:{$p_3$}]{}  edge from parent[draw=none]}
                child{node[red node, label=below:{$p_3$}]{}  edge from parent[draw=none]}
                child{node[red node, label=below:{$p_4$}]{}  edge from parent[draw=none]}

;
\end{tikzpicture}
    \end{center}
Note que ahora si tenemos que reorganizar, luego
\begin{center}
       \begin{tikzpicture}[level 1/.style={sibling distance=30mm}, level 2/.style={sibling distance=25mm}, level 3/.style={sibling distance=20mm}, level 4/.style={sibling distance=15mm}]

            \node[](0){}
                child{node[red node, label=below:{$p_3$}]{}  edge from parent[draw=none]}
                child{node[red node, label=below:{$p_3$}]{}  edge from parent[draw=none]}
                child{node[red node, label=above:{$0,109$}]{}  edge from parent[draw=none]
                    child{node[solid]{}
                        child{node[solid]{}
                            child{node[solid, label=below:{$p_1$}]{}}
                            child{node[solid, label=below:{$p_2$}]{}}}
                        child{node[solid]{} 
                            child{node[solid, label=below:{$p_2$}]{}}
                            child{node[solid, label=below:{$p_2$}]{}}}}
                    child{node[solid, label=below:{$p_3$}]{}}}
                child{node[red node, label=below:{$p_4$}]{}  edge from parent[draw=none]}

;
\end{tikzpicture}
    \end{center}
Procedemos a juntar los de menor probabilidad
\begin{center}
       \begin{tikzpicture}[level 1/.style={sibling distance=33mm}, level 2/.style={sibling distance=25mm}, level 3/.style={sibling distance=20mm}, level 4/.style={sibling distance=15mm}]

            \node[](0){}
                child{node[red node, label=above:{$0,162$}]{}  edge from parent[draw=none]
                    child{node[solid, label=below:{$p_3$}]{}}
                    child{node[solid, label=below:{$p_3$}]{}}}
                child{node[red node, label=above:{$0,109$}]{}  edge from parent[draw=none]
                    child{node[solid]{}
                        child{node[solid]{}
                            child{node[solid, label=below:{$p_1$}]{}}
                            child{node[solid, label=below:{$p_2$}]{}}}
                        child{node[solid]{} 
                            child{node[solid, label=below:{$p_2$}]{}}
                            child{node[solid, label=below:{$p_2$}]{}}}}
                    child{node[solid, label=below:{$p_3$}]{}}}
                child{node[red node, label=below:{$p_4$}]{}  edge from parent[draw=none]}

;
\end{tikzpicture}
    \end{center}
Como no estan en orden, los vamos a organizar y de paso juntamos los de menor probabilidad, tal que
\begin{center}
       \begin{tikzpicture}[level 1/.style={sibling distance=35mm}, level 2/.style={sibling distance=30mm}, level 3/.style={sibling distance=25mm}, level 4/.style={sibling distance=20mm}, level 5/.style={sibling distance=15mm}]

            \node[](0){}
                child{node[red node, label=above:{$0,271$}]{}  edge from parent[draw=none]
                child{node[solid]{}
                    child{node[solid]{}
                        child{node[solid]{}
                            child{node[solid, label=below:{$p_1$}]{}}
                            child{node[solid, label=below:{$p_2$}]{}}}
                        child{node[solid]{} 
                            child{node[solid, label=below:{$p_2$}]{}}
                            child{node[solid, label=below:{$p_2$}]{}}}}
                    child{node[solid, label=below:{$p_3$}]{}}}
                child{node[solid]{}
                    child{node[solid, label=below:{$p_3$}]{}}
                    child{node[solid, label=below:{$p_3$}]{}}}}
                child{node[red node, label=below:{$p_4$}]{}  edge from parent[draw=none]}

;
\end{tikzpicture}
    \end{center}
Luego como ya están en orden, simplemente creamos un nodo mas como raíz ya que solo quedan 2. Así el árbol de codificación es
\begin{center}
       \begin{tikzpicture}[level 1/.style={sibling distance=35mm}, level 2/.style={sibling distance=30mm}, level 3/.style={sibling distance=25mm}, level 4/.style={sibling distance=20mm}, level 5/.style={sibling distance=15mm}]

            \node[hollow](0){}
                child{node[solid]{}
                child{node[solid]{}
                    child{node[solid]{}
                        child{node[solid]{}
                            child{node[solid, label=below:{$p_1$}]{}}
                            child{node[solid, label=below:{$p_2$}]{}}}
                        child{node[solid]{} 
                            child{node[solid, label=below:{$p_2$}]{}}
                            child{node[solid, label=below:{$p_2$}]{}}}}
                    child{node[solid, label=below:{$p_3$}]{}}}
                child{node[solid]{}
                    child{node[solid, label=below:{$p_3$}]{}}
                    child{node[solid, label=below:{$p_3$}]{}}}}
                child{node[solid, label=below:{$p_4$}]{}}

;
\end{tikzpicture}
    \end{center}
Luego si asignamos a las ramas de la izquierda 0, y a las de la derecha 1 obtenemos el código $C$ asociado a cada valor de la distribución de probabilidad tal que
\begin{center}
    \begin{tabular}{|c|c|c|c|c|c|c|c|}
       \hline
        0,729& 0,081& 0,081& 0,081& 0,009& 0,009& 0,009& 0,001\\
        \hline
        1& 001& 010& 011 & 00001 &00010&00011&00000\\
        \hline
    \end{tabular}
\end{center}
Luego la longitud promedio de palabra viene dada por 
\begin{align*}
    L(C)&=\sum_{i=1}^8p_i|C(s_i)|\\
    &=(0,001+0,009+0,009+0,009)5+(0,081+0,081+0,081)3+0,729\\
    &=0,14+0,729+0,729\\
    &=1,598\quad bits/simbolo.
\end{align*}
Mientras que la entropía de la fuente es
\begin{align*}
    H(F)&=-\sum_{i=1}^8p_i\log_2(p_i)\\
    &=-(0,001\log_2(0,001)+3\cdot0,009\log_2(0,009)+3\cdot0,081\log_2(0,081)+0,729\log_2(0,729))\\
    &\approx 1,4069
\end{align*}
Luego
$$\eta=\frac{H(F)}{L(C)}\approx 88\%$$
\end{sols}