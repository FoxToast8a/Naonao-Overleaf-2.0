%!TEX root = main.tex

Determine si el código triario $C = \{ab, cb, abbc, cbc, abb\}$ es unívocamente decodificable. Existe un código binario instantáneo con las longitudes de palabras de C ? Si existe construyalo.

\begin{sols}
    Procedamos por medio del algoritmo. Al igual que hicimos el el punto 4 cuando consideremos ecuaciones, solo tendremos en cuenta los casos con longitudes de palabra que al menos puedan coincidir. Definimos $C_0=C,$ ahora por definición
    $$C_1=\{w\in D^+|\,uw=v, \text{ con } u\in C_0 \text{ y } v\in C \text{ o } u\in C \text{ y } v\in C_0 \}.$$
    Por lo que tenemos los siguientes casos:
    \begin{itemize}
        \item Si $u=ab,$ las ecuaciones con longitudes de palabra coherente son
        $$abw=\begin{cases}
            abbc\\
            cbc\\
            abb
        \end{cases}$$ 
        Luego posibles soluciones para $w$ son $bc$ en la primera y $b$ en la ultima, la segunda ecuación note que la palabra de la izquierda empieza por $a$ mientras que la derecha por $c$, así no hay solución.
        \item Si $u=cb$, por las longitudes de palabra consideramos solamente
        $$cbw=\begin{cases}
            abbc\\
            cbc\\
            abb
        \end{cases}$$
        Note que la primera ecuación y la ultima no tienen solución ya que la palabra de la izquierda empieza por $c$ mientras la de la derecha por $a.$ Pero note que para la segunda si tomamos $w=c$, cumplimos la ecuación
        \item Si $u=cbc,$ la única ecuación por longitudes es $cbcw=abbc,$ pero esta no tiene solución por que la palabra de la izquierda empieza por $c$, mientras la de la derecha empieza por $a$. En cambio si $u=abb,$ la ecuación $abbw=abbc$ tiene como solución $w=c.$
    \end{itemize}
    Así con todos los casos inspeccionados tenemos que $C_1=\{b,c,bc\}.$ Ahora tenemos que 
    $$C_2=\{w\in D^+|\,uw=v, \text{ con } u\in C_1 \text{ y } v\in C \text{ o } u\in C \text{ y } v\in C_1 \}.$$
    Note que por las longitudes de palabra las ecuaciones de la forma $uw=v$, donde $u\in C$ y $v\in C_1$, ya que $|uw|\geq 3$ mientras que $|v|\leq 2.$ Así que solo observaremos cuando $u\in C_1$ y $c\in C$.
    \begin{itemize}
        \item Note que para los casos donde $u=b$ y $u=bc,$ la palabra $uw$ empieza por $b,$ mientras que todas las palabras del código $C$ empiezan por $a$ o $c,$ así ninguna ecuación tiene solución en este caso.
        \item Si $u=c$, basta considerar las ecuaciones que empiezan por ese símbolo, es decir, $cw=cb$ y $cw=cbc.$ Luego $w=b$ o $w=bc.$
    \end{itemize}
    De esta manera concluimos que $C_2=\{b,bc\}.$ Note ahora que para $C_3$ podemos argumentar de manera muy similar a $C_2.$ Para el caso donde $u\in C_2$ y $v\in C$ note que $u=b$ o $u=bc,$ luego $uw$ empieza por $b$, pero ninguna palabra en $C$ empieza por $b,$ así no hay soluciones. En caso de que $u\in C$ y $v\in C_2,$ tenemos que $|uw|\geq 3,$ mientras que $|v|\leq 2.$ Así concluimos que $C_3=C_4=\cdots=\varnothing.$ Luego
    $$C_\infty=\bigcup_{i=1}^\infty C_i=C_1\cup C_2=\{b,c,bc\}.$$
    De esta manera $C\cap C_\infty=\varnothing,$ y por el teorema de Sardinas-Patterson tenemos que $C$ es unívocamente decodificable.\\

    Ahora queremos ver si existe un código binario, es decir, $D=2,$ con longitudes de palabras $\ell_1=2,\ell_2=2,\ell_3=3,\ell_4=3$ y $\ell_5=4.$ Que sea instantáneo, esto lo podemos ver por medio de la desigualdad de Kraft
    \begin{align*}
        \sum_{i=1}^5D^{-\ell_i}&=2^{-2}+2^{-2}+2^{-3}++2^{-3}++2^{-4}\\
        &=2^{-4}(2^2+2^2+2+2+1)\\
        &\frac{4+4+2+2+1}{16}\\
        &=\frac{13}{16}\leq 1
    \end{align*}
    Luego tal código si existe, por lo tanto podemos construirlo.
    \textcolor{blue}{Mañana hago este grafo ya tengo sueñito}
\end{sols}