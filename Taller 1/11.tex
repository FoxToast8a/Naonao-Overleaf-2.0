%!TEX root = main.tex

Resuelva el ejercicio 2.3.1 de la página 36 de las notas de clase, calcule la eficiencia en cada caso.

Considere el alfabeto $S$ con la distribución de probabilidades que se muestra en la tabla:

\[
\begin{array}{|c|c|c|c|c|c|c|c|}
\hline
A & B & C & D & E & F & G & H \\
\hline
0.02 & 0.03 & 0.04 & 0.04 & 0.12 & 0.20 & 0.20 & 0.35 \\
\hline
\end{array}
\]

Suponga que se ha usado un código de Huffman para codificar los mensajes sobre un alfabeto binario. Si en el árbol de codificación se le asigna 1 a las ramas sobre la izquierda y 0 a las ramas sobre la derecha, ¿qué palabra representa la secuencia 111011111101110? Determine la longitud promedio de palabra para este código. ¿Cuál sería la codificación de los símbolos sobre un código triario?


\begin{sol}
    Hola :p
\end{sol}