%!TEX root = main.tex

Suponga que una fuente genera dígitos binarios con distribución uniforme, los mensajes son enviados a través de un canal cambia los símbolos que genera una fuente binaria con las probalidades que se muestran en la gráfica.\\
\textcolor{blue}{poner foto}
Responda cada una de las siguientes preguntas justificando su razonamiento:\\
\textbf{a).} ¿Cuál es el número más probable de errores que se puede encontrar en un mensaje de 5000 bits que pase por el canal?\\
\textbf{b).} Si se codifica cada bit que genera la fuente triplicandolo, en qué porcentaje se reduce el número de errores? Suponga que para la decodificación
se usa mayoría de bits por tripla.\\
\textbf{c).} Si se codifica cada bit que genera la fuente con una n-tupla y se usa mayoría de bit por n tupla en la decodificación, cuál sería la longitud
mínima n - tupla para obtener por lo menos un 95 $\%$ de certeza en la decondificación?

\begin{sol}
    Hola :p
\end{sol}